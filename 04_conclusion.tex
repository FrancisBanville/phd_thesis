%%
%% This is file `04_conclusion.tex',
%% generated with the docstrip utility.
%%

% Utilisez la macro de langue appropriée.
\francais   
%ou
%%\anglais
 
\chapter{Conclusion générale}

Cette thèse pose les fondements de la théorie de l'entropie maximale des réseaux
trophiques (Trophique-METE). Basée sur le principe d'entropie maximale, cette
théorie permet d'inférer des distributions de probabilité caractérisant la
structure émergente des réseaux trophiques. Les distributions prédites sont
celles qui représentent le mieux les informations fournies par les contraintes
écologiques, sans faire de supposition additionnelle sur la forme de la
distribution au-delà des contraintes choisies. Plusieurs versions de la théorie
peuvent être développées selon les contraintes utilisées et les distributions
prédites. Cette thèse développe et teste deux versions restreintes de la théorie
(article~\ref{art:article2}). 

L'article~\ref{art:article1} constitue le cadre conceptuel général de la
théorie. Il explique pourquoi les interactions entre espèces, telles que les
interactions prédateurs-proies et plantes-herbivores, sont intrinsèquement
probabilistes. Notre manque de connaissances sur les interactions locales et
régionales, en partie attribuable aux limites d'observation des interactions
entre espèces (\cite{Jordano2016Sampling}), ainsi que la variabilité spatiale et
temporelle des interactions locales (\cite{Poisot2015Species}), introduisent une
incertitude dans notre mesure des interactions. Cette incertitude est
irréductible à l'échelle locale, c.-à-d. qu'elle persiste même après l'ajout de
nouvelles données empiriques. Cet article souligne l'importance d'identifier, de
quantifier et de communiquer cette incertitude inhérente aux interactions entre
espèces. 

Trophique-METE fournit un cadre d'analyse permettant de quantifier plus
facilement et avec cohérence l'incertitude au sein des réseaux trophiques à
partir d'un nombre limité d'informations écologiques. Elle s'inscrit donc dans
la vision probabiliste des réseaux élaborée dans l'article~\ref{art:article1} en
générant des prédictions probabilistes de différentes mesures de la structure
des réseaux trophiques. En effet, puisque l'incertitude des interactions entre
espèces se propage à la structure émergente des réseaux, cette dernière est
également probabiliste. En approfondissant notre compréhension des interactions
probabilistes, nous pouvons mieux interpréter les prédictions de la théorie
(comme la distribution jointe de degrés, qui décrit la probabilité qu'une espèce
ait un nombre donné de proies et de prédateurs). Les sources d'incertitude et
mécanismes écologiques sous-jacents aux interactions entre espèces, décrits dans
l'article~\ref{art:article1}, peuvent inspirer le développement de différentes
versions de la théorie, p. ex. en prenant comme variables d'état celles qui
conditionnent les probabilités d'interactions entre espèces. Puisque les
prédictions de Trophique-METE ne sont basées que sur ces variables d'état, cette
théorie infère la structure des réseaux sans supposer (à tort) que les
interactions entre espèces sont indépendantes les unes des autres, ce qui
contribue à la robustesse de la théorie.

En reconnaissant que la structure émergente des réseaux trophiques est
écologiquement et statistiquement contrainte, l'article~\ref{art:article2} jette
les bases de la théorie de l'entropie maximale des réseaux trophiques. Il montre
deux approches (analytique et heuristique) pour prédire la structure des réseaux
trophiques à l'aide du principe d'entropie maximale et des contraintes imposées
par les variables d'état. L'approche analytique permet d'inférer directement une
distribution de probabilité à l'aide de la méthode des multiplicateurs de
Lagrange, alors que l'approche heuristique permet d'identifier le réseau
d'entropie (ou de complexité) maximale en permutant les interactions entre
espèces selon la méthode du recuit simulé. Les deux versions de la théorie
développées dans cet article, qui diffèrent selon les variables d'état
utilisées, mettent en œuvre ces deux approches. Cet article fournit également
les outils nécessaires pour développer d'autres versions de la théorie reposant
sur d'autres variables d'état. Dans les sous-sections suivantes, je discute des
développements actuels et futurs, des tests à venir et des applications
potentielles de la théorie. 

%% Développer la théorie

\section{Développement de la théorie de l'entropie maximale des réseaux trophiques} 

\subsection{Où en sommes-nous?} 

L'article~\ref{art:article2} propose deux premières versions de la théorie,
employant les approches analytique et heuristique basées sur le principe
d'entropie maximale. La première version de Trophique-METE prédit la structure
des réseaux trophiques à partir du nombre d'espèces et du nombre d'interactions.
Cette version a permis de prédire la distribution jointe de degrés (approche
analytique) et la matrice d'adjacence utilisée pour calculer différentes mesures
de la structure des réseaux (approche heuristique). La deuxième version, quant à
elle, prédit la matrice d'adjacence à partir du nombre de proies et de
prédateurs pour chaque espèce dans le réseau (approche heuristique). La première
version prédit bien la distribution jointe de degrés lorsque la connectance du
réseau est élevée, alors que la seconde version prédit mieux les autres mesures
testées (comme l'emboîtement et la proportion des motifs à trois espèces).
Cependant, ces deux versions surestiment le niveau trophique maximal et le
diamètre des réseaux, ce qui suggère que les variables utilisées ne capturent
pas adéquatement certains mécanismes importants sous-jacents aux réseaux,
notamment en ce qui a trait au transfert d'énergie le long de chaînes
trophiques. Malgré cela, les réseaux trophiques empiriques sont proches de leur
entropie maximale tel que prédit par mes modèles, ce qui suggère que les
variables utilisées parviennent tout de même à bien capturer la complexité des
réseaux trophiques, et par conséquent, une grande partie des mécanismes
écologiques sous-jacents aux réseaux. 

\subsection{Comment améliorer et étendre la théorie?} 

Différentes versions de la théorie de l'entropie maximale des réseaux trophiques
peuvent être développées selon les variables d'état choisies. La sélection des
variables d'état peut se faire sur la base des limites et erreurs de prédiction
des versions précédentes. Par exemple, puisque les versions élaborées dans
l'article~\ref{art:article2} n'ont pas réussi à prédire adéquatement le niveau
trophique maximal et le diamètre des réseaux, une version contraignant la
longueur des chaînes trophiques pourrait s'avérer plus précise. Une telle
version pourrait utiliser l'énergie métabolique totale de la communauté comme
variable d'état (imposant une limite au transfert d'énergie au sein des réseaux)
ou le niveau trophique moyen comme contrainte écologique. Bien que le
développement de telles versions améliorées de la théorie puisse permettre
d'obtenir de meilleures prédictions de la structure des réseaux trophiques,
elles se cantonnent à une seule représentation des réseaux (c.-à-d. aux réseaux
d'interactions binaires entre espèces), sans tenir compte des différents niveaux
d'organisation et de mesure des interactions.

Pour être plus complète, une théorie élargie de l'entropie maximale des réseaux
trophiques devrait pouvoir prédire simultanément différentes représentations des
réseaux trophiques. Cette théorie élargie serait l'extension logique de la
théorie de l'entropie maximale de l'écologie (METE, \cite{Harte2011Maximum}) aux
réseaux trophiques (d'où l'appellation «~Trophique-METE~»). La version ASNE de
METE utilise la superficie $A_0$, le nombre d'espèces $S_0$, le nombre total
d'individus $N_0$ et l'énergie métabolique totale $E_0$ (ou la biomasse) d'une
communauté comme variables d'état pour prédire plusieurs distributions d'intérêt
en écologie (\cite{Harte2008Maximum}, \cite{Harte2011Maximum},
\cite{Harte2014Maximum}). Trophique-METE, dans sa version élargie, pourrait
notamment ajouter le nombre total d'interactions entre espèces $L_0$ à ces
variables d'état pour former la version ASNEL de la théorie. En plus des
prédictions macroécologiques de METE (telles que la relation aire-espèce et les
distributions du nombre d'individus et du taux métabolique par espèce), ASNEL
permettrait de prédire la relation entre la structure des réseaux et leur
superficie, tout en faisant le pont entre les réseaux d'interactions entre
individus et entre espèces. Cette théorie élargie permettrait ainsi d'avoir une
vision plus harmonieuse et exhaustive des réseaux trophiques représentés à
différentes échelles spatiales et taxonomiques.

Les versions élargies de Trophique-METE peuvent être construites autour de
plusieurs autres variables d'état pour tenir compte des conditions locales
mentionnées dans l'article~\ref{art:article1}. Par exemple, le temps $t_0$ (qui
peut correspondre, entre autres, à la durée d'échantillonnage ou au temps écoulé
depuis le début de la succession primaire) peut être ajouté aux variables
d'état, ce qui permettrait de prédire la relation entre la structure des réseaux
et leur durée. Le temps pourrait également être intégré dans une version
dynamique de la théorie en laissant les valeurs des variables d'état fluctuer au
fil du temps, tel qu'effectué par \textcite{Harte2021Dynamete} dans leur version
dynamique de METE. Par ailleurs, le nombre de genres ou de familles au sein
d'une communauté peut également être ajouté aux variables d'état pour dériver
davantage de propriétés liées au niveau taxonomique, comme la relation
aire-genre ou aire-famille (\cite{Harte2014Maximum}). Dans le même ordre
d'idées, le nombre total d'interactions entre individus, genres ou familles peut
être intégré à la théorie pour prédire plus précisément la structure des réseaux
à différentes échelles taxonomiques. Le nombre total d'interactions entre
individus (c.-à-d. le nombre total d'événements de prédation survenus au cours
d'une période de temps donné) peut également être utilisé pour prédire la
structure des réseaux d'interactions quantitatives (p. ex. pour dériver la
distribution de fréquences des interactions entre espèces). Le principal
obstacle à la construction d'une telle théorie élargie des réseaux trophiques
est le manque actuel de données empiriques nécessaires pour tester et valider
différentes versions de la théorie. Cependant, poursuivre activement le
développement de Trophique-METE nous permet de mieux identifier les données et
variables d'état à échantillonner. 

Quelle que soit la version développée, les prédictions de Trophique-METE
demeurent intrinsèquement probabilistes. Cependant, la matrice d'adjacence
prédite pour un réseau particulier, avec la méthode présentée dans
l'article~\ref{art:article2}, est unique et composée d'interactions binaires.
Cela est dû à l'approche heuristique employée, qui maximise l'entropie de la
distribution des valeurs singulières de la matrice d'adjacence, permettant ainsi
d'identifier celle ayant la plus grande complexité interne. Trophique-METE peut
également être développée pour inférer analytiquement des distributions
d'entropie maximale \textit{sur} les réseaux (c.-à-d. générant une distribution
de réseaux probabilistes tels que définis dans l'article~\ref{art:article1}).
Ces distributions peuvent être obtenues en utilisant des contraintes rigides (où
chaque réseau ayant une probabilité non nulle satisfait exactement les
contraintes) ou souples (où les réseaux prédits satisfont les contraintes en
moyenne). Mesurer la structure des réseaux à partir de ces distributions
d'entropie maximale nous permettrait de générer des prédictions probabilistes
pour l'ensemble des propriétés des réseaux. Le développement de Trophique-METE
autour des réseaux probabilistes constitue une avancée naturelle dans la
théorie, nous permettant d'aller au-delà des prédictions uniques des réseaux,
tout en offrant une alternative aux réseaux \textit{d'interactions} probabilistes.  

%% Tester la théorie

\section{Validation de la théorie} 

En tant que théorie écologique en cours de développement, Trophique-METE doit
être testée à répétition et validée par des observations empiriques fiables.
Comme indiqué plus haut, ce processus itératif de validation permet de
développer des versions plus robustes et complètes de la théorie. Cette thèse
réalise la première évaluation de la théorie en la testant sur des données
empiriques recueillies à l'échelle globale, tant en milieu aquatique que
terrestre (article~\ref{art:article2}). Lorsqu'elles surviennent, les erreurs de
prédiction peuvent provenir des variables d'état utilisées, qui peuvent être mal
mesurées, inadéquates ou insuffisantes. Cependant, plusieurs questions relatives
à l'évaluation de la théorie restent en suspens, notamment en ce qui a trait aux
particularités et à l'équilibre écologique des réseaux trophiques d'entropie
maximale (c.-à-d. des réseaux dont la complexité est bien prédite par la
théorie). En approfondissant notre compréhension des caractéristiques des
réseaux trophiques d'entropie maximale, nous pourrons mieux définir les champs
d'application et limites de la théorie.

\subsection{Quelles sont les particularités des réseaux trophiques d'entropie maximale?} 

Jusqu'à présent, Trophique-METE a été développée pour l'ensemble des réseaux
trophiques, sans tenir compte des caractéristiques propres aux différents types
d'écosystèmes. Autrement dit, les prédictions de la théorie sont identiques pour
tout écosystème disposant des mêmes valeurs de variables d'état, indépendamment
de leurs particularités écologiques. Toutefois, les écosystèmes pourraient être
contraints par différentes variables d'état, selon les mécanismes écologiques
qui les régissent. Par exemple, nous pourrions tester si la biomasse totale
contraint autant les réseaux trophiques aquatiques que terrestres. Les
caractéristiques des écosystèmes, telles que leur habitat
(\cite{Shurin2005All}), leur niveau de perturbation anthropique
(\cite{Tylianakis2007Habitat}) et leur superficie (\cite{Galiana2018Spatiala}),
peuvent en effet impacter les processus écologiques sous-jacents à leur
structure émergente et, par conséquent, la capacité des variables d'état
utilisées à capturer adéquatement ces processus. De même, adapter la théorie de
l'entropie maximale des réseaux trophiques à d'autres types de réseaux (comme
aux réseaux d'interactions hôtes-parasites ou plantes-pollinisateurs)
impliquerait d'identifier les variables d'état qui les déterminent
spécifiquement, lesquelles pourraient différer de celles gouvernant les réseaux
trophiques. Les particularités écologiques des réseaux moins bien prédits par
une version donnée de la théorie définissent les limites de celle-ci et peuvent
nous guider dans le développement de différentes versions mieux adaptées à
divers contextes écologiques.

Au-delà des caractéristiques des écosystèmes, la représentation des réseaux
trophiques peut également impacter la capacité de la théorie à faire de bonnes
prédictions. Dans cette thèse, j'ai montré que le principe d'entropie maximale
prédit bien la structure des réseaux d'interactions binaires entre espèces
(article~\ref{art:article2}). Le développement et la validation d'une théorie
élargie des réseaux trophiques aideront à déterminer si MaxEnt est plus efficace
pour prédire la structure des réseaux à un niveau taxonomique (interactions
entre individus, populations, espèces ou clades) et à une échelle de mesure
(interactions binaires ou quantitatives) particuliers, ou s'il prédit
convenablement l'ensemble de ces représentations. Tester la théorie sur
plusieurs types de réseaux trophiques, qui présentent différents niveaux de
complexité écologique, est essentiel pour bien comprendre ce qui détermine la
structure de différentes représentations des systèmes écologiques complexes. 

\subsection{Un réseau d'entropie maximale est-il à l'équilibre?} 

Le temps requis pour que les réseaux trophiques atteignent leur entropie
maximale, dans la limite des contraintes imposées par les variables d'état,
n'est pas bien compris. Nous pouvons supposer que les écosystèmes en transition,
qui n'ont pas eu assez de temps pour atteindre un état d'équilibre à la suite
d'une perturbation écologique, sont moins susceptibles d'être près de leur
entropie maximale. En effet, les prédictions de METE sont plus robustes pour les
communautés stables, c.-à-d. pour les communautés dont les variables d'état ne
fluctuent pas trop rapidement (\cite{Newman2020Disturbance},
\cite{Harte2021Dynamete}). Cela s'explique par le fait que les variables d'état
et les distributions prédites sont évaluées pour le même instant dans le temps.
Lorsque la valeur des variables d'état change trop rapidement, laissant un délai
insuffisant pour permettre aux distributions de s'ajuster, l'exactitude des
prédictions est compromise. Puisque Trophique-METE est aussi basée sur la valeur
instantanée des variables d'état, il convient de vérifier si la validité de ses
prédictions ne tient également que pour les écosystèmes dont les variables
d'état sont stables.

Les réseaux trophiques doivent également respecter le principe de conservation
de biomasse, selon lequel le bilan global entre les flux intrants et sortants de
matière est à l'équilibre. Ce principe s'applique autant à l'écosystème dans son
ensemble qu'à chaque espèce individuelle, puisque le taux moyen de production
d'énergie d'une espèce doit être équivalent à son taux moyen d'utilisation
(croissance, métabolisme et perte) et de consommation par les autres espèces
(\cite{Sterner2002Ecological}). La validation d'une version élargie de
Trophique-METE, qui utiliserait notamment l'énergie métabolique totale de la
communauté comme variable d'état, peut donc en partie être effectuée en
vérifiant le respect du principe de conservation de biomasse à l'échelle de
l'écosystème et de l'espèce. Pour ce faire, la théorie doit prédire les
distributions du taux d'utilisation d'énergie et du nombre d'individus par
espèce, ainsi que de la force d'interaction entre prédateurs et proies. Elle
peut toutefois s'appuyer sur des relations allométriques existantes, comme la
relation entre l'énergie métabolique et la biomasse fournie par la théorie
métabolique de l'écologie (\cite{Brown2004Metabolic}, \cite{West1997General}),
pour estimer les paramètres manquants, le cas échéant (tel qu'effectué par
\cite{Harte2022Equation} dans leurs efforts d'unification de METE avec la
théorie métabolique de l'énergie). Cette méthode de validation permettrait à la
fois de vérifier la cohérence interne et le réalisme écologique de
Trophique-METE, tout en permettant d'évaluer sa compatibilité avec les relations
allométriques acceptées en écologie.


%% Appliquer la théorie

\section{Applications potentielles de la théorie} 

La théorie de l'entropie maximale des réseaux trophiques nous aide à élucider
les déterminants des réseaux trophiques.

cohérentes sur les phénomènes observés. Elles doivent 
générer un ensemble de pré


Trophique-METE 



La théorie de l'entropie maximale des réseaux trophiques fournit une explication
étayée des déterminants et propriétés des réseaux trophiques. 

Une théorie scientifique fournit une explication approfondie et étayée d'un phénomène, 
soutenue par des données probantes. Elle génère 
des prédictions vérifiables sur des choses qui n'ont pas encore été observées.

a pour but d'unifier
nos connaissances sur les réseaux trophiques en offrant une compréhension
globale et cohérente de leur structure émergente. Elle y parvient en prédisant
les propriétés émergentes des réseaux trophiques, les mécanismes sous-jacents
aux réseaux pouvant être déduits des variables d'état utilisées et des erreurs
de prédiction de la théorie. Ces prédictions sont facilitées par la nature
parcimonieuse de la théorie, qui ne repose que sur un ensemble limité
d'informations écologiques. Le rôle de Trophique-METE est donc double~: elle
permet de mieux comprendre les déterminants des réseaux trophiques et de générer
des prédictions vérifiables sur des réseaux qui n'ont pas encore été observés.

\subsection{Compréhension des mécanismes sous-jacents aux réseaux trophiques} 

Bien que Trophique-METE ne soit pas basée sur des mécanismes écologiques
explicites, elle facilite l'identification des mécanismes régissant la structure
des réseaux trophiques. D'abord, tel qu'expliqué par
\textcite{Harte2014Maximum}, une théorie écologique basée sur l'entropie
maximale tient compte des mécanismes écologiques de manière implicite par le
biais de nos connaissances préalables. En effet, les mécanismes sous-jacents à
la structure des réseaux trophiques sont capturés par la valeur empirique des
variables d'état. Ainsi, pour mieux comprendre les mécanismes qui déterminent la
structure des réseaux trophiques, nous pouvons examiner ceux qui régissent les
variables d'état. Ensuite, les erreurs de prédiction de la théorie nous
informent des mécanismes importants n'ayant pas été capturés par les variables
d'état. Trophique-METE peut donc être utilisée comme modèle nul, fournissant des
distributions de référence auxquelles la structure des réseaux empiriques est
comparée. À cet égard, les écarts entre la structure des réseaux empiriques et
prédits peuvent fournir des informations écologiques plus révélatrices que les
mesures brutes. Lorsque ces erreurs sont systématiques (comme ce fut le cas pour
le niveau trophique maximal et le diamètre des réseaux dans
l'article~\ref{art:article2}), elles peuvent guider le développement de versions
plus robustes et précises de la théorie, permettant ainsi d'identifier les
variables d'état capturant plus efficacement les mécanismes écologiques
sous-jacents. 

\subsection{Prévision de la structure des réseaux trophiques} 

Nécessitant très peu de données empiriques préalables, Trophique-METE peut
faciliter la prédiction des réseaux trophiques dans le temps et l'espace. Selon
le niveau de validation de la version employée, les distributions prédites
peuvent être utilisées comme distributions informatives \textit{a priori}
(c.-à-d. comme distributions mises à jour par de nouvelles données empiriques
dans des analyses bayésiennes de la structure des réseaux) ou comme prédictions
directes de la structure des réseaux. Une version de la théorie bien soutenue
empiriquement peut simplifier l'étude des réseaux trophiques sur de vastes
étendues spatiales en générant des prédictions pour différentes communautés dont
nous connaissons la valeur des variables d'état. De même, Trophique-METE peut
être utilisée pour prévoir les changements futurs dans la structure des réseaux
trophiques. Pour ce faire, elle peut utiliser les projections des variables
d'état à des moments précis dans le temps selon différents scénarios climatiques
et environnementaux (p. ex. la projection de la richesse spécifique d'une
communauté lors de la prochaine décennie ou la projection de la superficie
occupée par un écosystème forestier après un projet d'expansion urbain). Cette
prévision suppose que les nouvelles valeurs des variables d'état ont atteint un
nouvel équilibre stable, sauf si une version dynamique de Trophique-METE est
utilisée pour les écosystèmes en transition. Dans un contexte où les changements
climatiques et la perte des habitats naturels causés par les activités humaines
impactent la distribution et la biologie des espèces, une théorie des réseaux
trophiques permettant de prédire leur structure émergente dans le temps et
l'espace peut nous aider à mieux anticiper les effets de ces bouleversements sur
le fonctionnement des milieux naturels.

\raggedleft

\
\

\textit{« The more clearly we can focus our attention on the wonders and
realities of the universe about us, the less taste we shall have for
destruction. »} 
  

– Rachel Carson, The real world around us\footnote{Discours prononcé lors d'un sommet des femmes journalistes, Theta Sigma Phi} (1954)

\endinput
%%


%% End of file `04_conclusion.tex'.
