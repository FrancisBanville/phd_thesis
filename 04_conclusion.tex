%%
%% This is file `04_conclusion.tex',
%% generated with the docstrip utility.
%%

% Utilisez la macro de langue appropriée.
\francais   
%ou
%%\anglais

\chapter{Conclusions générales}

%% Développer la théorie

Cette thèse pose les fondements de la théorie de l'entropie maximale des réseaux
trophiques (Trophique-METE). Basée sur le principe d'entropie maximale, cette
théorie permet d'inférer des distributions de probabilité caractérisant la
structure émergente des réseaux trophiques. Les distributions prédites sont
celles qui représentent le mieux les informations fournies par les contraintes
écologiques, sans faire de supposition additionnelle sur la forme de la
distribution au-delà des contraintes choisies. Plusieurs versions de la théorie
peuvent être développées selon les contraintes utilisées et les distributions
prédites. Cette thèse développe et teste deux versions restreintes de la théorie
(Article~\ref{art:article2}). 

L'article~\ref{art:article1} constitue le cadre conceptuel de la théorie. Il
explique pourquoi les interactions entre espèces, telles que les interactions
prédateurs-proies et plantes-herbivores, sont intrinsèquement probabilistes.
Notre manque de connaissances sur les interactions locales et régionales, en
partie attribuable aux limites d'observation des interactions entre espèces
(\cite{Jordano2016Sampling}), ainsi que la variabilité spatiale et temporelle
des interactions locales (\cite{Poisot2015Species}), introduisent une
incertitude dans notre mesure des interactions. Cette incertitude est
irréductible à l'échelle locale, c.-à-d. qu'elle persiste même avec l'ajout de
nouvelles données empiriques. Cet article souligne l'importance d'identifier, de
quantifier et de communiquer cette incertitude inhérente aux interactions entre
espèces. 

Trophique-METE fournit un cadre d'analyse permettant de quantifier plus
facilement et avec cohérence l'incertitude au sein des réseaux trophiques à
partir d'un nombre limité d'informations écologiques. Elle s'inscrit donc dans
la vision probabiliste des réseaux élaborée dans l'article~\ref{art:article1} en
générant des prédictions probabilistes de différentes mesures de la structure
des réseaux trophiques. En effet, puisque l'incertitude des interactions entre
espèces se propage à la structure émergente des réseaux, cette dernière est
également probabiliste. En approfondissant notre compréhension des interactions
probabilistes, nous pouvons mieux interpréter les prédictions de la théorie
(comme la distribution jointe de degrés, qui décrit la probabilité qu'une espèce
ait un nombre donné de proies et de prédateurs). Les sources d'incertitude et
mécanismes écologiques sous-jacents aux interactions entre espèces, décrits dans
l'article~\ref{art:article1}, peuvent inspirer le développement de différentes
versions de la théorie, p. ex. en prenant comme variables d'état celles qui
conditionnent les probabilités d'interactions entre espèces. Puisque les
prédictions de Trophique-METE ne sont basées que sur ces variables d'état, cette
théorie infère la structure des réseaux sans supposer (à tort) que les
interactions entre espèces sont indépendantes les unes des autres, ce qui
contribue à la robustesse de la théorie.

En reconnaissant que la structure émergente des réseaux trophiques est
écologiquement et statistiquement contrainte, l'article~\ref{art:article2} jette
les bases de la théorie de l'entropie maximale des réseaux trophiques. Il montre
deux approches (analytique et heuristique) pour prédire la structure des réseaux
trophiques à l'aide du principe d'entropie maximale. L'approche analytique
permet d'inférer directement une distribution de probabilité à l'aide de la
méthode des multiplicateurs de Lagrange, alors que l'approche heuristique permet
d'identifier le réseau d'entropie (ou de complexité) maximale en permutant
aléatoirement les interactions entre espèces tout en respectant les contraintes
imposées par les variables d'état. Les deux versions de la théorie développées
dans cet article, qui diffèrent selon les variables d'état utilisées, mettent en
œuvre ces deux approches. Cet article fournit également les outils nécessaires
pour développer d'autres versions de la théorie reposant sur d'autres variables
d'état. Dans les sous-sections suivantes, je discute des développements actuels
et futurs, des tests à venir et des applications potentielles de la théorie. 

\section{Développement de la théorie de l'entropie maximale des réseaux trophiques} 

\subsection{Où en sommes-nous?} 

L'article~\ref{art:article2} propose deux premières versions de la théorie,
employant les approches analytiques et heuristiques basées sur le principe
d'entropie maximale. La première version de trophique-METE prédit la structure
des réseaux trophiques à partir du nombre d'espèces et du nombre d'interactions.
Cette version a permis de prédire la distribution jointe de degrés (approche
analytique) et la matrice d'adjacence utilisée pour calculer différentes mesures
de la structure des réseaux (approche heuristique). La deuxième version, quant à
elle, prédit la matrice d'adjacence à partir du nombre de proies et de
prédateurs pour chaque espèce dans le réseau (approche heuristique). La première
version prédit bien la distribution jointe de degrés lorsque la connectance du
réseau est élevée, alors que la seconde version prédit mieux les autres mesures
testées (comme l'emboîtement). Cependant, ces deux versions surestiment
systématiquement le niveau trophique maximal, ce qui suggère que les variables
utilisées ne capturent pas adéquatement certains mécanismes importants
sous-jacents aux réseaux, notamment en ce qui a trait au transfert d'énergie au
sein des systèmes écologiques complexes. Malgré cela, les réseaux trophiques
empiriques sont proches de leur entropie maximale tel que prédit par mes
modèles, ce qui suggère que les variables utilisées parviennent tout de même à
bien capturer la complexité des réseaux trophiques, et par conséquent, une
grande partie des mécanismes écologiques sous-jacents aux réseaux. 

\subsection{Comment améliorer et étendre la théorie?} 

Différentes versions de la théorie de l'entropie maximale des réseaux trophiques
peuvent être développées selon les variables d'état choisies. La sélection des
variables d'état peut se faire sur la base des limites et erreurs de prédiction
des versions précédentes. Par exemple, puisque les versions élaborées dans
l'article~\ref{art:article2} n'ont pas réussi à prédire adéquatement le niveau
trophique maximal et le diamètre des réseaux, une version contraignant la
longueur des chaînes trophiques pourrait s'avérer plus précise. Une telle
version pourrait utiliser l'énergie métabolique totale de la communauté comme
variable d'état (imposant une limite au transfert d'énergie au sein des réseaux)
ou le niveau trophique moyen comme contrainte écologique. Bien que le
développement de telles versions améliorées de la théorie puisse permette
d'obtenir de meilleures prédictions de la structure des réseaux trophiques,
elles se cantonnent à une seule représentation des réseaux (p. ex. aux réseaux
d'interactions binaires entre espèces), sans tenir compte des différents niveaux
d'organisation et de mesure des interactions dans les réseaux.

Pour être plus complète, une théorie élargie de l'entropie maximale des réseaux
trophiques devrait pouvoir prédire simultanément différentes représentations des
réseaux trophiques. Cette théorie élargie serait l'extension logique de la
théorie de l'entropie maximale de l'écologie (METE, \cite{Harte2011Maximum}) aux
réseaux trophiques (d'où l'appellation «~Trophique-METE~»).


\begin{itemize}
    \item ANSEL 
    \item Champs d'application 
    \item Prédire les réseaux quantitatifs
    \item Dériver la distribution probabiliste MaxEnt des réseaux
    \item MaxEnt pour prédire la structure des réseaux dans l'espace
    \item MaxEnt pour prévoire (forecast) la structure des réseaux dans le temps
    \item MaxEnt pour générer des priors informatifs 
    \item Indépendance entre in et out degree
    \item Pour mieux prédire les interactions entre paires d'espèces
\end{itemize}

The degree distributions and probabilistic food webs that will be derived in
Chapter 1 may dismiss relevant ecological information easily measurable in many
communities. The total area $A_o$ of a community is such meaningful measure.
Network-area relationships (NARs) describe the spatial scaling of many
ecological network properties [ThomTown05; WoodRuss15a; GaliLurg18a]; and
according to MacDBanv20b, many different NARs are generated from the scaling of
species richness with area and from the scaling of the number of links with
species richness. In Chapter 1, I will have shown how spatial scale impacts
worldwide predictions of food-web structure. Another community measure relevant
to network ecology is the total number of individuals $N_0$, which has been
shown to have a strong effect on the nestedness of mutualistic networks
[KrisJr08]. On a different level, pairwise species interactions could also be a
result of species abundances due to neutral or stochastic processes [Vazq05a;
VazqMeli07; CoelRang18; but see OlitFox15]. Moreover, the total metabolic
energy $E_0$ of a community (or its total biomass) could also impact food-web
structure. For instance, the body mass of two species could predict the strength
of trophic interactions [Anon05; BerlDunn09]. These measures were used in the
ASNE version of the maximum entropy theory of ecology [HartZill08;
HartNewm14a], along with species richness $S_0$, to derive macroecological
metrics of maximum entropy. METE notably derives the species-abundance
distribution (SAD), the species-area relationship (SAR), the intra-specific
spatial abundance distribution (ISSAD), and the intra-specific energy
distribution (ISED). 

In Chapter 2, in contrast with Chapter 1, I will derive food-web structure using
the principle of maximum entropy and 5 state variables. Those state variables
are the area $A$, the total number of individuals $N$, the number of species
$S$, the total energetic requirement of the community $E$, and the number of
interactions $L$. Using $N$ and $E$ will allow me to predict many properties of
weighted food webs. The main objective of this chapter will be to develop a
novel version (ANSEL) of the maximum entropy theory of ecology (METE) applied to
food webs. This version will be called *Trophic-METE*.

Because Trophic-METE will necessitate data often difficult to sample or
estimate, it will not be used to map food-web structure at a large spatial scale
as in Chapter 1. Rather, I will model specific and well-defined communities for
which those variables have been measured or estimated. When not measured
empirically, the number of links will again be estimated using species richness.
My model will be used to predict food-web structure with, as I suspect, more
accuracy than the model of the first chapter, as long as communities are not in
a transition state and that the state variables have been measured or estimated
accurately. As in Chapter 1, the maximum entropy distributions could be used as
informative prior distributions in Bayesian inference, and deviations from these
predictions could be used to identify ecological mechanisms leading to observed
network structures.

Chapter 2 aims at bringing together predictive models of food-web structure and
the derivations of METE to derive a distribution of feasible and plausible
networks, from which measures could be computed. In that respect, the main
objective of this chapter is to derive and validate another version of METE
(namely Trophic-METE) applied to the spatial prediction of food webs. I believe
that the SAD, the SAR, the ISSAD, the ISED, and the joint degree distribution of
maximum entropy will provide sufficient information to predict accurately the
probability of the interaction strength between two species $i$ and $j$, i.e.
$P(w_{ij}|A, n_i, n_j, S_0, e_i, e_j, k_i, k_j)$, or, more generally, the
probability of a weighted network $W$ given the values of the state variables,
i.e. $P(W|A, N_0, S_0, E_0, L_0)$. Again, these predictions should be more
precise for more stable communities, in contrast to communities undergoing rapid
changes.  

These predictions could be used in a similar way as the ones of the first
chapter. Notably, my model could be used to predict the food web of given
communities. However, in contrast to Chapter 1, the model of Chapter 2 will not
be built to generate first-order approximations of food-web structure at a large
spatial scale, i.e. for communities in each grid of a map, unless $N_0$ and
$E_0$ are estimated rigorously at the given spatial resolution. Rather, I expect
that my model could be used to predict the network of more well-defined
community. For example, it could be used to predict the food web of a given
forest patch if values of $A_0$, $S_0$, $N_0$, and $E_0$ are well sampled. If my
model generates accurate predictions of food webs, it could greatly facilitate
the conduction of *preliminary* studies on the functioning and stability of
ecological communities, without directly relying on data on species
interactions. In a related way, it could be used to generate informative prior
distributions of food-web structure that could be updated with empirical data.
Finally, my model could be used as a null model to investigate the role of
ecological mechanisms in the shaping of food-web topology.

My model could *a priori* be applied on any taxonomic groups or at any level of
organization, as long as state variables and derived metrics are defined
coherently. I however hypothesize that my model will be more robust when used
with a set of taxons that encompasses all trophic levels (i.e. comprising
autotrophs, herbivores, and carnivores), so that the modeled community is well
represented. On another note, I believe that my model will be suitable to any
spatial scales. However, it is not clear to me how it will perform with very
small or very large communities.

In this chapter, I will derive a distribution of feasible weighted networks
using the area $A_0$, the number of species $S_0$, the total number of
individuals $N_0$, the total energetic requirement of the community $E_0$, and
the (predicted) number of links $L_0$. 

For communities where $L_0$ has not been measured or estimated empirically, I
will estimate $L_0$ from $S_0$ using the flexible links model of MacDBanv20b.
Similarly, for communities where $E_0$ has not been estimated empirically, but
for which the total biomass $M_0$ has been estimated, I will estimate $E_0$ from
$M_0$ using the metabolic theory of ecology [MTE; Klei32; BrowGill04; and
Hart11]:



$$E_o = cM_o^{3/4}$$

where $c$ is a positive constant whose value can be estimated empirically.

These predictions (eq:SAD to eq:ISED) could be used directly to predict
weighted food webs, along with the joint degree distribution of maximum entropy
derived in Chapter 1 (eq:maxentjoint). This would constitute the ANSEL version
of the maximum entropy theory of ecology (Trophic-METE). However, I will attempt
to derive other metrics directly. One of these metrics could be the interaction
strength distribution $P(w_{ij}|A, n_i, n_j, S_o, e_i, e_j, k_i, k_j)$, where
$P(w_{ij})$ is the probability density that if two species are taken at random
from the species pool, that they will interact and that this interaction will
have a strength of $w_{ij}$. Note that this probability density function might
depend only on a subset of the ANSEL variables. Another approach could be to
attempt to predict emerging network properties, such as the distribution of
trophic levels in a food web, from the five state variables.

These predictions will be compared to observed data using ranking plots. To do
so, I might have to estimate a few state variables in existing datasets in order
to conduct this validation process.

I will use the distributions derived from Trophic-METE to constrain the space of
feasible networks. More specifically, I will predict the probability
distribution $P(W|A, N_0, S_0, E_0, L_0)$ given the values of the state
variables. To do so, I might use simulating annealing algorithms to filter out
potential adjacency matrices that do not respect the derived distributions. Note
that $P(W)$ will take into account the uncertainty around $L_0$ for networks
whose number of links will be predicted from species richness. I believe,
however, that my version of METE will be more appealing if I can predict $P(W)$
analytically from these distributions, probably using some approximations. This
might however prove very challenging, and perhaps unfeasible as part of my
thesis.

Emerging food-web properties, such as nestedness and modularity, could be
computed for all potential food webs in the distribution of feasible networks.
The resulting distribution of food-web properties will take into account the
likelihood of every potential network. These predictions will then be compared
with empirical data.

The probability distribution of weighted food webs derived in this chapter will
not be the result of any mechanistic or ecological processes. It will instead
characterize the networks we expect to see, by chance, given values of the state
variables ($A_0$, $N_0$, $S_0$, $E_0$, and $L_0$). This network-centered version
of the maximum entropy theory of ecology will make testable predictions that
will be compared with empirical data. Deviations from the maximum entropy degree
distributions will be ecologically informative. However, if I find systematic
deviations from my predictions, I would need to revisit my model, either by
choosing a different set of state variables or by choosing different constraints
on the distributions. On another note, in contrast to the model of Chapter 1, I
believe that adapting Trophic-METE to other types of ecological networks would
require important changes in the choices of state variables and of constraints.

Overall, Trophic-METE could be very useful to derive least-biased distributions
of food-web structure and of weighted interactions, given prior knowledge on a
biological community. These least-biased distributions could be used as prior
distributions, or could be used directly to estimate various emerging food-web
properties. Predictions will be snapshots of an ecosystem, and knowing the value
of the state variables at a given time point would allow me to predict its food
web at that time point. In the next and final chapter, I will take advantage of
this feature and use Trophic-METE, along with the model of Chapter 1, to
forecast food webs over time.

%% Tester la théorie

\section{Validation de la théorie} 

\subsection{Quels réseaux sont d'entropie maximale?} 

\subsection{Un réseau d'entropie maximale est-il à l'équilibre?} 


%% Appliquer la théorie

\section{Applications potentielles de la théorie} 

\subsection{Compréhension des mécanismes sous-jacents aux réseaux d'interactions} 

\subsection{Prévision de la structure des réseaux trophiques} 


\endinput
%%
%% End of file `04_conclusion.tex'.
