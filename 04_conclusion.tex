%%
%% This is file `04_conclusion.tex',
%% generated with the docstrip utility.
%%

% Utilisez la macro de langue appropriée.
\francais   
%ou
%%\anglais

\chapter{Conclusion générale}

Cette thèse pose les fondements de la théorie de l'entropie maximale des réseaux
trophiques (Trophique-METE). Basée sur le principe d'entropie maximale, cette
théorie permet d'inférer des distributions de probabilité caractérisant la
structure émergente des réseaux trophiques. Les distributions prédites sont
celles qui représentent le mieux les informations fournies par les contraintes
écologiques, sans faire de supposition additionnelle sur la forme de la
distribution au-delà des contraintes choisies. Plusieurs versions de la théorie
peuvent être développées selon les contraintes utilisées et les distributions
prédites. Cette thèse développe et teste deux versions restreintes de la théorie
(Article~\ref{art:article2}). 

L'article~\ref{art:article1} constitue le cadre conceptuel de la théorie. Il
explique pourquoi les interactions entre espèces, telles que les interactions
prédateurs-proies et plantes-herbivores, sont intrinsèquement probabilistes.
Notre manque de connaissances sur les interactions locales et régionales, en
partie attribuable aux limites d'observation des interactions entre espèces
(\cite{Jordano2016Sampling}), ainsi que la variabilité spatiale et temporelle
des interactions locales (\cite{Poisot2015Species}), introduisent une
incertitude dans notre mesure des interactions. Cette incertitude est
irréductible à l'échelle locale, c.-à-d. qu'elle persiste même avec l'ajout de
nouvelles données empiriques. Cet article souligne l'importance d'identifier, de
quantifier et de communiquer cette incertitude inhérente aux interactions entre
espèces. 

Trophique-METE fournit un cadre d'analyse permettant de quantifier plus
facilement et avec cohérence l'incertitude au sein des réseaux trophiques à
partir d'un nombre limité d'informations écologiques. Elle s'inscrit donc dans
la vision probabiliste des réseaux élaborée dans l'article~\ref{art:article1} en
générant des prédictions probabilistes de différentes mesures de la structure
des réseaux trophiques. En effet, puisque l'incertitude des interactions entre
espèces se propage à la structure émergente des réseaux, cette dernière est
également probabiliste. En approfondissant notre compréhension des interactions
probabilistes, nous pouvons mieux interpréter les prédictions de la théorie
(comme la distribution jointe de degrés, qui décrit la probabilité qu'une espèce
ait un nombre donné de proies et de prédateurs). Les sources d'incertitude et
mécanismes écologiques sous-jacents aux interactions entre espèces, décrits dans
l'article~\ref{art:article1}, peuvent inspirer le développement de différentes
versions de la théorie, p. ex. en prenant comme variables d'état celles qui
conditionnent les probabilités d'interactions entre espèces. Puisque les
prédictions de Trophique-METE ne sont basées que sur ces variables d'état, cette
théorie infère la structure des réseaux sans supposer (à tort) que les
interactions entre espèces sont indépendantes les unes des autres, ce qui
contribue à la robustesse de la théorie.

En reconnaissant que la structure émergente des réseaux trophiques est
écologiquement et statistiquement contrainte, l'article~\ref{art:article2} jette
les bases de la théorie de l'entropie maximale des réseaux trophiques. Il montre
deux approches (analytique et heuristique) pour prédire la structure des réseaux
trophiques à l'aide du principe d'entropie maximale. L'approche analytique
permet d'inférer directement une distribution de probabilité à l'aide de la
méthode des multiplicateurs de Lagrange, alors que l'approche heuristique permet
d'identifier le réseau d'entropie (ou de complexité) maximale en permutant
aléatoirement les interactions entre espèces tout en respectant les contraintes
imposées par les variables d'état. Les deux versions de la théorie développées
dans cet article, qui diffèrent selon les variables d'état utilisées, mettent en
œuvre ces deux approches. Cet article fournit également les outils nécessaires
pour développer d'autres versions de la théorie reposant sur d'autres variables
d'état. Dans les sous-sections suivantes, je discute des développements actuels
et futurs, des tests à venir et des applications potentielles de la théorie. 

%% Développer la théorie

\section{Développement de la théorie de l'entropie maximale des réseaux trophiques} 

\subsection{Où en sommes-nous?} 

L'article~\ref{art:article2} propose deux premières versions de la théorie,
employant les approches analytiques et heuristiques basées sur le principe
d'entropie maximale. La première version de Trophique-METE prédit la structure
des réseaux trophiques à partir du nombre d'espèces et du nombre d'interactions.
Cette version a permis de prédire la distribution jointe de degrés (approche
analytique) et la matrice d'adjacence utilisée pour calculer différentes mesures
de la structure des réseaux (approche heuristique). La deuxième version, quant à
elle, prédit la matrice d'adjacence à partir du nombre de proies et de
prédateurs pour chaque espèce dans le réseau (approche heuristique). La première
version prédit bien la distribution jointe de degrés lorsque la connectance du
réseau est élevée, alors que la seconde version prédit mieux les autres mesures
testées (comme l'emboîtement et la proportion des motifs à trois espèces).
Cependant, ces deux versions surestiment systématiquement le niveau trophique
maximal et le diamètre des réseaux, ce qui suggère que les variables utilisées
ne capturent pas adéquatement certains mécanismes importants sous-jacents aux
réseaux, notamment en ce qui a trait au transfert d'énergie au sein des systèmes
écologiques complexes. Malgré cela, les réseaux trophiques empiriques sont
proches de leur entropie maximale tel que prédit par mes modèles, ce qui suggère
que les variables utilisées parviennent tout de même à bien capturer la
complexité des réseaux trophiques, et par conséquent, une grande partie des
mécanismes écologiques sous-jacents aux réseaux. 

\subsection{Comment améliorer et étendre la théorie?} 

Différentes versions de la théorie de l'entropie maximale des réseaux trophiques
peuvent être développées selon les variables d'état choisies. La sélection des
variables d'état peut se faire sur la base des limites et erreurs de prédiction
des versions précédentes. Par exemple, puisque les versions élaborées dans
l'article~\ref{art:article2} n'ont pas réussi à prédire adéquatement le niveau
trophique maximal et le diamètre des réseaux, une version contraignant la
longueur des chaînes trophiques pourrait s'avérer plus précise. Une telle
version pourrait utiliser l'énergie métabolique totale de la communauté comme
variable d'état (imposant une limite au transfert d'énergie au sein des réseaux)
ou le niveau trophique moyen comme contrainte écologique. Bien que le
développement de telles versions améliorées de la théorie puisse permette
d'obtenir de meilleures prédictions de la structure des réseaux trophiques,
elles se cantonnent à une seule représentation des réseaux (p. ex. aux réseaux
d'interactions entre espèces), sans tenir compte des différents niveaux
d'organisation et de mesure des interactions.

Pour être plus complète, une théorie élargie de l'entropie maximale des réseaux
trophiques devrait pouvoir prédire simultanément différentes représentations des
réseaux trophiques. Cette théorie élargie serait l'extension logique de la
théorie de l'entropie maximale de l'écologie (METE, \cite{Harte2011Maximum}) aux
réseaux trophiques (d'où l'appellation «~Trophique-METE~»). La version ASNE de
METE utilise la superficie $A_0$, le nombre d'espèces $S_0$, le nombre total
d'individus $N_0$ et l'énergie métabolique totale $E_0$ (ou la biomasse) d'une
communauté comme variables d'état pour prédire plusieurs distributions d'intérêt
en écologie (\cite{Harte2008Maximum}, \cite{Harte2011Maximum},
\cite{Harte2014Maximum}). Trophique-METE, dans sa version élargie, pourrait
notamment ajouter le nombre total d'interactions entre espèces $L_0$ à ces
variables d'état pour former la version ASNEL de la théorie. En plus des
prédictions macroécologiques de METE (telles que la relation aire-espèce et les
distributions du nombre d'individus et du taux métabolique par espèce), ASNEL
permettrait de prédire la relation entre la structure des réseaux et leur
superficie, tout en faisant le pont entre les réseaux d'interactions entre
individus et entre espèces. Cette théorie élargie permettrait ainsi d'avoir une
vision plus harmonieuse et exhaustive des réseaux trophiques représentés à
différentes échelles spatiales et taxonomiques.

Les versions élargies de Trophique-METE peuvent être construites autour de
plusieurs autres variables d'état pour tenir compte des conditions locales
mentionnées dans l'article~\ref{art:article1}. Par exemple, le temps $t_0$ (qui
peut correspondre, entre autres, à la durée d'échantillonnage ou au temps écoulé
depuis le début de la succession primaire) peut être ajouté aux variables
d'état, ce qui permettrait de prédire la relation entre la structure des réseaux
et leur durée. Le temps pourrait également être intégré dans une version
dynamique de la théorie en laissant les valeurs des variables d'état fluctuer au
fil du temps, tel qu'effectué par \textcite{Harte2021Dynamete} dans leur version
dynamique de METE. Par ailleurs, le nombre de genres ou de familles au sein
d'une communauté peut également être ajouté aux variables d'état pour dériver
davantage de propriétés liées au niveau taxonomique, comme la relation
aire-genre ou aire-famille (\cite{Harte2014Maximum}). Dans le même ordre
d'idées, le nombre total d'interactions entre individus, genres ou familles peut
être intégré à la théorie pour prédire plus précisément la structure des réseaux
à différentes échelles taxonomiques. Le nombre total d'interactions entre
individus (c.-à-d. le nombre total d'événements de prédation survenus au cours
d'une période de temps donné) peut également être utilisé pour prédire la
structure des réseaux d'interactions quantitatives (p. ex. pour dériver la
distribution de fréquences d'interactions entre espèces). Le principal obstacle
à la construction d'une telle théorie élargie des réseaux trophiques est le
manque actuel de données empiriques nécessaires pour tester et valider
différentes versions de la théorie. Cependant, poursuivre activement le
développement de Trophique-METE nous permet de mieux identifier les données et
variables d'état à échantillonner. 

Quelle que soit la version développée, les prédictions de Trophique-METE
demeurent intrinsèquement probabilistes. Cependant, la matrice d'adjacence
prédite pour un réseau particulier, avec la méthode présentée dans
l'article~\ref{art:article2}, est unique et composée d'interactions binaires.
Cela est dû à l'approche heuristique employée, qui maximise l'entropie de la
distribution des valeurs singulières de la matrice d'adjacence, permettant ainsi
d'identifier celle ayant la plus grande complexité interne. Trophique-METE peut
également être développée pour inférer analytiquement des distributions
d'entropie maximale \textit{sur} les réseaux (c.-à-d. générant une distribution
de réseaux probabilistes tels que définis dans l'article~\ref{art:article1}).
Ces distributions peuvent être obtenues en utilisant des contraintes rigides (où
chaque réseau ayant une probabilité non nulle satisfait exactement les
contraintes) ou souples (où les réseaux prédits satisfont les contraintes en
moyenne). Mesurer la structure des réseaux à partir de ces distributions
d'entropie maximale nous permettrait de générer des prédictions probabilistes
pour l'ensemble des propriétés des réseaux. Le développement de Trophique-METE
autour des réseaux probabilistes constitue une avancée naturelle dans la
théorie, nous permettant d'aller au-delà des prédictions uniques des réseaux,
tout en offrant une alternative à l'estimation indépendante des probabilités
d'interactions entre espèces.  

%% Tester la théorie

\section{Validation de la théorie} 

En tant que théorie écologique en cours de développement, Trophique-METE doit
être testée à répétition et validée par des observations empiriques fiables.
Comme indiqué plus haut, ce processus itératif de validation permet de
développer des versions plus robustes et complètes de la théorie. Cette thèse
réalise la première évaluation de la théorie en la testant sur des données
empiriques recueillies à l'échelle globale, tant en milieu aquatique que
terrestre (Article~\ref{art:article2}). Lorsqu'elles surviennent, les erreurs de
prédiction peuvent provenir des variables d'état utilisées, qui peuvent être mal
mesurées, inadéquates ou insuffisantes. Cependant, plusieurs questions relatives
à l'évaluation de la théorie restent en suspens, notamment en ce qui a trait aux
particularités et à l'équilibre écologique des réseaux trophiques d'entropie
maximale (c.-à-d. des réseaux dont la complexité est bien prédite par la
théorie). En approfondissant notre compréhension des caractéristiques des
réseaux trophiques d'entropie maximale, nous pourrons mieux définir les champs
d'application et limites de la théorie.

\subsection{Quelles sont les particularités des réseaux trophiques d'entropie maximale?} 

Jusqu'à présent, Trophique-METE a été développée pour l'ensemble des réseaux
trophiques, sans tenir compte des caractéristiques propres aux différents types
d'écosystèmes. Autrement dit, les prédictions de la théorie sont identiques pour
tout écosystème disposant des mêmes valeurs de variables d'état, indépendamment
de leurs particularités. Toutefois, les écosystèmes pourraient être contraints
par différentes variables d'état, selon les mécanismes écologiques qui les
régissent. Par exemple, nous pourrions tester si la biomasse totale contraint
autant les réseaux trophiques aquatiques que terrestres. Les caractéristiques
des écosystèmes, telles que leur habitat (\cite{Shurin2005All}), leur niveau de
perturbation anthropique (\cite{Tylianakis2007Habitat}) et leur superficie
(\cite{Galiana2018Spatiala}), peuvent en effet impacter les processus
écologiques sous-jacents à leur structure émergente et, par conséquent, la
capacité des variables d'état utilisées à capturer adéquatement ces processus.
De même, adapter la théorie de l'entropie maximale des réseaux trophiques à
d'autres types de réseaux (comme aux réseaux d'interactions hôtes-parasites ou
plantes-pollinisateurs) impliquerait d'identifier les variables d'état qui les
déterminent spécifiquement, lesquelles pourraient différer de celles gouvernant
les réseaux trophiques. Les particularités des réseaux moins bien prédits par la
théorie définissent les limites de celle-ci et peuvent nous guider dans le
développement de différentes versions mieux adaptées à divers contextes
écologiques.

Au-delà des caractéristiques biologiques des écosystèmes, la représentation des
réseaux trophiques peut également impacter la capacité de la théorie à faire de
bonnes prédictions. Dans cette thèse, j'ai montré que le principe d'entropie
maximale prédit bien la structure des réseaux d'interactions binaires entre
espèces (Article~\ref{art:article2}). Le développement et la validation d'une
théorie élargie des réseaux trophiques aideront à déterminer si MaxEnt est plus
efficace pour prédire la structure des réseaux à un niveau taxonomique
(interactions entre individus, populations, espèces ou clades) et à une échelle
de mesure (interactions binaires ou quantitatives) particuliers, ou s'il prédit
convenablement l'ensemble de ces représentations. Étant donné que les échanges
de matière et d'énergie au sein des systèmes écologiques complexes s'effectuent
entre individus, je fais l'hypothèse que la théorie sera plus efficace pour
prédire la structure des réseaux trophiques lorsqu'ils reflètent au mieux ces
échanges, c.-à-d. lorsqu'ils représentent fidèlement la complexité écologique.
Tester la théorie sur plusieurs types de réseaux trophiques s'avère donc
essentiel pour bien comprendre ce qui détermine la structure de différentes
représentations des systèmes écologiques complexes. 

\subsection{Un réseau d'entropie maximale est-il à l'équilibre?} 

Le temps requis pour que les réseaux trophiques atteignent leur entropie
maximale, dans la limite des contraintes imposées par les variables d'état,
n'est pas bien compris. Nous pouvons supposer que les écosystèmes en transition,
qui n'ont pas eu assez de temps pour atteindre un état d'équilibre à la suite
d'une perturbation écologique, sont moins susceptibles d'être près de leur
entropie maximale. En effet, les prédictions de METE sont plus robustes pour les
communautés stables, c.-à-d. pour les communautés dont les variables d'état ne
fluctuent pas trop rapidement (\cite{Newman2020Disturbance},
\cite{Harte2021Dynamete}). Cela s'explique par le fait que les variables d'état
et les distributions prédites sont évaluées pour le même instant dans le temps.
Lorsque la valeur des variables d'état change trop rapidement, laissant un délai
insuffisant pour permettre aux distributions de s'ajuster, l'exactitude des
prédictions est compromise. Puisque Trophique-METE est aussi basée sur la valeur
instantanée des variables d'état, il convient de vérifier si la validité de ses
prédictions ne tient également que pour les écosystèmes dont les variables
d'état sont stables.

Les réseaux trophiques doivent également respecter le principe de conservation
de biomasse, selon lequel le bilan global entre les flux intrants et sortants de
matière est à l'équilibre. Ce principe s'applique autant à l'écosystème dans son
ensemble qu'à chaque espèce individuelle, puisque le taux moyen de production
d'énergie d'une espèce doit être équivalent à son taux moyen d'utilisation
(croissance, métabolisme et perte) et de consommation par les autres espèces
(\cite{Sterner2002Ecological}). La validation d'une version élargie de
Trophique-METE, qui utiliserait notamment l'énergie métabolique totale de la
communauté comme variable d'état, peut donc en partie être effectuée en
vérifiant le respect du principe de conservation de masse à l'échelle de
l'écosystème et de l'espèce. Pour ce faire, la théorie doit prédire les
distributions du taux d'utilisation d'énergie et du nombre d'individus par
espèce, ainsi que de la force d'interaction entre prédateurs et proies. Elle
peut toutefois s'appuyer sur des relations allométriques existantes, comme la
relation entre l'énergie métabolique et la biomasse fournie par la théorie
métabolique de l'écologie (\cite{Brown2004Metabolic}, \cite{West1997General}),
pour estimer les paramètres manquants, le cas échéant (tel qu'effectué par
\cite{Harte2022Equation} dans leurs efforts d'unification de METE avec la
théorie métabolique de l'énergie). Cette méthode de validation permettrait à la
fois de vérifier la cohérence interne et le réalisme écologique de
Trophique-METE, tout en permettant d'évaluer sa compatibilité avec les relations
allométriques largement acceptées en écologie.


%% Appliquer la théorie

\section{Applications potentielles de la théorie} 

La théorie de l'entropie maximale des réseaux trophiques a pour but d'unifier
nos connaissances sur les réseaux trophiques en offrant une compréhension
globale et cohérente de leur structure émergente et de leur incertitude. Elle y
parvient en faisant des prédictions probabilistes des propriétés émergentes des
réseaux trophiques dans le temps et l'espace. Ces prédictions sont facilitées
par la nature parcimonieuse de la théorie, qui ne repose que sur un ensemble
limité d'informations écologiques. Le rôle de Trophique-METE est donc double~:
elle permet de mieux comprendre les déterminants des réseaux trophiques et de
générer des prédictions vérifiables sur des réseaux qui n'ont pas encore été
observés.

\subsection{Compréhension des mécanismes sous-jacents aux réseaux trophiques} 

Les erreurs de prédiction peuvent orienter
la recherche sur les systèmes écologiques complexes en mettant en lumière les
mécanismes sous-jacents aux réseaux trophiques et en guidant le développement de
versions plus robustes et précises de la théorie.

When it comes to the environmental impacts of climate change and
habitat loss, seldom the potential changes in food-web topology are considered.
However, such changes in network structure could drastically alter how entire
ecosystems function and react to external perturbations. Simulation studies
could help us predict how ecological networks could look like in decades from
now. These are not perfect, as they rely on too few data, on too many biological
assumptions, and on other ecological models. Quantifying the uncertainty of
these simulations and their robustness to some assumptions, like the choice of
spatial scale, can make their predictions a lot more realistic and
contextualized. This thesis could potentially help identify regions where food
webs are more vulnerable to climate change and habitat loss in order to better
protect their integrity and the ecosystem services they provide. This thesis
also seeks to provide a methodological breakthrough in the macroecology of food
webs.

\subsection{Prévision de la structure des réseaux trophiques} 

Trophique-METE permet de faire des prédictions sur des événements qui n'ont pas
encore été observés.

When it comes to the environmental impacts of climate change and
habitat loss, seldom the potential changes in food-web topology are considered.
However, such changes in network structure could drastically alter how entire
ecosystems function and react to external perturbations. Simulation studies
could help us predict how ecological networks could look like in decades from
now. These are not perfect, as they rely on too few data, on too many biological
assumptions, and on other ecological models. Quantifying the uncertainty of
these simulations and their robustness to some assumptions, like the choice of
spatial scale, can make their predictions a lot more realistic and
contextualized. This thesis could potentially help identify regions where food
webs are more vulnerable to climate change and habitat loss in order to better
protect their integrity and the ecosystem services they provide. This thesis
also seeks to provide a methodological breakthrough in the macroecology of food
webs.

A second primary objective of my thesis is to develop first-order forecasts of
food-web structure. These forecasts will be based on different scenarios of
biodiversity and habitat loss, and will be computed using the predictive MaxEnt
models designed in my thesis. My forecasts should under no circumstances be used
in any environmental decision-making process, but should instead be seen as a
first attempt to study expected changes in food-web structure at large spatial
and temporal scales.

In Chapter 3, I will forecast food-web structure using the models of the two
previous chapters. More specifically, I will use forecasting models of species
richness at the global scale as inputs to the model developed in Chapter 1 to
forecast food-web structure worldwide according to different scenarios of
biodiversity loss. In contrast, Trophic-METE (Chapter 2) will be used to
forecast the structure of given food webs, by varying the values of their state
variables again according to different environmental scenarios.

These models could be used as a first assessment of the potential impact of
climate change, biodiversity loss, and habitat loss on food-web structure. I
will forecast network structure after a given event (e.g. an extreme climate
event) or in future years (e.g. in 2050), based on forecasts of the state
variables. Forecasting network structure continuously would require, to my
understanding, that I develop a dynamic version of my models, which will not be
realized as part of my thesis. I reiterate that my models should not be used in
environmental decision-making, but should rather be seen as a first step toward
the development of more realistic forecasting models of network structure.

A main application of my model is the prediction of the structure of ecological
networks globally. In this chapter, I will illustrate this application by
mapping terrestrial food webs worldwide. To do so, I will need to query data on
species distributions in order to estimate local species richness. Because my
objective will be to *illustrate* this application, I will not conduct species
distribution modelling (SDM) myself. Instead, I will query polygon or raster
data on species distributions from online databases, such as the [IUCN Red List
of Threatened
Species](https://www.iucnredlist.org/resources/spatial-data-download) or
[BiodiversityMapping](https://biodiversitymapping.org/). I will choose taxonomic
groups based on available data, but I will make sure to include terrestrial
mammals, birds, amphibians, and plants. Extinct and vagrant species will not be
included.

For each taxa, local species richness will be estimated on equal area grids.
Because my predictions might be highly sensible to the spatial scale of
analysis, species richness will be approximated at different spatial scales
(e.g. 10 km$^2$, 100 km$^2$ or 1,000 km$^2$). In each grid, I will sum the
number of species of each selected taxonomic group to estimate local total
species richness.

Every grid will represent a different biological community. I will consider that
the number of nodes in food webs are equivalent to the total number of species
present in each grid. I will thus make the assumption that every species has at
least one trophic interaction with another member of its community. Mean and
variance of different food-web measures will then be estimated for each food web
(i.e. in each grid) using the model developed in this chapter. Maps of different
spatial scales will be compared, and the spatial variation of network structure
will be briefly examined and discussed.

Because my model will be very parsimonious, predictions of network structure
could be conducted at a very large spatial scale, but perhaps with poorer
precision. This is in contrast with the model of the next chapter, which
considers more state variables, i.e. more biological information, in the
derivation of maximum entropy distributions and of food-web structure.

In the previous two chapters, I will have predicted food-web structure from a
given set of state variables. The objective of Chapter 3 will be to develop a
forecasting model of food-web structure using my predictive models, along with
forecasts of these state variables. More specifically, I will forecast food-web
structure at the global scale using the model developed in Chapter 1, and
forecast the structure of more well-defined ecological networks using the model
of Chapter 2. Note that the goal will not be to produce actionable forecasts.
Instead, my forecasts will be of first-orders, and should be seen as a first
attempt to conduct such analysis efficiently. Moreover, the reliability of my
forecasts of food-web structure will be conditional on the validity of my
predictive models and on the reliability of the forecasts of the state
variables. A more comprehensive forecasting framework, that integrates diverse
predictive and forecasting models (potentially including mine) and different
types of data (e.g. traits, phylogeny), should be established and rigorously
tested in order to use them in any environmental decision-making processes.

Because my predictive models of the two previous chapters will be snapshots of
biological communities, my forecasts will not be continuous. I will rather
forecast food-web structure at discrete points in time. Indeed, my predictive
models are meant to generate snapshots of *stable* biological communities. I
might assume, perhaps too liberally, that my forecasted food webs were
transiting from one state (present) to another (future). Analyzing and comparing
the structure of each state would allow me to assess the potential impacts of
given environmental scenarios on food-web topology. Generating continuous
forecasts of food-web structure would require a more dynamical version of my
MaxEnt predictive models.

he models of the two previous chapters predicted a probability distribution of
the form:

$$P(x|\textbf{Y}),$${eq:predictions}

where $x$ is the value of a given food-web measure, and $\textbf{Y}$ is the
vector of values of the state variables.

In Chapter 3, I will forecast food-web structure at a given timepoint $t$:

$$P(x(t)|\textbf{Y}(t)).$${eq:predictions}

My proposed methodology is outlined in fig:conceptual3. Food-web structure will
be forecasted at the global scale using worldwide forecasts of local species
richness as inputs to the maximum-entropy predictive model of Chapter 1. I will
also forecast the structure of sampled biological communities using forecasts of
the ANSEL state variables as inputs to Trophic-METE. Note that, in the latter
case, forecasts of $S_0$ for a given geolocated food web could be estimated from
worldwide forecasts of local species richness. The biggest challenge of this
chapter will thus be to derive and use realistic forecasts of the state
variables, under different environmental scenarios.

Food-web structure will be forecasted using forecasts of the
state variables $A_1$ (forecasted habitat area), $N_1$ (forecasted
number of individuals), $S_1$ (forecasted species richness), $E_1$
(forecasted total energetic requirement), and $L_1$ (forecasted number
of links). As part of my research, $A_1$ will be chosen arbitrarily
based on different scenarios of habitat loss for illustrative purposes.
The other state variables $N_1$, $S_1$, $E_1$, and $L_1$ will be
forecasted using $A_1$, the initial values of the state variables
($A_0$, $N_0$, $S_0$, $E_0$, $L_0$), as well as the metrics of maximum
entropy derived by METE or Trophic-METE. Next, forecasted values of the
state variables will be used as inputs to the model of Chapter 2 to
generate a probability distribution of feasible weighted food webs, and
diverse measures of these webs will be computed. The other goal of
Chapter 3 will be to used to model developed in Chapter 1, along with
forecasts of species richness based on different environmental
scenarios, to forecast food-web structure at the global scale. Note that
worldwide forecasts of species richness could also be used to estimate
$S_1$ for a given geolocalized food web.

In the first instance, I will forecast food-web structure at the global scale
using the model developed in Chapter 1. More specifically, I will forecast
worldwide local species richness in 2020, 2050, and 2080 (those dates are
however arbitrary and subject to change), under the four Representative
Concentration Pathways (RCP 2.6, RCP 4.5, RCP 6.0, and RCP 8.5) of the
Intergovernmental Panel on Climate Change [IPCC; PachMaye15]. To do so, I might
conduct species distribution modelling (SDMs) to predict species richness from
given [bioclimatic variables](https://www.worldclim.org/data/bioclim.html) using
the data on local species richness of Chapter 1. Forecasting of worldwide local
species richness could then be conducted using scenario-specific forecasts of
those bioclimatic variables. Next, I will use my forecasts of local species richness as inputs to the
maximum-entropy model developed in Chapter 1. More specifically, I will predict
the number of links from the forecasted numbers of species using the flexible
links model, derive the (joint) degree distributions of maximum entropy, predict
a distribution of adjacency matrices using simulated annealing and/or
topological models, and compute various measures of food-web structure from
those adjacency matrices. I could then map food-web structure at the global
scale. I will generate a total of 12 maps for each analyzed measure, i.e. at
three timepoints and under the four environmental scenarios. These measures will
essentially be the same as those analyzed in Chapter 1, including connectance,
nestedness, and the maximum trophic level. 
In the second instance, I will forecast weighted food webs using the derived
metrics of Trophic-METE. Following Hart11, I will vary the values of the ANSEL
state variables based on hypothetical scenarios. More specifically, I will
explore the potential impacts on food-web structure of a 10%, 25%, and 50%
reduction of suitable habitat area. This habitat loss could be the result of
land use change (e.g. deforestation) or of climate change (i.e. a shrunk or
displacement of suitable habitat due to global warming). More generally, I will
forecast weighted food webs and their structure given an area loss of
$A_{loss}$.

To do so, I could use the endemics-area relationship (EAR) predicted by METE
[Hart11], i.e. the expected number of species $E$ unique to a cell of area $A$
within $A_0$:

$$E(A) = S_0 \sum_{n_0=1}^{N_0}P(n_0)\Phi(n_0),$${eq:endemics}

where $\Phi(n_0)$ is the probability that a species has an abundance of $n_0$ in
$A_0$ (evaluated using eq:SAD), and $P(n_0)$ is the probability that all of
those $n_0$ individuals will be in the same cell of area $A$ (evaluated using
eq:ISSAD). Evaluating $E(A_{loss})$ gives me the expected number of species
unique to the cell of area $A_{loss}$; this is the expected number of species
lost if an area of $A_{loss}$ becomes unsuitable. The expected number of
remaining species $S_1$ could thus be estimated as follows:

$$S_1 = S_0 - E(A_{loss}).$${eq:remain}

Similarly, I could use the species-area relationship $S(A)$, evaluated at
$A_0-A_{loss}$, to estimate the number of remaining species. Note that,
according to Hart11, using the SAR would be more appropriate if a patch of area
$A_1 = A_0-A_{loss}$ remains, whereas we should typically use the EAR if a patch
of area $A_{loss}$ is lost.

The number of remaining individuals $N_1$ could be estimated as follows:

$$N_1 = N_0 -
S_0\sum_{n_0=1}^{N_0}\sum_{n=0}^{n_0}nP(n)\Phi(n_0)$$

where $P(n)$ is the probability that $n$ individuals of a species of $n_0$
individuals are found in the lost patch. Finally, the total energy requirement
of the community $E_1$ after the perturbation could be estimated as follows:

$$E_1 = E_0 -
(N_0-N_1)\sum_{n_0=1}^{N_0}\int_{\epsilon=1}^{E_0}\epsilon\Theta(\epsilon)\Phi(n_0)d\epsilon,$${eq:indremain}

where $\Theta(\epsilon)$ is the probability density function that an individual
of a species with $n_0$ individuals has a metabolic energy between $\epsilon$
and $d\epsilon$ (eq:ISED).

To summarize this forecasting model, I would start with a community of known
$A_0$, $N_0$, $S_0$, and $E_0$. I would predict $L_0$ from $S_0$ if it has not
been measured empirically, and predict food webs using the method presented in
Chapter 2. I would then investigate how food-web structure would be impacted if
an area $A_1$ remains after an habitat of area $A_{loss}$ is no longer suitable
for the community. I would predict $N_1$, $S_1$, $E_1$, and $L_1$ using the
metrics derived in Chapter 2 and by HartZill08. I could then predict food-web
structure using this second set of state variables as inputs to Trophic-METE. To
this date, I ignore on which sampled communities I will apply my method, and how
I could assess the robustness of my forecasts without any historical data.

The forecasts generated in this chapter will not be actionable. First, the
habitat loss scenarios will be completely arbitrary, and in that regard my
results will be of illustrative purposes only. However, Trophic-METE could be
used to generate *first-order* forecasting of food webs with *legitimate*
habitat loss scenarios. However, my model should be validated and rigorously
tested in order to be confident in its forecasts, and I am doubtful that
sufficient data is available to conduct such analysis. I might have to use a set
of proxies for historical data, but to this date it is still unclear to me what
would those proxies be. Second, the robustness of my worldwide forecasts of
local food-web structure will be conditional on the robustness of the model of
Chapter 1. However, on the same basis as the argument presented in the first
chapter, I expect that my forecasts will have a low reliability, although they
might have a high validity for more stable communities. Again, I am not sure how
I will assess the reliability and validity of those worldwide forecasts
empirically.

A main limitation of my models is that they will be snapshots of biological
communities and of their food webs, and as such I do not expect that they will
make adequate predictions for communities in transition. Because communities
undergoing rapid changes (e.g. deforestation, rapid rise in temperature) might
be in transition for a very long period of time, my model might not be used in
any environmental decision-making processes. In that respect, a more dynamical
version of Trophic-METE should be built. Better, my forecasting models could be
integrated within a framework of predictive and forecasting models. They could,
for instance, be used to constrain the space of feasible food webs that would be
fed to other models. This methodological framework could be very promising to
forecast pairwise species interactions and the structure of their ecological
networks more accurately and realistically.

Potential limitations of my models include its potential lack of accuracy.
Indeed, the predicted distribution of the number of links, obtained from the
number of species, is quite spread out [MacDBanv20b], and as a consequence I
believe that the confidence intervals around my predictions might be very large
as well. However, my predictions might be more precise for networks with
empirically-measured number of links. I could also use the *maximum a
posteriori* estimates of the number of links instead of the whole posterior
distribution, but in that case the outcomes of my models will be interpreted
differently.

Finally, my models will not predict actual networks, but will provide
first-order predictions of their structure and adjacency matrices. In this
regard, they could best be used with other predictive methods to obtain more
realistic predictions of ecological networks. For example, my predictions of
network structure could be used to constrain the space of feasible interactions
between given pairs of species. They could thus complement existing methods of
inference of pairwise interactions based on various ecological information such
as species traits, phylogenies, and expert knowledge. Similarly, my forecasts of
network structure might be improved by integrating my models into a more
comprehensive methodological framework for the prediction of ecological networks
in space.

Chapter 2 will have two main steps. First, I will estimate local
species richness worldwide under different scenarios of climate change and
habitat loss. Next, I will simulate food-web structure worldwide according to my
predictions in species richness.

I will use the Representative
Concentration Pathways (RCPs), adopted by the Intergovernmental Panel on Climate
Change (IPCC) in its [Fifth Assessment Report (2014)][IPCC], for the potential
trajectories of worldwide temperature and precipitation. Specifically, I will
use the IPCC climate models under the RCP 2.6, RCP 4.5, RCP 6.0, and RCP 8.5
scenarios. Regarding habitat loss, I will use as a baseline the mean rate, over a 30-year
period, of natural habitat loss for every country where such data is available.
Four scenarios will be considered: (1) a constant rate over time, (2) a 0.1
decrease by decades, (3) a 0.1 increase by decades, and (4) a 0.2 increase by
decades. My model will not account for other causes of biodiversity loss, such as habitat
fragmentation, species migration capabilities and occurrence of potential
predators or preys in the new range. Local rates of change in species richness will be
estimated at the global scale using machine learning techniques and species
distribution models (SDMs). According to the performance of the machine learning models, I might also use
species distribution models (SDMs). SDMs are used to predict species
distributions  from occurrence and environmental data [GuisTing13]. The temporal
change of these variables, under the 16 scenarios, could be used to forecast
species distribution and thus species richness through time. I might also use
joint species distribution models (JSDMs), which, unlike conventional SDMs, take
into account species co-occurrence in such modeling [PollTing14].

I will use the model presented and validated in
the first chapter to forecast food-web structure at the global scale.
Specifically, I will use the generated distributions, obtained from Chapter 1,
of the food-web measures for each level of species richness. I should then be
able to map the medians of these distributions at the global scale according to
my different predictions of worldwide local species richness. I will compute and map the predicted temporal variations of these measures, and
identify the hotspots where these changes might be the most extreme. I will also
explore what makes a food web robust to species richness variation, and identify
environmental thresholds beyond which significant changes in food-web structure
would likely occur. I will finally discuss the potential impacts of these changes on the functioning
of biological communities and on the ecosystem services they provide.

Because food-web structure
is so closely related to species richness, a change in local species richness
could have hurtful consequences on the structure of ecological networks, and
thus on their functioning. This would raise awareness on ecosystem fragility and help
identify hotspots of highly vulnerable food webs.


\endinput
%%
%% End of file `04_conclusion.tex'.
