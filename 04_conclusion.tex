%%
%% This is file `04_conclusion.tex',
%% generated with the docstrip utility.
%%

% Utilisez la macro de langue appropriée.
\francais   
%ou
%%\anglais
\chapter*{Conclusions générales}

%% Développer la théorie

Cette thèse pose les fondements de la théorie de l'entropie maximale des réseaux
trophiques (Trophique-METE). Basée sur le principe d'entropie maximale, cette
théorie permet d'inférer des distributions de probabilité caractérisant la
structure émergente des réseaux trophiques. Les distributions prédites sont
celles qui représentent le mieux les informations fournies par les contraintes
écologiques, sans faire aucune supposition additionnelle sur la forme de la
distribution au-delà des contraintes choisies. Plusieurs versions de la théorie
peuvent être développées selon les contraintes utilisées et les distributions
prédites. 

Le Chapitre 1 explique pourquoi les interactions entre espèces, telles que les
interactions prédateurs-proies et plantes-herbivores, sont intrinsèquement
probabilistes. Notre manque de connaissances sur les interactions locales et
régionales, attribuable aux limites d'observation des interactions entre espèces
(\cite{Jordano2016Sampling}), ainsi que la variabilité spatiale et temporelle
des interactions locales (\cite{Poisot2015Species}), font en sorte que
l'incertitude des interactions entre espèces ne puisse pas être complètement
éliminée indépendamment de nos efforts d'échantillonnage. 


Puisque l'incertitude
des interactions entre espèces est irréductible, il est d'autant plus important 
de quantifier cette incertitude et d

Ce chapitre contribue 
à 

La propagation de
l'incertitude des interactions entre espèces à la structure émergente des
réseaux locaux rend à son tour cette dernière incertaine. 


Les interactions entre espèces sont le résultat de nombreux
mécanismes écologiques agissant à différentes échelles spatiales et temporelles.

Ce chapitre 

souligne donc l'importance d'une théorie de l'entropie maximale des
réseaux trophiques permettant de quantifier de manière cohérente l'incertitude
au sein des réseaux d'interactions entre espèces à partir d'un nombre limité
d'informations écologiques.





Mieux
comprendre la source de cette incertitude nous permet de mieux interpréter les
prédictions probabilistes de Trophique-METE (p. ex. la distribution jointe de
degrés) 


Puisque l'incertitude
des interactions entre espèces se propage à la structure des réseaux, mieux
comprendre la source de cette incertitude nous permet de mieux interpréter les
prédictions probabilistes de Trophique-METE (p. ex. la distribution jointe de
degrés) 

 


et
que la structure émergente des réseaux d’interactions est écologiquement et statistiquement
contrainte

\section{Développement de la théorie de l'entropie maximale des réseaux trophiques} 

\subsection{Où en sommes-nous?} 

\subsection{Comment améliorer et étendre la théorie?} 



%% Tester la théorie

\section{Validation de la théorie} 

\subsection{Quels réseaux sont d'entropie maximale?} 

\subsection{Un réseau d'entropie maximale est-il à l'équilibre?} 


%% Appliquer la théorie

\section{Applications potentielles de la théorie} 

\subsection{Compréhension des mécanismes sous-jacents aux réseaux d'interactions} 

\subsection{Prévision de la structure des réseaux trophiques} 

\endinput
%%
%% End of file `04_conclusion.tex'.
