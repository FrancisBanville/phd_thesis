%%
%% This is file `04_conclusion.tex',
%% generated with the docstrip utility.
%%

% Utilisez la macro de langue appropriée.
\francais   
%ou
%%\anglais
\chapter*{Conclusions générales}

\section{Ce que l'on a appris}

\subsection{Sur les sources d'incertitude des interactions entre espèces} 

\begin{itemize}
    \item On connait mieux les différentes sources d'incertitude des interactions probabilistes 
    \item On connait mieux les propriétés des interactions probabilistes 
    \item On a de meilleures lignes directrices sur l'utilisation des interactions probabilistes
\end{itemize}

\subsection{Sur les mécanismes écologiques sous-jacents aux réseaux trophiques} 

\begin{itemize}
    \item La structure des réseaux est déterminé par un ensemble restreint de variables écologiques
    \item Mesurer la différence entre un réseau empirique et les prédictions de MaxEnt nous permet de mieux comprendre les mécanismes écologiques (modèles nuls)
\end{itemize}


\subsection{Sur notre capacité à prédire la structure des réseaux trophiques} 

\begin{itemize}
    \item Même si on n'a pas beaucoup de données, on peut prédire la stucture des réseaux avec MaxEnt
    \item Par contre, ce n'est pas parfait. Il manque encore des contraintes écologiques (comme le niveau trophique maximum)
\end{itemize}



\section{Perspectives futures}

\subsection{Développement de la théorie de l'entropie maximale des réseaux trophiques} 

\begin{itemize}
    \item ANSEL 
    \item Prédire les réseaux quantitatifs
    \item Dériver la distribution probabiliste MaxEnt des réseaux
    \item MaxEnt pour prédire la structure des réseaux dans l'espace
    \item MaxEnt pour prévoire (forecast) la structure des réseaux dans le temps
    \item MaxEnt pour générer des priors informatifs 
    \item Pour mieux prédire les interactions entre paires d'espèces
\end{itemize}


\subsection{Vérification du réalisme écologique des réseaux prédits} 

\begin{itemize}
    \item Est-ce que nos prédictions respectent le principe de mass balance ?
\end{itemize}


\subsection{Identification des sites optimaux d'échantillonnage} 

\begin{itemize}
    \item Nous pouvons minimiser nos incertitudes, mais pas la variabilité des interactions
    \item On doit identifier des endroits à échantillonner pour réduire au maximum notre incertitude des interactions
\end{itemize}


\subsection{Quelques questions en suspens} 

\begin{itemize}
    \item Combien de temps ça prend avant qu'un réseau soit proche de son entropie maximale ?
    \item Est-ce que c'est vrai que les réseaux régionaux ne sont pas d'entropie maximale ?
    \item Est-ce que différents types de réseaux sont régis par des contraintes différentes ?
\end{itemize}


\endinput
%%
%% End of file `04_conclusion.tex'.
