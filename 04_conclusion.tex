%%
%% This is file `04_conclusion.tex',
%% generated with the docstrip utility.
%%

% Utilisez la macro de langue appropriée.
\francais   
%ou
%%\anglais

\chapter{Conclusions générales}

%% Développer la théorie

Cette thèse pose les fondements de la théorie de l'entropie maximale des réseaux
trophiques (Trophique-METE). Basée sur le principe d'entropie maximale, cette
théorie permet d'inférer des distributions de probabilité caractérisant la
structure émergente des réseaux trophiques. Les distributions prédites sont
celles qui représentent le mieux les informations fournies par les contraintes
écologiques, sans faire de supposition additionnelle sur la forme de la
distribution au-delà des contraintes choisies. Plusieurs versions de la théorie
peuvent être développées selon les contraintes utilisées et les distributions
prédites. Cette thèse développe et teste deux versions restreintes de la théorie
(Article~\ref{art:article2}). 

L'article~\ref{art:article1} constitue le cadre conceptuel de la théorie. Il
explique pourquoi les interactions entre espèces, telles que les interactions
prédateurs-proies et plantes-herbivores, sont intrinsèquement probabilistes.
Notre manque de connaissances sur les interactions locales et régionales, en
partie attribuable aux limites d'observation des interactions entre espèces
(\cite{Jordano2016Sampling}), ainsi que la variabilité spatiale et temporelle
des interactions locales (\cite{Poisot2015Species}), introduisent une
incertitude dans notre mesure des interactions. Cette incertitude est
irréductible à l'échelle locale, c.-à-d. qu'elle persiste même avec l'ajout de
nouvelles données empiriques. Cet article souligne l'importance d'identifier, de
quantifier et de communiquer cette incertitude inhérente aux interactions entre
espèces. 

Trophique-METE fournit un cadre d'analyse permettant de quantifier plus
facilement et avec cohérence l'incertitude au sein des réseaux trophiques à
partir d'un nombre limité d'informations écologiques. Elle s'inscrit donc dans
la vision probabiliste des réseaux élaborée dans l'article~\ref{art:article1} en
générant des prédictions probabilistes de différentes mesures de la structure
des réseaux trophiques. En effet, puisque l'incertitude des interactions entre
espèces se propage à la structure émergente des réseaux, cette dernière est
également probabiliste. En approfondissant notre compréhension des interactions
probabilistes, nous pouvons mieux interpréter les prédictions de la théorie
(comme la distribution jointe de degrés, qui décrit la probabilité qu'une espèce
ait un nombre donné de proies et de prédateurs). Les sources d'incertitude et
mécanismes écologiques sous-jacents aux interactions entre espèces, décrits dans
l'article~\ref{art:article1}, peuvent inspirer le développement de différentes
versions de la théorie, p. ex. en prenant comme variables d'état celles qui
conditionnent les probabilités d'interactions entre espèces. Puisque les
prédictions de Trophique-METE ne sont basées que sur ces variables d'état, cette
théorie infère la structure des réseaux sans supposer (à tort) que les
interactions entre espèces sont indépendantes les unes des autres, ce qui
contribue à la robustesse de la théorie.

En reconnaissant que la structure émergente des réseaux trophiques est
écologiquement et statistiquement contrainte, l'article~\ref{art:article2} jette
les bases de la théorie de l'entropie maximale des réseaux trophiques. Il montre
deux approches (analytique et heuristique) pour prédire la structure des réseaux
trophiques à l'aide du principe d'entropie maximale. L'approche analytique
permet d'inférer directement une distribution de probabilité à l'aide de la
méthode des multiplicateurs de Lagrange, alors que l'approche heuristique permet
d'identifier le réseau d'entropie (ou de complexité) maximale en permutant
aléatoirement les interactions entre espèces tout en respectant les contraintes
imposées par les variables d'état. Les deux versions de la théorie développées
dans cet article, qui diffèrent selon les variables d'état utilisées, mettent en
œuvre ces deux approches. Cet article fournit également les outils nécessaires
pour développer d'autres versions de la théorie reposant sur d'autres variables
d'état. Dans les sous-sections suivantes, je discute des développements actuels
et futurs, des tests à venir et des applications potentielles de la théorie. 

\section{Développement de la théorie de l'entropie maximale des réseaux trophiques} 

\subsection{Où en sommes-nous?} 

L'article~\ref{art:article2} propose deux premières versions de la théorie,
employant les approches analytiques et heuristiques basées sur le principe
d'entropie maximale. La première version de trophique-METE prédit la structure
des réseaux trophiques à partir du nombre d'espèces et du nombre d'interactions.
Cette version a permis de prédire la distribution jointe de degrés (approche
analytique) et la matrice d'adjacence utilisée pour calculer différentes mesures
de la structure des réseaux (approche heuristique). La deuxième version, quant à
elle, prédit la matrice d'adjacence à partir du nombre de proies et de
prédateurs pour chaque espèce dans le réseau (approche heuristique). La première
version prédit bien la distribution jointe de degrés lorsque la connectance du
réseau est élevée, alors que la seconde version prédit mieux les autres mesures
testées (comme l'emboîtement). Cependant, ces deux versions surestiment
systématiquement le niveau trophique maximal, ce qui suggère que les variables
utilisées ne capturent pas adéquatement certains mécanismes importants
sous-jacents aux réseaux, notamment en ce qui a trait au transfert d'énergie au
sein des systèmes écologiques complexes. Malgré cela, les réseaux trophiques
empiriques sont proches de leur entropie maximale tel que prédit par mes
modèles, ce qui suggère que les variables utilisées parviennent tout de même à
bien capturer la complexité des réseaux trophiques, et par conséquent, une
grande partie des mécanismes écologiques sous-jacents aux réseaux. 

\subsection{Comment améliorer et étendre la théorie?} 

Différentes versions de la théorie de l'entropie maximale des réseaux trophiques
peuvent être développées selon les variables d'état choisies. La sélection des
variables d'état peut se faire sur la base des limites et erreurs de prédiction
des versions précédentes. Par exemple, puisque les versions élaborées dans
l'article~\ref{art:article2} n'ont pas réussi à prédire adéquatement le niveau
trophique maximal et le diamètre des réseaux, une version contraignant la
longueur des chaînes trophiques pourrait s'avérer plus précise. Une telle
version pourrait utiliser l'énergie métabolique totale de la communauté comme
variable d'état (imposant une limite au transfert d'énergie au sein des réseaux)
ou le niveau trophique moyen comme contrainte écologique. Bien que le
développement de telles versions améliorées de la théorie puisse permette
d'obtenir de meilleures prédictions de la structure des réseaux trophiques,
elles se cantonnent à une seule représentation des réseaux (p. ex. aux réseaux
d'interactions binaires entre espèces), sans tenir compte des différents niveaux
d'organisation et de mesure des interactions.

Pour être plus complète, une théorie élargie de l'entropie maximale des réseaux
trophiques devrait pouvoir prédire simultanément différentes représentations des
réseaux trophiques. Cette théorie élargie serait l'extension logique de la
théorie de l'entropie maximale de l'écologie (METE, \cite{Harte2011Maximum}) aux
réseaux trophiques (d'où l'appellation «~Trophique-METE~»). La version ANSE de
METE utilise la superficie $A_0$, le nombre total d'individus $N_0$, le nombre
d'espèces $S_0$ et l'énergie métabolique totale $E_0$ (ou la biomasse) d'une
communauté comme variables d'état pour prédire plusieurs distributions d'intérêt
en écologie (\cite{Harte2008Maximum}, \cite{Harte2011Maximum},
\cite{Harte2014Maximum}). Trophique-METE, dans sa version élargie, pourrait
notamment ajouter le nombre total d'interactions entre espèces $L_0$ à ces
variables d'état pour former la version ANSEL de la théorie. En plus des
prédictions macroécologiques de METE (telles que la relation aire-espèce et les
distributions du nombre d'individus et du taux métabolique par espèce), ANSEL
pourrait prédire la relation entre la structure des réseaux et leur superficie,
tout en faisant le pont entre les réseaux d'interactions entre individus et
entre espèces. Cette théorie élargie permettrait ainsi d'avoir une vision plus
harmonieuse des réseaux trophiques représentés à différentes échelles spatiales
et taxonomiques.

Les versions élargies de Trophique-METE peuvent être construites autour de
plusieurs autres variables d'état pour tenir compte des conditions locales
mentionnées dans l'article~\ref{art:article1}. Par exemple, le temps $t_0$ (qui
peut correspondre, entre autres, à la durée d'échantillonnage ou au temps écoulé
depuis le début de la succession primaire) peut être ajouté aux variables
d'état, ce qui permettrait de prédire la relation entre la structure des réseaux
et leur durée. Le temps pourrait également être intégré dans une version
dynamique de la théorie en laissant les valeurs des variables d'état fluctuer au
fil du temps, tel qu'effectué par \textcite{Harte2021Dynamete}. Par ailleurs, le
nombre de genres ou de familles au sein d'une communauté peut également être
ajouté aux variables d'état pour dériver davantage de mesures liées au niveau
taxonomique, comme la relation aire-genre ou aire-famille
(\cite{Harte2014Maximum}). Dans le même ordre d'idées, le nombre total
d'interactions entre individus, genres ou familles peut être intégré à la
théorie pour prédire plus précisément la structure des réseaux à différentes
échelles taxonomiques. Le nombre total d'interactions entre individus (c.-à-d.
le nombre total d'événements de prédation survenus au cours d'une période de
temps donné) peut également être utilisé pour prédire la structure des réseaux
d'interactions quantitatives (p. ex. pour dériver la distribution de fréquences
d'interactions par espèce). Le principal obstacle à la construction d'une telle
théorie élargie est le manque actuel de données empiriques nécessaires pour
tester et valider différentes versions de la théorie, bien que cela ne devrait
pas nous empêcher de poursuivre activement son développement.

%% En construction %%
En plus de prédire des distributions de probabilité au sein des réseaux,
Trophique-METE peut également être développé pour inférer des distributions
d'entropie maximale \textit{sur} les réseaux (c.-à-d. donnant la probabilité
qu'une configuration de réseau soit réalisée). Ces distributions peuvent être 
dérivées à partir de contraintes rigides (c.-à-d. ). Contraindre l'espace 
des réseaux possibles. La structure des réseaux peut être calculée à partir de 
cette distribution, pour avoir la probabilité des mesures. 

%% Tester la théorie

\section{Validation de la théorie} 

\subsection{Quels réseaux sont d'entropie maximale?} 

\subsection{Un réseau d'entropie maximale est-il à l'équilibre?} 


%% Appliquer la théorie

\section{Applications potentielles de la théorie} 

\subsection{Compréhension des mécanismes sous-jacents aux réseaux d'interactions} 

\subsection{Prévision de la structure des réseaux trophiques} 


\endinput
%%
%% End of file `04_conclusion.tex'.
