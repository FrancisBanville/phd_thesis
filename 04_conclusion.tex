%%
%% This is file `04_conclusion.tex',
%% generated with the docstrip utility.
%%

% Utilisez la macro de langue appropriée.
\francais   
%ou
%%\anglais
\chapter*{Conclusions générales}

\section{Ce que l'on a appris (3 pages)}

\subsection{Sur les sources d'incertitude des interactions entre espèces} 

\begin{itemize}
    \item On connait mieux les différentes sources d'incertitude des interactions probabilistes 
    \item On connait mieux les propriétés des interactions probabilistes 
    \item On a de meilleures lignes directrices sur l'utilisation des interactions probabilistes
\end{itemize}

\subsection{Sur les mécanismes écologiques sous-jacents aux réseaux trophiques} 

\begin{itemize}
    \item La structure des réseaux est déterminé par un ensemble restreint de variables écologiques
    \item Mesurer la différence entre un réseau empirique et les prédictions de MaxEnt nous permet de mieux comprendre les mécanismes écologiques (modèles nuls)
\end{itemize}


\subsection{Sur notre capacité à prédire la structure des réseaux trophiques} 

\begin{itemize}
    \item Même si on n'a pas beaucoup de données, on peut prédire la stucture des réseaux avec MaxEnt
    \item Par contre, ce n'est pas parfait. Il manque encore des contraintes écologiques (comme le niveau trophique maximum)
\end{itemize}

The maximum entropy (joint) degree distributions derived in this chapter are the
least-biased distributions that respect a set of constraints, here a normalizing
and averaging constraints. These distributions might not be the best fit to
empirical degree distributions, especially if ecological mechanisms not encoded
in these constraints are at play. In this regard, my predictions could be used
as first-order approximations, as prior distributions, or as null models.

My method has several limitations. One of these is that the assumption of
independence between in and out-degrees might be frequently violated in
empirical networks. An alternative method for deriving the joint degree
distribution of maximum entropy, which does not rely on this assumption, might
be needed. 

According to the current
method, species have a non-zero probability of having $k_{in} = k_{out} = 0$.
This should be forbidden by my model, because each species in a food web needs
to have at least one trophic interaction. The expression for the joint degree
distribution of maximum entropy should therefore be corrected accordingly.

\section{Perspectives futures (6 pages)}

\subsection{Développement de la théorie de l'entropie maximale des réseaux trophiques} 

\begin{itemize}
    \item ANSEL 
    \item Champs d'application 
    \item Prédire les réseaux quantitatifs
    \item Dériver la distribution probabiliste MaxEnt des réseaux
    \item MaxEnt pour prédire la structure des réseaux dans l'espace
    \item MaxEnt pour prévoire (forecast) la structure des réseaux dans le temps
    \item MaxEnt pour générer des priors informatifs 
    \item Pour mieux prédire les interactions entre paires d'espèces
\end{itemize}

In Chapter 2, in contrast with Chapter 1, I will derive food-web structure using
the principle of maximum entropy and 5 state variables. Those state variables
are the area $A$, the total number of individuals $N$, the number of species
$S$, the total energetic requirement of the community $E$, and the number of
interactions $L$. Using $N$ and $E$ will allow me to predict many properties of
weighted food webs. The main objective of this chapter will be to develop a
novel version (ANSEL) of the maximum entropy theory of ecology (METE) applied to
food webs. This version will be called *Trophic-METE*.

Because Trophic-METE will necessitate data often difficult to sample or
estimate, it will not be used to map food-web structure at a large spatial scale
as in Chapter 1. Rather, I will model specific and well-defined communities for
which those variables have been measured or estimated. When not measured
empirically, the number of links will again be estimated using species richness.
My model will be used to predict food-web structure with, as I suspect, more
accuracy than the model of the first chapter, as long as communities are not in
a transition state and that the state variables have been measured or estimated
accurately. As in Chapter 1, the maximum entropy distributions could be used as
informative prior distributions in Bayesian inference, and deviations from these
predictions could be used to identify ecological mechanisms leading to observed
network structures.

Chapter 2 aims at bringing together predictive models of food-web structure and
the derivations of METE to derive a distribution of feasible and plausible
networks, from which measures could be computed. In that respect, the main
objective of this chapter is to derive and validate another version of METE
(namely Trophic-METE) applied to the spatial prediction of food webs. I believe
that the SAD, the SAR, the ISSAD, the ISED, and the joint degree distribution of
maximum entropy will provide sufficient information to predict accurately the
probability of the interaction strength between two species $i$ and $j$, i.e.
$P(w_{ij}|A, n_i, n_j, S_0, e_i, e_j, k_i, k_j)$, or, more generally, the
probability of a weighted network $W$ given the values of the state variables,
i.e. $P(W|A, N_0, S_0, E_0, L_0)$. Again, these predictions should be more
precise for more stable communities, in contrast to communities undergoing rapid
changes.  

These predictions could be used in a similar way as the ones of the first
chapter. Notably, my model could be used to predict the food web of given
communities. However, in contrast to Chapter 1, the model of Chapter 2 will not
be built to generate first-order approximations of food-web structure at a large
spatial scale, i.e. for communities in each grid of a map, unless $N_0$ and
$E_0$ are estimated rigorously at the given spatial resolution. Rather, I expect
that my model could be used to predict the network of more well-defined
community. For example, it could be used to predict the food web of a given
forest patch if values of $A_0$, $S_0$, $N_0$, and $E_0$ are well sampled. If my
model generates accurate predictions of food webs, it could greatly facilitate
the conduction of *preliminary* studies on the functioning and stability of
ecological communities, without directly relying on data on species
interactions. In a related way, it could be used to generate informative prior
distributions of food-web structure that could be updated with empirical data.
Finally, my model could be used as a null model to investigate the role of
ecological mechanisms in the shaping of food-web topology.

My model could *a priori* be applied on any taxonomic groups or at any level of
organization, as long as state variables and derived metrics are defined
coherently. I however hypothesize that my model will be more robust when used
with a set of taxons that encompasses all trophic levels (i.e. comprising
autotrophs, herbivores, and carnivores), so that the modeled community is well
represented. On another note, I believe that my model will be suitable to any
spatial scales. However, it is not clear to me how it will perform with very
small or very large communities.

In this chapter, I will derive a distribution of feasible weighted networks
using the area $A_0$, the number of species $S_0$, the total number of
individuals $N_0$, the total energetic requirement of the community $E_0$, and
the (predicted) number of links $L_0$. 

For communities where $L_0$ has not been measured or estimated empirically, I
will estimate $L_0$ from $S_0$ using the flexible links model of MacDBanv20b.
Similarly, for communities where $E_0$ has not been estimated empirically, but
for which the total biomass $M_0$ has been estimated, I will estimate $E_0$ from
$M_0$ using the metabolic theory of ecology [MTE; Klei32; BrowGill04; and
Hart11]:



$$E_o = cM_o^{3/4}$$

where $c$ is a positive constant whose value can be estimated empirically.

These predictions (eq:SAD to eq:ISED) could be used directly to predict
weighted food webs, along with the joint degree distribution of maximum entropy
derived in Chapter 1 (eq:maxentjoint). This would constitute the ANSEL version
of the maximum entropy theory of ecology (Trophic-METE). However, I will attempt
to derive other metrics directly. One of these metrics could be the interaction
strength distribution $P(w_{ij}|A, n_i, n_j, S_o, e_i, e_j, k_i, k_j)$, where
$P(w_{ij})$ is the probability density that if two species are taken at random
from the species pool, that they will interact and that this interaction will
have a strength of $w_{ij}$. Note that this probability density function might
depend only on a subset of the ANSEL variables. Another approach could be to
attempt to predict emerging network properties, such as the distribution of
trophic levels in a food web, from the five state variables.

These predictions will be compared to observed data using ranking plots. To do
so, I might have to estimate a few state variables in existing datasets in order
to conduct this validation process.

I will use the distributions derived from Trophic-METE to constrain the space of
feasible networks. More specifically, I will predict the probability
distribution $P(W|A, N_0, S_0, E_0, L_0)$ given the values of the state
variables. To do so, I might use simulating annealing algorithms to filter out
potential adjacency matrices that do not respect the derived distributions. Note
that $P(W)$ will take into account the uncertainty around $L_0$ for networks
whose number of links will be predicted from species richness. I believe,
however, that my version of METE will be more appealing if I can predict $P(W)$
analytically from these distributions, probably using some approximations. This
might however prove very challenging, and perhaps unfeasible as part of my
thesis.

Emerging food-web properties, such as nestedness and modularity, could be
computed for all potential food webs in the distribution of feasible networks.
The resulting distribution of food-web properties will take into account the
likelihood of every potential network. These predictions will then be compared
with empirical data.

The probability distribution of weighted food webs derived in this chapter will
not be the result of any mechanistic or ecological processes. It will instead
characterize the networks we expect to see, by chance, given values of the state
variables ($A_0$, $N_0$, $S_0$, $E_0$, and $L_0$). This network-centered version
of the maximum entropy theory of ecology will make testable predictions that
will be compared with empirical data. Deviations from the maximum entropy degree
distributions will be ecologically informative. However, if I find systematic
deviations from my predictions, I would need to revisit my model, either by
choosing a different set of state variables or by choosing different constraints
on the distributions. On another note, in contrast to the model of Chapter 1, I
believe that adapting Trophic-METE to other types of ecological networks would
require important changes in the choices of state variables and of constraints.

Overall, Trophic-METE could be very useful to derive least-biased distributions
of food-web structure and of weighted interactions, given prior knowledge on a
biological community. These least-biased distributions could be used as prior
distributions, or could be used directly to estimate various emerging food-web
properties. Predictions will be snapshots of an ecosystem, and knowing the value
of the state variables at a given time point would allow me to predict its food
web at that time point. In the next and final chapter, I will take advantage of
this feature and use Trophic-METE, along with the model of Chapter 1, to
forecast food webs over time.

\subsection{Prévision de la structure des réseaux trophiques} 

When it comes to the environmental impacts of climate change and
habitat loss, seldom the potential changes in food-web topology are considered.
However, such changes in network structure could drastically alter how entire
ecosystems function and react to external perturbations. Simulation studies
could help us predict how ecological networks could look like in decades from
now. These are not perfect, as they rely on too few data, on too many biological
assumptions, and on other ecological models. Quantifying the uncertainty of
these simulations and their robustness to some assumptions, like the choice of
spatial scale, can make their predictions a lot more realistic and
contextualized. This thesis could potentially help identify regions where food
webs are more vulnerable to climate change and habitat loss in order to better
protect their integrity and the ecosystem services they provide. This thesis
also seeks to provide a methodological breakthrough in the macroecology of food
webs.

A second primary objective of my thesis is to develop first-order forecasts of
food-web structure. These forecasts will be based on different scenarios of
biodiversity and habitat loss, and will be computed using the predictive MaxEnt
models designed in my thesis. My forecasts should under no circumstances be used
in any environmental decision-making process, but should instead be seen as a
first attempt to study expected changes in food-web structure at large spatial
and temporal scales.

In Chapter 3, I will forecast food-web structure using the models of the two
previous chapters. More specifically, I will use forecasting models of species
richness at the global scale as inputs to the model developed in Chapter 1 to
forecast food-web structure worldwide according to different scenarios of
biodiversity loss. In contrast, Trophic-METE (Chapter 2) will be used to
forecast the structure of given food webs, by varying the values of their state
variables again according to different environmental scenarios.

These models could be used as a first assessment of the potential impact of
climate change, biodiversity loss, and habitat loss on food-web structure. I
will forecast network structure after a given event (e.g. an extreme climate
event) or in future years (e.g. in 2050), based on forecasts of the state
variables. Forecasting network structure continuously would require, to my
understanding, that I develop a dynamic version of my models, which will not be
realized as part of my thesis. I reiterate that my models should not be used in
environmental decision-making, but should rather be seen as a first step toward
the development of more realistic forecasting models of network structure.

A main application of my model is the prediction of the structure of ecological
networks globally. In this chapter, I will illustrate this application by
mapping terrestrial food webs worldwide. To do so, I will need to query data on
species distributions in order to estimate local species richness. Because my
objective will be to *illustrate* this application, I will not conduct species
distribution modelling (SDM) myself. Instead, I will query polygon or raster
data on species distributions from online databases, such as the [IUCN Red List
of Threatened
Species](https://www.iucnredlist.org/resources/spatial-data-download) or
[BiodiversityMapping](https://biodiversitymapping.org/). I will choose taxonomic
groups based on available data, but I will make sure to include terrestrial
mammals, birds, amphibians, and plants. Extinct and vagrant species will not be
included.

For each taxa, local species richness will be estimated on equal area grids.
Because my predictions might be highly sensible to the spatial scale of
analysis, species richness will be approximated at different spatial scales
(e.g. 10 km$^2$, 100 km$^2$ or 1,000 km$^2$). In each grid, I will sum the
number of species of each selected taxonomic group to estimate local total
species richness.

Every grid will represent a different biological community. I will consider that
the number of nodes in food webs are equivalent to the total number of species
present in each grid. I will thus make the assumption that every species has at
least one trophic interaction with another member of its community. Mean and
variance of different food-web measures will then be estimated for each food web
(i.e. in each grid) using the model developed in this chapter. Maps of different
spatial scales will be compared, and the spatial variation of network structure
will be briefly examined and discussed.

Because my model will be very parsimonious, predictions of network structure
could be conducted at a very large spatial scale, but perhaps with poorer
precision. This is in contrast with the model of the next chapter, which
considers more state variables, i.e. more biological information, in the
derivation of maximum entropy distributions and of food-web structure.

In the previous two chapters, I will have predicted food-web structure from a
given set of state variables. The objective of Chapter 3 will be to develop a
forecasting model of food-web structure using my predictive models, along with
forecasts of these state variables. More specifically, I will forecast food-web
structure at the global scale using the model developed in Chapter 1, and
forecast the structure of more well-defined ecological networks using the model
of Chapter 2. Note that the goal will not be to produce actionable forecasts.
Instead, my forecasts will be of first-orders, and should be seen as a first
attempt to conduct such analysis efficiently. Moreover, the reliability of my
forecasts of food-web structure will be conditional on the validity of my
predictive models and on the reliability of the forecasts of the state
variables. A more comprehensive forecasting framework, that integrates diverse
predictive and forecasting models (potentially including mine) and different
types of data (e.g. traits, phylogeny), should be established and rigorously
tested in order to use them in any environmental decision-making processes.

Because my predictive models of the two previous chapters will be snapshots of
biological communities, my forecasts will not be continuous. I will rather
forecast food-web structure at discrete points in time. Indeed, my predictive
models are meant to generate snapshots of *stable* biological communities. I
might assume, perhaps too liberally, that my forecasted food webs were
transiting from one state (present) to another (future). Analyzing and comparing
the structure of each state would allow me to assess the potential impacts of
given environmental scenarios on food-web topology. Generating continuous
forecasts of food-web structure would require a more dynamical version of my
MaxEnt predictive models.

he models of the two previous chapters predicted a probability distribution of
the form:

$$P(x|\textbf{Y}),$${eq:predictions}

where $x$ is the value of a given food-web measure, and $\textbf{Y}$ is the
vector of values of the state variables.

In Chapter 3, I will forecast food-web structure at a given timepoint $t$:

$$P(x(t)|\textbf{Y}(t)).$${eq:predictions}

My proposed methodology is outlined in fig:conceptual3. Food-web structure will
be forecasted at the global scale using worldwide forecasts of local species
richness as inputs to the maximum-entropy predictive model of Chapter 1. I will
also forecast the structure of sampled biological communities using forecasts of
the ANSEL state variables as inputs to Trophic-METE. Note that, in the latter
case, forecasts of $S_0$ for a given geolocated food web could be estimated from
worldwide forecasts of local species richness. The biggest challenge of this
chapter will thus be to derive and use realistic forecasts of the state
variables, under different environmental scenarios.

Food-web structure will be forecasted using forecasts of the
state variables $A_1$ (forecasted habitat area), $N_1$ (forecasted
number of individuals), $S_1$ (forecasted species richness), $E_1$
(forecasted total energetic requirement), and $L_1$ (forecasted number
of links). As part of my research, $A_1$ will be chosen arbitrarily
based on different scenarios of habitat loss for illustrative purposes.
The other state variables $N_1$, $S_1$, $E_1$, and $L_1$ will be
forecasted using $A_1$, the initial values of the state variables
($A_0$, $N_0$, $S_0$, $E_0$, $L_0$), as well as the metrics of maximum
entropy derived by METE or Trophic-METE. Next, forecasted values of the
state variables will be used as inputs to the model of Chapter 2 to
generate a probability distribution of feasible weighted food webs, and
diverse measures of these webs will be computed. The other goal of
Chapter 3 will be to used to model developed in Chapter 1, along with
forecasts of species richness based on different environmental
scenarios, to forecast food-web structure at the global scale. Note that
worldwide forecasts of species richness could also be used to estimate
$S_1$ for a given geolocalized food web.

In the first instance, I will forecast food-web structure at the global scale
using the model developed in Chapter 1. More specifically, I will forecast
worldwide local species richness in 2020, 2050, and 2080 (those dates are
however arbitrary and subject to change), under the four Representative
Concentration Pathways (RCP 2.6, RCP 4.5, RCP 6.0, and RCP 8.5) of the
Intergovernmental Panel on Climate Change [IPCC; PachMaye15]. To do so, I might
conduct species distribution modelling (SDMs) to predict species richness from
given [bioclimatic variables](https://www.worldclim.org/data/bioclim.html) using
the data on local species richness of Chapter 1. Forecasting of worldwide local
species richness could then be conducted using scenario-specific forecasts of
those bioclimatic variables. Next, I will use my forecasts of local species richness as inputs to the
maximum-entropy model developed in Chapter 1. More specifically, I will predict
the number of links from the forecasted numbers of species using the flexible
links model, derive the (joint) degree distributions of maximum entropy, predict
a distribution of adjacency matrices using simulated annealing and/or
topological models, and compute various measures of food-web structure from
those adjacency matrices. I could then map food-web structure at the global
scale. I will generate a total of 12 maps for each analyzed measure, i.e. at
three timepoints and under the four environmental scenarios. These measures will
essentially be the same as those analyzed in Chapter 1, including connectance,
nestedness, and the maximum trophic level. 
In the second instance, I will forecast weighted food webs using the derived
metrics of Trophic-METE. Following Hart11, I will vary the values of the ANSEL
state variables based on hypothetical scenarios. More specifically, I will
explore the potential impacts on food-web structure of a 10%, 25%, and 50%
reduction of suitable habitat area. This habitat loss could be the result of
land use change (e.g. deforestation) or of climate change (i.e. a shrunk or
displacement of suitable habitat due to global warming). More generally, I will
forecast weighted food webs and their structure given an area loss of
$A_{loss}$.

To do so, I could use the endemics-area relationship (EAR) predicted by METE
[Hart11], i.e. the expected number of species $E$ unique to a cell of area $A$
within $A_0$:

$$E(A) = S_0 \sum_{n_0=1}^{N_0}P(n_0)\Phi(n_0),$${eq:endemics}

where $\Phi(n_0)$ is the probability that a species has an abundance of $n_0$ in
$A_0$ (evaluated using eq:SAD), and $P(n_0)$ is the probability that all of
those $n_0$ individuals will be in the same cell of area $A$ (evaluated using
eq:ISSAD). Evaluating $E(A_{loss})$ gives me the expected number of species
unique to the cell of area $A_{loss}$; this is the expected number of species
lost if an area of $A_{loss}$ becomes unsuitable. The expected number of
remaining species $S_1$ could thus be estimated as follows:

$$S_1 = S_0 - E(A_{loss}).$${eq:remain}

Similarly, I could use the species-area relationship $S(A)$, evaluated at
$A_0-A_{loss}$, to estimate the number of remaining species. Note that,
according to Hart11, using the SAR would be more appropriate if a patch of area
$A_1 = A_0-A_{loss}$ remains, whereas we should typically use the EAR if a patch
of area $A_{loss}$ is lost.

The number of remaining individuals $N_1$ could be estimated as follows:

$$N_1 = N_0 -
S_0\sum_{n_0=1}^{N_0}\sum_{n=0}^{n_0}nP(n)\Phi(n_0)$$

where $P(n)$ is the probability that $n$ individuals of a species of $n_0$
individuals are found in the lost patch. Finally, the total energy requirement
of the community $E_1$ after the perturbation could be estimated as follows:

$$E_1 = E_0 -
(N_0-N_1)\sum_{n_0=1}^{N_0}\int_{\epsilon=1}^{E_0}\epsilon\Theta(\epsilon)\Phi(n_0)d\epsilon,$${eq:indremain}

where $\Theta(\epsilon)$ is the probability density function that an individual
of a species with $n_0$ individuals has a metabolic energy between $\epsilon$
and $d\epsilon$ (eq:ISED).

To summarize this forecasting model, I would start with a community of known
$A_0$, $N_0$, $S_0$, and $E_0$. I would predict $L_0$ from $S_0$ if it has not
been measured empirically, and predict food webs using the method presented in
Chapter 2. I would then investigate how food-web structure would be impacted if
an area $A_1$ remains after an habitat of area $A_{loss}$ is no longer suitable
for the community. I would predict $N_1$, $S_1$, $E_1$, and $L_1$ using the
metrics derived in Chapter 2 and by HartZill08. I could then predict food-web
structure using this second set of state variables as inputs to Trophic-METE. To
this date, I ignore on which sampled communities I will apply my method, and how
I could assess the robustness of my forecasts without any historical data.

The forecasts generated in this chapter will not be actionable. First, the
habitat loss scenarios will be completely arbitrary, and in that regard my
results will be of illustrative purposes only. However, Trophic-METE could be
used to generate *first-order* forecasting of food webs with *legitimate*
habitat loss scenarios. However, my model should be validated and rigorously
tested in order to be confident in its forecasts, and I am doubtful that
sufficient data is available to conduct such analysis. I might have to use a set
of proxies for historical data, but to this date it is still unclear to me what
would those proxies be. Second, the robustness of my worldwide forecasts of
local food-web structure will be conditional on the robustness of the model of
Chapter 1. However, on the same basis as the argument presented in the first
chapter, I expect that my forecasts will have a low reliability, although they
might have a high validity for more stable communities. Again, I am not sure how
I will assess the reliability and validity of those worldwide forecasts
empirically.

A main limitation of my models is that they will be snapshots of biological
communities and of their food webs, and as such I do not expect that they will
make adequate predictions for communities in transition. Because communities
undergoing rapid changes (e.g. deforestation, rapid rise in temperature) might
be in transition for a very long period of time, my model might not be used in
any environmental decision-making processes. In that respect, a more dynamical
version of Trophic-METE should be built. Better, my forecasting models could be
integrated within a framework of predictive and forecasting models. They could,
for instance, be used to constrain the space of feasible food webs that would be
fed to other models. This methodological framework could be very promising to
forecast pairwise species interactions and the structure of their ecological
networks more accurately and realistically.

Potential limitations of my models include its potential lack of accuracy.
Indeed, the predicted distribution of the number of links, obtained from the
number of species, is quite spread out [MacDBanv20b], and as a consequence I
believe that the confidence intervals around my predictions might be very large
as well. However, my predictions might be more precise for networks with
empirically-measured number of links. I could also use the *maximum a
posteriori* estimates of the number of links instead of the whole posterior
distribution, but in that case the outcomes of my models will be interpreted
differently.

Finally, my models will not predict actual networks, but will provide
first-order predictions of their structure and adjacency matrices. In this
regard, they could best be used with other predictive methods to obtain more
realistic predictions of ecological networks. For example, my predictions of
network structure could be used to constrain the space of feasible interactions
between given pairs of species. They could thus complement existing methods of
inference of pairwise interactions based on various ecological information such
as species traits, phylogenies, and expert knowledge. Similarly, my forecasts of
network structure might be improved by integrating my models into a more
comprehensive methodological framework for the prediction of ecological networks
in space.

Chapter 2 will have two main steps. First, I will estimate local
species richness worldwide under different scenarios of climate change and
habitat loss. Next, I will simulate food-web structure worldwide according to my
predictions in species richness.

I will use the Representative
Concentration Pathways (RCPs), adopted by the Intergovernmental Panel on Climate
Change (IPCC) in its [Fifth Assessment Report (2014)][IPCC], for the potential
trajectories of worldwide temperature and precipitation. Specifically, I will
use the IPCC climate models under the RCP 2.6, RCP 4.5, RCP 6.0, and RCP 8.5
scenarios. Regarding habitat loss, I will use as a baseline the mean rate, over a 30-year
period, of natural habitat loss for every country where such data is available.
Four scenarios will be considered: (1) a constant rate over time, (2) a 0.1
decrease by decades, (3) a 0.1 increase by decades, and (4) a 0.2 increase by
decades. My model will not account for other causes of biodiversity loss, such as habitat
fragmentation, species migration capabilities and occurrence of potential
predators or preys in the new range. Local rates of change in species richness will be
estimated at the global scale using machine learning techniques and species
distribution models (SDMs). According to the performance of the machine learning models, I might also use
species distribution models (SDMs). SDMs are used to predict species
distributions  from occurrence and environmental data [GuisTing13]. The temporal
change of these variables, under the 16 scenarios, could be used to forecast
species distribution and thus species richness through time. I might also use
joint species distribution models (JSDMs), which, unlike conventional SDMs, take
into account species co-occurrence in such modeling [PollTing14].

I will use the model presented and validated in
the first chapter to forecast food-web structure at the global scale.
Specifically, I will use the generated distributions, obtained from Chapter 1,
of the food-web measures for each level of species richness. I should then be
able to map the medians of these distributions at the global scale according to
my different predictions of worldwide local species richness. I will compute and map the predicted temporal variations of these measures, and
identify the hotspots where these changes might be the most extreme. I will also
explore what makes a food web robust to species richness variation, and identify
environmental thresholds beyond which significant changes in food-web structure
would likely occur. I will finally discuss the potential impacts of these changes on the functioning
of biological communities and on the ecosystem services they provide.

Because food-web structure
is so closely related to species richness, a change in local species richness
could have hurtful consequences on the structure of ecological networks, and
thus on their functioning. This would raise awareness on ecosystem fragility and help
identify hotspots of highly vulnerable food webs.

\subsection{Vérification du réalisme écologique des réseaux prédits} 

\begin{itemize}
    \item Est-ce que nos prédictions respectent le principe de mass balance ?
\end{itemize}

Harte (METE) dérive la probabilité jointe $R(n, e)$, où $e$ est le taux
d'utilisation d'énergie d'un individu hétérotrophe (taux métabolique ou perte)
et $n$ le nombre d'individus d'une espèce. $R(n, e)$ est basée sur les variables
$N_0$, $S_0$ et $E_0$. 

On peut utiliser une relation allométrique pour trouver la masse à partir de
l'énergie. 

$e_i = c \times m_i^b = m_i^{3/4}$ si on pose $c = 1$. On a donc une
distribution de probabilité $\phi (n, m)$.

Dans ma recherche, j'ai dérivé $P(k_{in}, k_{out}$ et une matrice binaire $A$ à
partir de $L_0$ et $S_0$. 

On sait également qu'il existe une relation allométrique entre la masse de la
proie et la masse du prédateur (équation à trouver) et entre le coefficient de
prédation $\alpha_j$ et la masse du prédateur (travail de Ben). 

Finalement, selon le principe de mass balance, la consommation moins les perte
moins la prédation devrait être égale à 0. 

$N_i (\sum \alpha_{ij} N_j - e_j - \sum \alpha_{ij} N_j) = 0$ (équation à
vérifier). 

Modèle de Ben pour $\alpha_{ij}$:

$F_{i,j} = \alpha_{i,j} \times B_i \times N_j$ (relation fonctionnelle type 1)
ou $F_{i,j} = \frac{\alpha_{i,j} \times B \times N}{1 + h_j \times \alpha_{i,j}
\times B_i \times N_j}$. 

$F$ en tonne / km2 year $\alpha$ en km2 / ind $B$ en tonne / km2 $N$ en ind /
km2

Théorie métabolique: Libre de toute écologie des communautés. Relation linéaire négative entre le
log(N) et le log(m). Par contre, pas à Bylot, où la relation est non linéaire
avec un minimum (moins de régulation des meso). Relation entre productivité
primaire ou secondaire et la température. Également relation entre S et la
température. 

Random:

$\alpha_j \sum_{i = 1}^{S} \int R(n, e) \times A(e) \times n de$ : comsommation 

A : interaction binaire ici (lien ou non, 0 ou 1)

Références:

Voir Harte et al. (2022):https://www.nature.com/articles/s42003-022-03817-8.

Objectif 1: Est-ce que les équations MaxEnt dérivées et les relations allométriques
respectent le mass balance ? 

Pour chaque espèce, on peut calculer $n_i$ et $e_i$ à partir de MaxEnt et
calculer $m_i$ et $\alpha_i$ avec les relations allométriques. On peut calculer
la consommation et la prédation à partir de ces paramètres. 

Vérifier si la mass balance (bilan) est respectée pour chaque espèce ou
globalement. Qu'est-ce qui manque pour qu'elle soit respectée ? 


Objectif 2: 

Plusieurs avenues possibles. 

1) Trouver la matrice de poids qui respecte la masse balance. 

Utiliser le principe de masse balance comme contraintes et résoudre MaxEnt
heuristiquement (par permutations ou numériquement). 

$N = A^{-1} e$ ou e est la somme et intégrale des e de Hart (pour chaque
individus d'une espèce). 

Trouver la probabilité jointe $P(N_i, E_i, A_{i.}$, où $N_i$ et $E_i$ sont des
scalaires et $A_{i.}$ un vecteur. Résoudre N, A et e simultanément. 

2) On peut également dériver numériquement $N$ et $e$ (avec MaxEnt et $A$ comme
contraintes (données empiriques). 

Comment l'équation change la relation entre $N$ et $m$ trouvée par Hart (ajout
d'une contrainte basée sur les interactions). 

3) Isoler A et comparer avec matrice binaire MaxEnt ?


\subsection{Identification des sites optimaux d'échantillonnage} 

\begin{itemize}
    \item Nous pouvons minimiser nos incertitudes, mais pas la variabilité des interactions
    \item On doit identifier des endroits à échantillonner pour réduire au maximum notre incertitude des interactions
\end{itemize}


\subsection{Quelques questions en suspens} 

\begin{itemize}
    \item Combien de temps ça prend avant qu'un réseau soit proche de son entropie maximale ?
    \item Est-ce que c'est vrai que les réseaux régionaux ne sont pas d'entropie maximale ?
    \item Est-ce que différents types de réseaux sont régis par des contraintes différentes ?
\end{itemize}


\endinput
%%
%% End of file `04_conclusion.tex'.
