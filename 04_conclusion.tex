%%
%% This is file `04_conclusion.tex',
%% generated with the docstrip utility.
%%

% Utilisez la macro de langue appropriée.
\francais   
%ou
%%\anglais

\chapter{Conclusions générales}

%% Développer la théorie

Cette thèse pose les fondements de la théorie de l'entropie maximale des réseaux
trophiques (Trophique-METE). Basée sur le principe d'entropie maximale, cette
théorie permet d'inférer des distributions de probabilité caractérisant la
structure émergente des réseaux trophiques. Les distributions prédites sont
celles qui représentent le mieux les informations fournies par les contraintes
écologiques, sans faire de supposition additionnelle sur la forme de la
distribution au-delà des contraintes choisies. Plusieurs versions de la théorie
peuvent être développées selon les contraintes utilisées et les distributions
prédites. Cette thèse développe et teste deux versions restreintes de la théorie
(Article~\ref{art:article2}). 

L'article~\ref{art:article1} explique pourquoi les interactions entre espèces,
telles que les interactions prédateurs-proies et plantes-herbivores, sont
intrinsèquement probabilistes. Notre manque de connaissances sur les
interactions locales et régionales, en partie attribuable aux limites
d'observation des interactions entre espèces (\cite{Jordano2016Sampling}), ainsi
que la variabilité spatiale et temporelle des interactions locales
(\cite{Poisot2015Species}), introduisent une incertitude dans notre mesure des
interactions entre espèces. Cette incertitude est irréductible à l'échelle
locale, c.-à-d. qu'elle persiste même avec l'ajout de nouvelles données
empiriques. Cet article souligne l'importance d'identifier, de quantifier et de
communiquer cette incertitude inhérente aux interactions entre espèces. 

Trophique-METE définit un cadre d'analyse permettant de quantifier plus
facilement et avec cohérence l'incertitude au sein des réseaux trophiques à
partir d'un nombre limité d'informations écologiques. Elle s'inscrit donc dans
la vision probabiliste des réseaux mise de l'avant dans
l'article~\ref{art:article1} en générant des prédictions probabilistes de
différentes mesures de la structure des réseaux trophiques. En effet, puisque
l'incertitude des interactions entre espèces se propage à la structure émergente
des réseaux, cette dernière est également probabiliste. En approfondissant notre
compréhension des interactions probabilistes, nous pouvons mieux interpréter les
prédictions de la théorie (comme la distribution jointe de degrés, qui décrit la
probabilité qu'une espèce ait un nombre donné de proies et de prédateurs). Les
sources d'incertitude et mécanismes écologiques sous-jacents aux interactions
entre espèces, décrits dans l'article~\ref{art:article1}, peuvent inspirer le
développement de différentes versions de la théorie, p. ex. en prenant comme
variables d'état celles qui conditionnent les probabilités d'interactions entre
espèces. Puisque les prédictions de la théorie ne sont basées que sur ces
variables d'état, la structure probabiliste des réseaux inférée par
Trophique-METE ne repose pas sur la supposition souvent erronée d'indépendance
entre les interactions, ce qui peut contribuer à la robustesse de la théorie.

En reconnaissant que la structure émergente des réseaux trophiques est
écologiquement et statistiquement contrainte, l'article~\ref{art:article2} jette
les bases de la théorie de l'entropie maximale des réseaux trophiques. Il montre
deux approches (analytique et heuristique) pour prédire la structure des réseaux
trophiques à l'aide du principe d'entropie maximale. L'approche analytique
permet d'inférer directement une distribution de probabilité à l'aide de la
méthode des multiplicateurs de Lagrange, alors que l'approche heuristique permet
d'identifier le réseau d'entropie (ou de complexité) maximale en permutant
aléatoirement les interactions entre espèces tout en respectant les contraintes
imposées par les variables d'état. Les deux versions de la théorie développées
dans cet article, qui diffèrent selon les variables d'état utilisées, mettent en
œuvre ces deux approches. Cet article fournit également les outils nécessaires
pour développer d'autres versions de la théorie reposant sur d'autres variables
d'état. Dans les sous-sections suivantes, je discute des développements actuels
et futurs, des tests à venir et des applications potentielles de la théorie. 

\section{Développement de la théorie de l'entropie maximale des réseaux trophiques} 

\subsection{Où en sommes-nous?} 

L'article~\ref{art:article2} propose deux premières versions de la théorie,
employant les approches analytiques et heuristiques basées sur le principe
d'entropie maximale. La première version de trophique-METE prédit la structure
des réseaux trophiques à partir du nombre d'espèces et du nombre d'interactions.
Cette version a permis de prédire la distribution jointe de degrés (approche
analytique) et la matrice d'adjacence utilisée pour calculer différentes mesures
de la structure des réseaux (approche heuristique). La deuxième version, quant à
elle, prédit la matrice d'adjacence à partir du nombre de proies et de
prédateurs pour chaque espèce dans le réseau (approche heuristique). La première
version prédit bien la distribution jointe de degrés lorsque la connectance du
réseau est élevée, alors que la seconde version prédit mieux les autres mesures
testées (comme l'emboîtement). Cependant, ces deux versions surestiment
systématiquement le niveau trophique maximal, ce qui suggère que les variables
utilisées ne capturent pas adéquatement certains mécanismes importants
sous-jacents aux réseaux, notamment en ce qui a trait au transfert d'énergie au
sein des systèmes écologiques complexes. Malgré cela, les réseaux prédits par la
deuxième version de la théorie sont proches de leur entropie maximale, ce qui
suggère que les variables utilisées parviennent tout de même à bien capturer la
complexité des réseaux trophiques, et par conséquent, une grande partie des
mécanismes écologiques sous-jacents.

\subsection{Comment améliorer et étendre la théorie?} 



%% Tester la théorie

\section{Validation de la théorie} 

\subsection{Quels réseaux sont d'entropie maximale?} 

\subsection{Un réseau d'entropie maximale est-il à l'équilibre?} 


%% Appliquer la théorie

\section{Applications potentielles de la théorie} 

\subsection{Compréhension des mécanismes sous-jacents aux réseaux d'interactions} 

\subsection{Prévision de la structure des réseaux trophiques} 

\endinput
%%
%% End of file `04_conclusion.tex'.
