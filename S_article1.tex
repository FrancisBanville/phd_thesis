%%
%% This is file `S_article1.tex',
%% generated with the docstrip utility.

%% Utilisez la macro de langue appropriée.
%\francais   
%ou
\anglais

\chapter{Supplementary material of Article~\ref{art:article1}}\label{supp:A}


\begin{refsection}

\addtocontents{toc}{\protect\setcounter{tocdepth}{0}}

\section{Host-parasite network data}

\subsection{Data description}

We use the collection of tripartite host-parasite networks sampled across Europe
of \textcite{Kopelke2017Foodweb}. This dataset contains well-resolved binary local
interactions between willows ($52$ species), willow-galling sawflies ($96$
species), and their parasitoids ($126$ species). Out of a total of $374$ local
networks, we retained those containing at least $5$ species, resulting in a set
of $233$ georeferenced local networks (networks sampled within areas of $0.1$ to
$0.3$ km² during June and/or July spanning $29$ years). Given its replicated
networks spanning large spatiotemporal scales, this dataset is well-suited for
analyzing network variability.

We built a metaweb of binary interactions by aggregating all local interactions,
which gave us a regional network composed of $274$ species and $1080$
interactions. 

\subsection{Metawebs of probabilistic interactions}

We converted these binary regional interactions into probabilistic ones using
simple assumptions. Our aim is not to estimate precise probability values, but
to create plausible metawebs of probabilistic interactions for our illustrative
examples. 

We created two metawebs of probabilistic interactions by employing constant
false positive and false negative rates for all regional interactions. In the
first metaweb, we set both false positive and false negative rates to zero to
prevent artificially inflating the total number of interactions, enabling a more
accurate comparison with binary interaction networks. This gave us a probability
of regional interaction of $1$ when at least one interaction has been observed
locally and of $0$ in the absence of any observed interaction between a given
pair of species. This metaweb was used in Box~\ref{box2.2}. 

In the second metaweb, we introduced a $5\%$ false positive rate to account for
spurious interactions and a $10\%$ false negative rate to address the elevated
occurrence of missing interactions in ecological networks
(\cite{Catchen2023Missinga}). We believe these rates represent reasonable
estimates of missing and spurious potential interactions, but confirming their
accuracy is challenging due to the unavailability of data on the actual
feasibility of interactions. Observed interactions were thus given a probability
of regional interaction of $95\%$, whereas unobserved ones were assigned a
probability of $10\%$. This metaweb was used in Boxes~\ref{box2.3} and
\ref{box2.5}.

\subsection{Local networks of probabilistic interactions}

We built local networks of probabilistic interactions using the taxa found in
the empirical local networks and attributing pairwise interaction probabilities
based on the metawebs of probabilistic interactions $P(M_{i, j})$ and a constant
value of $P(L_{i, j, k}|M_{i, j})$ across interactions:

\begin{eqnarray}
    \label{eq:local_meta_sup}
    P(L_{i, j, k}) = P(L_{i, j, k} | M_{i, j})
    \times P(M_{i, j}).
\end{eqnarray}

We set all values of $P(L_{i, j, k}|M_{i, j})$ to $0.5$, $0.75$, or $1.0$
depending on the simulation. Intermediate values of $P(L_{i, j, k}|M_{i, j})$
around $50\%$ indicate considerable spatiotemporal variability, while higher
values close to $100\%$ indicate that regional interactions are nearly always
realized locally. 

\section{Additional methods for Box~\ref{box2.2}: Dissimilarity of local host-parasite networks}

\subsection{Dissimilarity between local networks and the metaweb}

We aggregated local networks of binary interactions by sequentially and randomly
selecting a number of local networks and aggregating both their species and
interactions. 

We compared the metaweb of binary interactions and the aggregated local networks
of binary interactions using the dissimilarity in species composition
$\beta_{S}$, and the dissimilarity of interactions between common species
$\beta_{OS}$ indices. Both dissimilarity indices were calculated based on the
number of items shared by the two networks ($c_{LM}$) and the number of items
unique to the metaweb ($u_M$) and the aggregated local network ($u_L$). The
$\beta_{S}$ dissimilarity index uses species (nodes) as items being compared,
while the $\beta_{OS}$ index assesses dissimilarity based on interactions
between shared species. Both indices were calculated following the $\beta_W$
index of \textcite{Whittaker1960Vegetation}: 

\begin{eqnarray}
    \label{eq:betadiv_network}
    \beta_W = \frac{c_{LM} + u_L + u_M}{(2 c_{LM} + u_L + u_M) / 2} - 1.
\end{eqnarray}

We repeated the aggregation process one hundred times and highlighted the median
dissimilarity values across simulations, as well as the $50\%$ and $95\%$
percentile intervals. 

\subsection{Aggregation of local networks of probabilistic interactions}

We aggregated local networks of probabilistic interactions similarly to the
networks of binary interactions, with the distinction that we also adjusted the
value of $P(L_{i, j, k})$ when sampling networks. The constancy of the
probability of regional interaction across the entire study area means that any
rise in the probability of local interaction is solely attributable to an
increase in $P(L_{i, j, k}|M_{i, j})$. We adjusted the value of $P(L_{i, j,
k}|M_{i, j})$ as follows. Let $L_1$ and $L_2$ be two local networks and
$L_{1,2}$ the aggregated network. If $P(L_{i, j, 1}|M_{i, j})$ and $P(L_{i, j,
2}|M_{i, j})$ are the probabilities that two potentially interacting taxa
interact respectively in $L_1$ and $L_2$, the probability $P(L_{i, j, 1,2}|M_{i,
j})$ that these taxa interact in the aggregated network $L_{1,2}$ is obtained
by: 

\begin{eqnarray}
    \label{eq:aggregate}
    P(L_{i, j, 1, 2}|M_{i, j}) = 1 - [1 - P(L_{i, j, 1}|M_{i, j})] \times [1 -
P(L_{i, j, 2}|M_{i, j})],
\end{eqnarray}

assuming independence between the interaction of the two taxa in different
networks. This equation represents the probability that the interaction is
realized in either (1) exclusively the local network $L_1$, (2) exclusively the
local network $L_2$ or (3) both, given that the two taxa have the biological
capacity to interact. 

We then calculated the probabilities of local interaction of the aggregated
networks using Eq~\ref{eq:local_meta_sup}. The value of $P(L_{i, j, k}|M_{i,
j})$ for each curve in Figure~\ref{fig:accumulation}c-d is the probability before
aggregating networks.  

\subsection{Calculation of the expected number of local interactions and connectance}

We investigated how the number of local interactions and connectance scale with
the number of sampled (aggregated) local networks. We calculated the expected
numbers of interactions by taking the sum of all binary or probabilistic
interaction values. Connectance was calculated as the ratio of the expected
number of interactions to the number of possible (non-forbidden) interactions.
Because our networks are tripartite, connectance was calculated as follows: 

\begin{eqnarray}
    \label{eq:co_tri}
    Co = \frac{I}{S_S S_G + S_G S_P},
\end{eqnarray}

where $I$ is the expected number of interactions, $S_S$ the number of Salix
species, $S_G$ the number of galler species, and $S_P$ the number of parasitoid
species in the network. 

\section{Additional methods for Box~\ref{box2.3}: Spatial and temporal scaling of interactions}

\subsection{Aggregation of local and regional networks of probabilistic interactions} 

Local probabilistic interactions were derived from probabilistic regional
interactions by setting the value of $P(L_{i, j, k}|M_{i,j})$ (the local
probability of interaction among potentially interacting species) to $1$,
ensuring a conservative comparison between aggregated local networks and
metawebs. Aggregated local and regional networks were obtained by aggregating
both the species and interactions found within a particular latitudinal window.
The values of $P(L_{i, j, k}|M_{i, j})$ in local networks remained at their
maximum value of $1$ following Eq~\ref{eq:aggregate}. Latitudinal windows had
different positions (central latitudes) and widths (latitudinal widths).

\subsection{Calculation of the expected number of interactions}

We calculated the expected number of local and regional interactions by taking
the sum of all probabilistic interaction values of the aggregated networks. 

\section{Additional methods for Box~\ref{box2.5}: Sampling for binary interaction networks}

\subsection{Sampling using regional interaction probabilities}

We sampled for binary interaction networks across space, predicting a binary
interaction network for each location in our dataset. We performed a single
Bernoulli trial for each pair of taxa based on their regional probability of
interaction: 

\begin{eqnarray}
    \label{eq:sample1_sup}
    M_{i, j} \sim {\rm Bernoulli}(P(M_{i, j})).
\end{eqnarray}

Every pair of taxa predicted to interact in this metaweb was treated as
interacting in all localized networks where they co-occurred, i.e. $L_{i, j, k}
= M_{i, j}$ when $X_{i,j,k} = 1$.

We performed between $1$ and $100$ simulations for each location to get a
distribution of networks of binary interactions sampled using regional
interaction probabilities.

\subsection{Sampling using local interaction probabilities}

We sampled binary interaction networks across space, predicting a binary
interaction network for each location in our dataset. We first generated
distinct probabilistic interaction networks for each location. The local
probability of interaction between potentially interacting species was set to
three different values: $P(L_{i, j, k}|M_{i, j}) = 1.0$, $P(L_{i, j, k}|M_{i,
j}) = 0.75$, and $P(L_{i, j, k}|M_{i, j}) = 0.50$. We then sampled each local
network of probabilistic interactions independently: 

\begin{eqnarray}
    \label{eq:sample2_sup}
    L_{i, j, k} \sim {\rm Bernoulli}(P(L_{i, j, k})).
\end{eqnarray}

We performed between $1$ and $100$ simulations for each location to get a
distribution of networks of binary interactions sampled using local interaction
probabilities. 

\subsection{Calculation of connectance}

We calculated the connectance of our predicted tripartite networks of binary
interactions following Eq~\ref{eq:co_tri}. We calculated the average connectance
across simulations for each location.

\subsection{Calculation of the mean squared logarithmic error (MSLE)}

The mean squared logarithmic error was calculated as follows: 

\begin{eqnarray}
    \label{eq:MSLE_sup}
    MSLE = \frac{\sum (log(\overline{Co_L}) - log(\overline{Co_M}))^2}{n},
\end{eqnarray}

where $\overline{Co_L}$ and $\overline{Co_M}$ are the average connectance across
simulations for each location, respectively for local and regional samples, and
$n$ is the number of locations.

\addtocontents{toc}{\protect\setcounter{tocdepth}{-1}}

\begingroup
\let\clearpage\relax
\printbibliography
\endgroup
\end{refsection}

\addtocontents{toc}{\protect\setcounter{tocdepth}{0}}

\endinput
%%
%% End of file `S_article1.tex'.