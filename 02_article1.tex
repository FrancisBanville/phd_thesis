%%
%% This is file `02_article1.tex',
%% generated with the docstrip utility.
%%
%% The original source files were:
%%
%% dms.dtx  (with options: `article')

%% To change chapter header dynamically from french to english, use
\entetedynamique
\setcounter{corA}{0} % Pour recommancer à compter les def,
                     % theo, etc. à partir de 1
 % Pour écrire un article en français
%% \francais
 % Pour écrire un article en anglais
\anglais
%% NOTE: La plupart des macros ont un nom en anglais.
%% P.ex. \adresse et \address fonctionnent et sont équivalents.
%% \revue=\journal
%% \auteur=\author
%% \titre=\title

%% Les contributions apparaîtront habituellement après
%% \maketitle (voir un peu plus bas). Selon les goûts, il est
%% possible de mettre les contributions
%% avant la page titre de l'article, simplement en les écrivant
%% directement ici. Par exemple :
 % \cleardoublepage
 % \pdfbookmark[chapter]{Contributions}{contrib1} % Remplacer par contrib2 pour l'article 2 etc.
 % {\bfseries\Large\noindent Contributions de <mon nom> et rôle joué par les coauteurs}
 % J'ai contribué en...
 %
 % Le rôle des coauteurs a été de...

%% Nom de la revue de publication
\revue{Une revue}
\article{Titre de l'article}\label{chapter1}
%% On peut se référer aux numéros de chapitre ou d'article comme suit.
%% Si on fait
%% \label{chap:article1},
%% alors \ref{chap:article1} donnera le numéro du chapitre. On peut ensuite faire
%% \labelart{art:article1}
%% et alors \ref{art:article1} donnera le numéro d'article.
%% Par exemple, si cette article est le premier article et le deuxième chapitre,
%% alors si on écrit
%% Voir le chapitre~\ref{chap:article1} (l'article~\ref{art:article1}).
%% deviendra
%% Voir le chapitre 2 (l'article 1).
%% Si on veut écrire « premier article » au lieu « article 1 », on peut
%% simplement faire
%% \ordinal{\ref{art:article1}}~article  % devient première article
%% ou
%% \Ordinal{\ref{art:article1}}~article  % devient Première article (avec la majuscule)
%% Si on est en mode \anglais, \ordinal écrire first, second,...

 % Contribution(s) peronnelle(s) à l'article et rôle joué par tous les coauteur·e·s
 %
 % Nécessaire seulement lorsque vous n'êtes pas seul·e auteur·e.
 % Les contributions peuvent apparaître ailleur dans la thèse.
 % Si \contributions est laissé vide (p.ex. si vous effacez
 % celui ci-bas), aucune contributions ne seront générées sur
 % la page titre de l'article. Vous pouvez alors mettre un
 % \newpage si vous souhaitez que les résumé et abstract soient
 % la page suivante.
 %
 % REMARQUE : À peu près toutes les constructions \LaTeX\ sont permises
 % dans les contributions.
 %
 % La commande admet une option [<entête>]
\contributions%[Mes contributions et le rôle des coauteurs]
{
J'ai fait telle chose à l'article. J'ai rédigé telle partie
etc.

  % Exemple avec un <itemize>
  %  \begin{itemize}
  %      \item Calcul de telle chose;
  %      \item Vérification de telle équation;
  %      \item Idée pour telle définition;
  %      \item Démonstration de tel théorème.
  %  \end{itemize}

    Le coauteur1 a suggéré telle chose.

    Le coauteur2 a fait telle calcul.\\[1cm]
}

%%% INFORMATIONS POUR LA PAGE TITRE
 % Premier auteur·e et adresse
\auteur{Hima Ginère}
\adresse{1252i rue complexe\\ Université du plan complexe}
 % Deuxième auteur·e et adresse (si différente de la première)
%%\auteur{Hana Lietick}
%%\adresse{4242 rue imaginaire\\ Universität von der gau\ss sche Zahlenebene}
%%
%% et ainsi de suite pour les autres auteurs

\maketitle

\begin{resume}{Mots clés}
  Le résumé en français.
\end{resume}

\begin{abstract}{Key words}
  The english abstract.
\end{abstract}

\section{Introduction}
%%
%% Le reste de l'article...
%%

Exemple de citation~: Consultez le \LaTeX\ companien
de Mittelbach {\it et al\/}~\cite{exemple}.

 % Pour générer la bibliographie à la fin
 % de l'article, utiliser la commande de la
 % classe <dms> \sectionbibliography{<nom du .bib>}.
 % Il y a aussi la possibilité d'utiliser le package
 % <chapterbib>, auquel cas on utilise simplement
 % \bibliography normalement.
 %
 % IMPORTANT : Dans tous les cas, il faut faire
 %    pdflatex these
 %    bibtex chapitre1
 %    bibtex chapitre2
 %    .
 %    .
 %    .
 %    bibtex chapitreN
 %    pdflatex these
 %    pdflatex these
 %
 % où <these> est le nom du .tex principal
 % (qui contient le \documentclass).
 % bibtex a besoin du .aux de chapitre1 et
 % non du .tex. Il est parfois nécessaire
 % d'effacer le .aux et de recommencer la
 % compilation du début.
%%\bibliographystyle{amsplain} % style plain anglais ou
%% \bibliographystyle{amsplain-french} % style plain francais
%%\bibliographystyle{<style>} % autre
%% \sectionbibliography{ref.bib} %Donner le nom du .bib

\endinput
%%
%% End of file `02_article1.tex'.
