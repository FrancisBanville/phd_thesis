%%
%% This is file `introduction.tex',
%% generated with the docstrip utility.
%%
%% The original source files were:
%%
%% dms.dtx  (with options: `intro')
%% Example TeX file for the documentation
%% of the jurabib package
%% Copyright (C) 1999, 2000, 2001 Jens Berger
%% See dms.ins  for the copyright details.
%% 
%%% ====================================================================
%%%  @LaTeX-file{
%%%     filename        = "dms.dtx",
%%%     author    = "Nicolas Beauchemin, Damien Rioux-Lavoie, Victor Fardel, Jonathan Godin",
%%%     copyright = "Copyright (C) 2000 , DMS
%%%                  all rights reserved.  Copying of this file is
%%%                  authorized only if either:
%%%                  (1) you make absolutely no changes to your copy,
%%%                  including name; OR
%%%                  (2) if you do make changes, you first rename it
%%%                  to some other name.",
%%%     address   = "Département de Mathématiques et de Statistique",
%%%     telephone = "514-343-6705",
%%%     FAX       = "514-343-5700",
%%%     email     = "aide@dms.umontreal.ca (Internet)",
%%%     keywords  = "latex, amslatex, ams-latex, theorem",
%%%     abstract  = " Ce fichier est un package conçu pour être
%%%                  utilisé avec la version de LaTeX2e 1995/06/01. Il
%%%                  est prévue pour la classe ``amsbook''. Il en
%%%                  modifie le format des pages, l'entête des
%%%                  sections, etc, afin d'être  conforme au modèle de
%%%                  mémoire de maîtrise de l'Université de
%%%                  Montréal. Finalement ce fichier est grandement
%%%                  inspiré du fichier amsclass.dtx.",
%%%     docstring = "The checksum field contains: CRC-16 checksum,
%%%                  word count, line count, and character count, as
%%%                  produced by Robert Solovay's checksum utility."}
%%%  ====================================================================


 % Utilisez la macro de langue appropriée.
 % Noter que toutes les parties du document,
 % à part les articles, doivent être en français.
 % Pour rédiger une thèse en anglais, il faut
 % une permission. Consulter le guide de présentation
 % des mémoires et des thèses pour de l'information
 % plus détaillé et à jour.
\francais   %ou
%%\anglais
\chapter*{Introduction}

...introduction...


\endinput
%%
%% End of file `introduction.tex'.
