%%
%% This is file `01_introduction.tex',
%% generated with the docstrip utility.
%%

% Utilisez la macro de langue appropriée.
\francais   
\doublespacing
%ou
%%\anglais
\chapter*{Introduction}


%%%%% Mise en contexte 

\section{L'écologie des réseaux pour mieux comprendre les écosystèmes complexes}

Les systèmes complexes sont partout autour de nous. Considérez les transactions
effectuées sur les marchés financiers mondiaux (\cite{Anderson2018Economy}), où
une décision prise à Toronto peut avoir des répercussions à Londres et à Bombay
en quelques secondes seulement. Pensez également au réseau de neurones dans
notre cerveau (\cite{Sporns2011Human}), qui orchestre des milliers de signaux
électriques à chaque instant pour créer notre expérience subjective du monde. Ou
encore, à une colonie de fourmis (\cite{Bonabeau1999Swarm}), où chaque individu
suit des règles simples, mais qui ensemble génère des comportements complexes
tels que la construction de nids et la défense du territoire. Les systèmes
complexes régissent plusieurs phénomènes importants, mais souvent difficiles à
analyser, dans notre quotidien et la nature. 

Quel que soit leur objet d'étude, les systèmes complexes sont constitués de
nombreux éléments qui interagissent entre eux (\cite{Rind1999Complexity}). Ces
interactions impliquent des échanges directs d'information, de matière ou
d'énergie (\cite{Ladyman2013What}), donnant lieu à des propriétés émergentes de
niveau supérieur (\cite{Foote2007Mathematics}). Caractéristiques fondamentales
des systèmes complexes, ces propriétés émergentes (p. ex. un krach boursier, la
conscience humaine ou le comportement d'un superorganisme) ne sont pas
déductibles des éléments pris individuellement (\cite{Nielsen2000Emergent}). Le
désordre apparent des éléments et de leurs interactions génère donc une
structure ordonnée qui est souvent au cœur des recherches sur les systèmes
complexes. 

Pour faciliter l'étude des systèmes complexes, il est courant de les représenter
sous forme de réseaux, où les éléments forment des nœuds et les interactions,
des liens entre ces nœuds (\cite{Newman2003Structure}). Dans cette thèse, je me
suis intéressé plus particulièrement aux réseaux d'interactions entre espèces,
qui servent d'outil pour mieux comprendre les écosystèmes complexes. Bien ancrée
dans la science des réseaux et la théorie des graphes, l'écologie des réseaux
fournit un cadre d'analyse pour étudier le fonctionnement des écosystèmes et
leur résilience face aux changements environnementaux
(\cite{McCann2007Protecting}, \cite{McCann2011Food}, \cite{Proulx2005Network},
\cite{Rooney2012Integrating}). Elle y parvient en offrant des méthodes pour
mesurer les interactions entre espèces (p. ex. \cite{Jordano2016Sampling}) et
analyser la structure émergente des réseaux d'interactions (p. ex.
\cite{Delmas2019Analysing}), nous permettant ainsi de mieux comprendre les
caractéristiques de ces systèmes complexes. 

La théorie de l'entropie maximale des réseaux trophiques, dont je jette les
bases dans cette thèse, propose une nouvelle approche en écologie des réseaux.
Reposant sur la complexité des interactions entre espèces, elle permet de
prédire la structure émergente des réseaux d'interactions à partir d'un nombre
limité d'informations écologiques. Avant d'exposer cette théorie, il convient
d'examiner certains concepts clés de l'écologie des réseaux afin de mieux saisir
la nature des écosystèmes complexes. Les sous-sections suivantes traitent de la
nature et des mesures des interactions entre espèces avant d'aborder la
structure émergente des réseaux d'interactions. Ces précisions permettront de
mieux comprendre les fondements et domaines d'application de cette théorie.  

\subsection{Qu'est-ce qu'une interaction entre espèces?} 

Une communauté biologique est composée d'espèces différentes interagissant entre
elles dans un espace donné (\cite{Stroud2015Community}). Par exemple, les
milieux forestiers tempérés du Québec sont riches en espèces animales et
végétales. À cause des changements climatiques et de la fragmentation des
habitats naturels, on y retrouve maintenant des tiques \textit{Ixodes
scapularis} infectées par la bactérie \textit{Borrelia burgdorferi}, responsable
de la maladie de Lyme (\cite{Ogden2009Emergence}, \cite{Simon2014Climate}). Les
tiques juvéniles se nourrissent sur des petits mammifères, dont la souris à
pattes blanches (\textit{Peromyscus leucopus}) qui est un réservoir compétent de
la bactérie (\cite{Donahue1987Reservoir}). Les tiques adultes, quant à elles, se
nourrissent et s'accouplent sur les cerfs de Virginie (\textit{Odocoileus
virginianus}, \cite{Lane1991Lyme}). Ces deux espèces (\textit{P. leucopus} et
\textit{O. virginianus}) s'alimentent à leur tour de glands de chêne
(\cite{McShea1993Variablea}, \cite{Elkinton1996Interactions},
\cite{Wolff1996Population}, \cite{McShea2000Influence}), ce qui peut expliquer
qu'il y ait une relation entre l'abondance des tiques et celle des glands
(\cite{Ostfeld2006Climate}). Comprendre l'écologie et l'épidémiologie de la
maladie de Lyme nécessite donc une compréhension approfondie des interactions
entre espèces au sein des communautés biologiques où la tique \textit{I.
scapularis} est présente.

Plus généralement, il existe différents types d'interactions entre espèces au
sein d'un écosystème. Les interactions antagonistes, telles que les interactions
hôtes-parasites, prédateurs-proies et plantes-herbivores, profitent à une espèce
au détriment de l'autre. Les interactions hôtes-parasites procurent un habitat
et une source d'alimentation au parasite au prix du \textit{fitness} de l'hôte,
alors que les interactions prédateurs-proies et plantes-herbivores transfèrent
l'énergie de la ressource vers le consommateur, entraînant généralement la mort
de la proie mais pas nécessairement celle de la plante. Les interactions
mutualistes, telles que les interactions plantes-pollinisateurs, profitent au
contraire aux deux espèces. Le pollinisateur se nourrit du nectar de la plante,
qui en retour en profite pour disperser son pollen à des fins de reproduction.
Les réseaux d'interactions entre espèces peuvent être composés d'un seul type
d'interaction (p. ex. les réseaux trophiques composés exclusivement
d'interactions prédateurs-proies et plantes-herbivores) ou de plusieurs (p. ex.
les réseaux multicouches, \cite{Pilosof2017Multilayer}), selon le niveau de
complexité écologique représenté.

Construire un réseau d'interactions demande de faire des choix concernant les
interactions, ce qui impactera notre interprétation ses propriétés émergentes.
Étant donné le nombre élevé d'organismes et la diversité de leurs interactions,
il est nécessaire de simplifier notre représentation d'une communauté biologique
en mettant de l'avant les processus écologiques qui nous intéressent (p. ex.
propagation de maladies infectieuses, flux d'énergie dans l'écosystème,
reproduction végétale). Au delà du type d'interaction, les décisions concernant
le niveau d'organisation des nœuds (individus ou espèces), la mesure des liens
(binaire ou quantitatif) et l'échelle spatiale (local ou régional) déterminent
également les aspects de la complexité écologique représentés. Ces décisions
dépendent autant des objectifs de recherche que des données disponibles.

\subsubsection{Interactions entre individus ou espèces: une question d'organisation} 

\begin{itemize}
    \item Les individus sont à la base des interactions
    \item On peut vouloir grouper les individus en populations ou espèces
    \item On peut aussi les grouper à des niveaux d'organisation supérieurs (p.ex. espèces trophiques)
\end{itemize}

Ecological networks are mathematical representations of interactions occurring
within a biological community. Interactions typically occur at the individual
level [Guim20a]. For instance, pollination occurs between pollinator and plant
individuals, and predation between predator and prey individuals. However,
ecological networks are typically representations of *species* and their
interactions [DelmBess19], species interactions being aggregated among
individuals. In binary networks, two species are linked if at least one of their
individuals interact. Interactions can also be quantitative if they represent,
in the case of food webs, the energy flow between two species [Post02a;
DelmBess19], or probabilistic if they represent the probability that two
species interact [PoisCirt16].

\subsubsection{Interactions binaires ou quantitatives: une question de mesure} 

\begin{itemize}
    \item Les réseaux binaires nous disent si l'interaction a lieu
    \item Les réseaux quantitatifs peuvent vouloir dire différentes choses (flux d'énergie, impact d'une espèce sur une autre)
\end{itemize}

\subsubsection{Interactions locales ou régionales: une question d'échelle spatiale} 

\begin{itemize}
    \item Les réseaux locaux nous disent si l'interaction est réalisé 
    \item Les réseaux régionaux nous disent si l'interaction a un potentiel d'être réalisé
\end{itemize}

Studying and understanding
ecological networks require that we understand their relation with spatial
scale.  Species–area relationships (SARs) describe how species richness varies
with habitat area. According to Deng09, a power-law model is an appropriate
type of SARs. In this model, the total number of species $S$ in a community
increases with the local area $A$ according to the equation:

$$ S = c \times A^z, $$ {eq:area}



where $c$ and $z$ are positive constants.

Because the number of links $L$ in a food web scales with the number of species
(see Annex 1), $L$ also scales with habitat area [BrosOstl04]. Large areas
typically contain more species, that globally establish more interactions.
According to the flexible links model (see Annex 1), the association between $L$
and $A$, however, depends on the proportion of realized links $p$, which varies
from one community to another. In other words, the association between $L$ and
$A$ can take multiple forms, and it is not clear how this could influence the
outcomes of any spatial food-web simulation.

Not only $L$ scales with $A$, but multiple measures of food-web structure also
vary with spatial scale [BrosOstl04; ThomTown05]. This could directly be due
to the scaling of species richness with area, but this could also occur because
of other ecological phenomena [GaliLurg18].

Ecological networks cannot be indefinitely large, however. Abnormally large food
webs are typically unstable [May72] and prone to collapsing. This
imposes a limit to their size, and perhaps as well to the area that food webs
typically cover.

%%%%%%%%%% 

Moreover, analyzing ecological networks can help us understand how biological
communities function and respond to changes [see for example PascPasc06;
GomePerf11]. For instance, many ecological networks are sensitive to the loss
of keystone species [JordBasc03] and to the introduction of successful invasive
species [AizeMora08; LopeHaye07]. Understanding species interactions and the
underlying structure of their ecological networks is essential to assess the
potential harm of climate change [TyliDidh08; GilmUrba10; ArauRoze11],
habitat loss [EvanPoco13], and other environmental catastrophes on ecosystems.
In short, species interactions are closely related to the ongoing biodiversity
crisis [ValiAize19].  

Network
ecology has established itself as a proper approach in the study of biological
communities and their inherent complexity [BorrMood14; DelmBess19]. Species
interactions can provide meaningful ecological information, as they notably
underlie how an ecosystem functions and responds to changes [PoisStou16]. One
of network ecology's major challenges, however, lies in the considerable
difficulty to sample data on interspecific interactions [Jord16; Jord16a]. As
a result, such data are scarce, and this raises substantial challenges in the
study of ecological networks, especially at a large scale. Simulations are one
way to tackle this ongoing difficulty efficiently.

Ecological networks are dynamic entities whose
structure can be altered following an ecological perturbation [TyliTsch07]. For
example, the introduction of an invasive species [SmitMoor17] or the
extirpation of one [DunnWill02a] could drastically alter trophic links in a
food web. However, some ecological networks have a structure more stable or
resilient than others. For instance, food-web stability could increase with
connectance [DunnWill02a].

One of the biggest challenge humanity will ever have to face is climate change.
Global warming will have an unprecedent impact on species distribution and
biodiversity [BotkSaxe07]. With habitat loss and fragmentation, climate change
is expected to exacerbate the ongoing biodiversity crisis [OpdaWasc04;
HellZava09]. Species ranges are expected to change, and these changes might
impact the functioning of any type of ecological networks [TyliLali10].
Interspecific interactions can vary because of species turnover, but they can
also vary between two co-occurring species notably because of a change in their
relative abundance or in their phenology [Vazq05; CanaMouq12; PoisCana12]. In
mutualistic networks, these shifts depend, among other things, on the rise of
temperature due to human activities [HeglNiel09]. In short, the amplitude of
biodiversity loss varies locally, and can be accentuated depending on the
humanity's effort to mitigate climate change and habitat loss. The
Intergovernmental Panel on Climate Change (IPCC) is the authority in that
respect, and their emissions scenarios are still today the reference in the
modeling of the potential impacts of climate change. Because food-web structure
is so closely related to species richness, a change in local species richness
could have hurtful consequences on the structure of ecological networks, and
thus on their functioning.

The impact of climate change and habitat loss on network structure has been
studied empirically at the local scale [see for example WoodBrow12]. However,
at the best of my knowledge, no rigorous simulation has been conducted at the
global scale. This would raise awareness on ecosystem fragility and help
identify hotspots of highly vulnerable food webs.

The main challenge in the conduction of a large-scale study of ecological
networks is the lack of data on interspecific interactions. However, network
ecologists often summarize entire networks with a few measures only because of
the information they provide on the functions, dynamics, and stability of
biological communities. It is thus unnecessary to know exactly who eats who in a
community if one is mainly interested in the emergent structure of a food web.


\subsection{Quelle est la structure émergente des réseaux d'interactions?} 

\begin{itemize}
    \item Il existe plusieurs mesures sur les réseaux 
    \item Écologie des réseaux (stabilité des écosystèmes, résilience, fonctions)
\end{itemize}

Cette thèse jette les bases d'une théorie permettant de reconstituer la
structure émergente des réseaux d'interactions entre espèces à partir d'un
nombre limité d'informations écologiques. Cette théorie repose sur la prémisse
selon laquelle les interactions entre espèces 

These three types of networks (binary, quantitative, and probabilistic) are
analyzed using different measures. DelmBess19 and PoisCirt16 outline and
propose measures to analyze binary and probabilistic networks, respectively.
Quantitative food webs can be analyzed at the species or network level using
measures defined in BersBana02. For deterministic networks (i.e.
non-probabilistic), the most commonly used measures are the total number of
species $S$, the total number of links $L$, connectance $L/S^2$, and the degree
distribution $p_k$ [DelmBess19]. The degree of a species is the number of
interactions it has with other species in the network, and the degree
distribution is the probability distribution of such degrees. Similarly, the
joint degree distribution is the joint probability distribution for in-degrees
(number of incoming interactions) and out-degrees (number of outgoing
interactions). Modularity and nestedness are two other measures widely used in
network ecology [DelmBess19]. Overall, network structure is an important source
of information on the functioning, stability, and dynamic of biological
communities [ProuProm05a; PascPasc06; DelmBess19].

Because species interactions are difficult to sample empirically [Jord16c],
analyzing network structure empirically can be challenging or, in some
instances, infeasible. Indeed, the interaction between two cooccurring species
can be observed *in situ* or inferred for example using biopsies or
metagenomics. It can also be inferred from literature, i.e. by modeling the
*potential* interaction between two cooccurring species that have already been
observed interacting. However, because species interactions vary spatially
[PoisStou15], inferring species interactions entirely from literature generates
the *metaweb* of the community, i.e. the network of all potential interactions
that might not occur locally [TyliMorr17b; SaraMari18]. Another approach is to
predict species interactions using statistical and machine learning tools.

A degree
distribution is a discrete probability distribution $p_k$ that a species
establishes $k$ links in a food web, with $k$ varying between $1$ and $S$.

These authors (Barbarasi) observed that a wide range of
networks, as different in nature as could be, unexpectedly present strong
topological similarities.

Like
for non-ecological networks, a rich scientific literature has sought to
understand the mechanisms shaping ecological networks [see for example Will11].
Are networks shaped the way they are mostly because of stochasticity and
topological rules? What is the role of biological processes in the shaping of
ecological networks? Uncovering these mechanisms typically requires the use of
null models, relevant measures of network structure, and empirical data.
Deviations from null models, i.e. from models that do not rely on given
ecological mechanisms, are in that respect highly informative [FlorMeye11a;
FortStou10a].

\subsubsection{Propriétés des espèces} 

\begin{itemize}
    \item Peuvent nous informer du rôle des espèces au sein des réseaux (distribution de degrés)
\end{itemize}

\subsubsection{Propriétés émergentes} 

\begin{itemize}
    \item Peuvent nous informer de la stabilité et résilience des écosystèmes (modularité, connectance)
    \item Peuvent nous informer du fonctionnement des écosystèmes 
    \item Complexité 
\end{itemize}

The emergent structure of ecological networks is a unique number, or a set of
numbers, that summarizes a particular property of a network [DelmBess19]. The
number of nodes (or number of species $S$) and the number of edges (or number of
interactions $L$) are among the most fundamental ones. Until recently, the
relationship between these two quantities best followed a power law
[BrosOstl04; RiedRall10]:

$$ L = b \times S^a, $$ {eq:powerlaw}

where $a$ and $b$ are positive constants.

However, we recently observed that power laws do not account for the minimum
$S-1$ and maximum $S^2$ numbers of links in ecological networks. The power law model had the undesirable property of
making predictions that were below the minimal number of links.

We also observed that only a proportion $p$ of the $S^2 - (S-1)$ *flexible*
links in a food web are in fact realized. We modeled this proportion from a
Bayesian approach, and obtained a probability distribution for the total number
of links in a food web from its number of species. Our model fitted surprisingly
well, outperforming the popular power law model. In addition, and perhaps more
importantly, our predictions never trespassed the boundaries set by ecological
principles.

Food webs and
other ecological networks can be simplified into various measures describing
their emergent structure [DelmBess19]. These provide valuable information on
the functioning of biological communities [McCa11; ThomBros12; PoisStou16].
Two of these measures are linkage density ($L_D = L/S$) and connectance ($C_O =
L/S^2$). Linkage density is the expected number of interactions that would be
added into a network after the introduction of a novel species [DelmBess19].
This value is half of the average degree, which is the number of links a species
is expected to establish in a network. Connectance is, on the other hand, the
proportion of realized links among the maximum $S^2$ [DelmBess19]. It measures
the saturation of a network with regards to potential interactions. Connectance
is perhaps the most useful value in network ecology. It has been associated with
the stability [DunnWill02; MontPimm06a] and dynamics of ecological communities
[VieiAlme15]. It has also been shown that an invasive species has more
difficulty to enter a network of high connectance [SmitMoor17]. Connectance is
so fundamental that it covaries with the majority of measures commonly used by
network ecologists [PoisGrav14].

The degree distribution is another fundamental measure that has been associated
with many components of food-web structure [Will11a; PoisGrav14]. The degree
of a species is the total number of interactions it establishes, and the
distribution of such values typically follows a power law [BaraAlbe99a].
According to BaraAlbe99, the probability $P_k$ that a node has a total degree
of $k$ is:

$$ P_k = \frac{k^{-\gamma}}{\zeta(\gamma)},$$ {eq:degree}

where $\zeta$ is the Riemann-zeta function and $\gamma$ the degree exponent.

Various measures describe specific aspects of food-web structure [DelmBess19].
The mean
generality (the average number of preys per predators), the mean vulnerability
(the average number of predators per preys), along with the proportions of basal
(species without preys), top (species without predators), and intermediate
(species with both preys and predators) are some of the key features that will
be explored. These measures, along with the maximum and mean trophic levels,
provide meaningful information on the flux of energy in a food web
[ThomBros12]. The energy of a prey is partially transferred to its predators,
and how this energy circulates in a food web depends notably on the proportion
of species at each trophic level [ThomBros12]. The ratio between the maximum
and mean trophic levels can also be computed. It describes the shape of the
trophic level distribution, by measuring how high is the maximum in comparison
to the mean. Trophic levels have been extensively studied by ecologists, notably
because of their association with the functions of biological communities
[PoliStro96]. Another measure that will be explored in this project is
nestedness, which describes, in simple terms, how the diets of specialist
species are subsets of the ones of generalists [StanKopp13a]. Nestedness has
been negatively associated with network stability [AlleTang12a; StanKopp13a].
Nestedness can be computed in different ways [UlriAlme09a], but it is typically
associated with the spectral radius, which is the largest (absolute) eigenvalue
of the matrix of interactions (the adjacency matrix) [StanKopp13a]. We chose the ones presented
above because of the ecological information they provide and because of
computational efficiency. These measures could all be computed with relative
ease for simulated food webs.

%%%%% Stochasticité des interactions 

\section{Incertitude et variabilité des interactions entre espèces (3 pages)}

\begin{itemize}
    \item Le défi principal en écologie des réseaux est le manque de données
\end{itemize}

\subsection{Incertitude des interactions: La peur de manquer quelque chose d'important} 

\begin{itemize}
    \item On ne peut pas échantillonner toutes les interactions (faux positifs et négatifs)
    \item On n'est pas certain des conditions permettant aux interactions d'être réalisés
\end{itemize}

\subsection{Variabilité des interactions: Quand le lion s'endort} 

\begin{itemize}
    \item Deux espèces qui co-occurrent ne vont pas nécessairement interagir
    \item Conditions pour qu'une interaction locale ait lieu 
    \item Exemples de variation dans le temps et l'espace due aux conditions environmentales 
\end{itemize}

\subsection{Représentation probabiliste des interactions entre espèces} 

\begin{itemize}
    \item Nous pouvons représenter la variabilité et l'incertitude de manière probabiliste
    \item Par contre, cette probabilité est difficile à interpréter 
    \item C'est important de bien connaître la source d'incertitude puisque ça nous permet de savoir comment réduire cette incertitude
    \item C'est important de bien connaître la source d'incertitude puisque ça nous informe sur les processus écologiques 
\end{itemize}

%%%%% Modèles prédictifs 

\section{De la nécessité de développer des modèles prédictifs (3 pages)} 

\begin{itemize}
    \item Nous avons besoin de modèles prédictifs pour combler le manque de données 
    \item Ces modèles prédictifs doivent être explicites sur l'incertitude des interactions
    \item Il existe plusieurs modèles pour prédire les interactions et les réseaux, mais ces modèles peuvent être complexes
\end{itemize}

Although the rise in computational power and advances in machine learning have
significantly improved our ability to predict species interactions and the
structure of their ecological networks, insufficient data availability remains
an important limiting factors in our capacity to study and predict ecological
networks. Nevertheless, species interactions can and have previously been
predicted using null models, ecological models, statistical models, and machine
learning methods.

Previous authors predicted missing interactions in ecological networks. For
instance, DesjLaig17 used machine learning (ML) algorithms to predict which
species are more likely to be preyed upon by a predator based on the
"preferences" of similar predators. Moreover, PichBore20 predicted pairwise
interactions using species traits and various ML algorithms. The prediction of
ecological networks can also be conducted at the network level. Ecological
models are based on given ecological mechanisms and include the cascade model
[CoheBria12], the niche model [WillMart00a], and the nested-hierarchy model
[CattBers04a]. GravPois13 used the niche model of WillMart00a to predict the
adjacency matrix using predator-prey body size relationships. Predicting the
topology of an ecological network before pairwise interactions can in some
instances be preferable. Network structure can indeed constrain the space of
feasible ecological networks and, on another level, of pairwise interactions.

The number of links $L$ in food webs is typically predicted from the number of
species $S$ using the links-species scaling law of CoheBria84 ($L=bS$, with
$b\approx 2$), the constant connectance model of Mart92b ($L=bS^2$, with $0 < b
< 1$ ), or the power law model of BrosOstl04a ($L=bS^a$). However, our flexible
links model [MacDBanv20b] fits data better than these models, while respecting
biological constraints on the minimal $S-1$ and maximal $S^2$ numbers of links
in food webs. The flexible links model predicts the probability that flexible
links are realized, i.e. the proportion of the $S^2 - (S-1)$ potential links
that are realized. In fig:flexiblelinks, we compared the predictive capability
of these four models in a completely Bayesian framework.  

The species pool can be predicted using diverse species distribution modeling
(SDM) methods along with observational data [PhilAnde06b]. Spatially-explicit
predictions of species interactions can be conducted from the species pool and
adequate data [e.g. traits, phylogenetic relationships, number of species].

To the best of my knowledge, few studies aimed at forecasting ecological
networks (in the future), in contrast to predicting them (in the present).
However, predictive models can be turned into forecasting ones if we use
forecasted measures as inputs to the predictive models.

While studying ecological networks has proven its worth in environmental
protection and sustainability, poor data availability makes it difficult to
analyze them efficiently, especially at large spatial scales. This scarcity of
data arises from the inherent difficult in sampling species interactions
[Jord16c]. On the same basis as monitoring other biodiversity variables such as
species richness and total biomass is considered of environmental importance
[PereFerr13; SchmJull15], monitoring species interactions is also important.
Even if the amount of species interaction data has increased in recent years
[ProuProm05a], there are important biases in sampled data [AguiNewm19;
PoisBerg20], and the rate at which data is accumulating remains largely
insufficient to allow us to empirically answer many ecological questions of
interest.

Existing methods of inference of ecological networks from literature review or
databases, such as the [Global Biotic Interactions
(GloBI)](https://www.globalbioticinteractions.org/about) database, are
unsatisfying because pairwise species interactions can and often do vary
spatially [PoisCana12]. Indeed, two species interacting at a given location
might not interact at another location because of differences in traits,
abundance or stochastic processes [PoisStou15]. Building the metaweb, i.e. the
network of potential species interactions, does not account for this spatial
variability. Predicting species interactions and ecological networks spatially
using another method is thus necessary. Existing predictive methods of species
interactions and of the structure of ecological networks have been discussed in
the previous subsection. To the best of my knowledge, there is currently no
method that can predict network structure across space rigorously without
directly relying on species interactions data. Just like for other types of
networks, simple topological and statistical rules that do not rely on specific
ecological mechanisms underlie the structure of ecological networks, and
significant deviations from these null models, if any, suggest that other
processes are also of importance for deviating networks [FlorMeye11a;
FortStou10a; DelmBess19]. This topological method should mainly rely on
monitorable and essential biodiversity variables to maximize its applicability.
Such topological predictive method is highly needed, as it would greatly
facilitate the study of ecological networks at large spatial and temporal
scales. Its integration with existing predictive methods of ecological networks
through space and with species distribution modeling (SDMs) could become a
standard procedure in network macroecology (see our paper in preparation in the
Appendices). On the other hand, predicting the structure of ecological networks
might not be sufficient to study the potential consequences of given
environmental events on species interactions networks. To this end, forecasting
models of ecosystems in general [Carp02], and of ecological networks in
particular, are deeply needed. My thesis aims at filling this gap in the prediction and forecasting of the
structure of ecological networks across space. 

The scarce data regime characterizing network ecology limits our ability to
understand ecological networks, especially at a large spatial scale. Yet, the
structure of ecological networks is a gold mine of information on biological
communities [ProuProm05a; PascPasc06; DelmBess19], with notable applications
in conservation, biological control, habitat management, and climate change
[Memm09; TyliLali10b; TyliBinz14; HarvGoun17]. In the absence of sufficient
data, predicting ecological networks efficiently is our best option to
understand them and meet numerous environmental challenges.

Reliable forecasting of biodiversity is important, as it can greatly improve
environmental decision-making processes by providing stakeholders with useful
ecological knowledge [ClarCarp01]. In that respect, a rich literature has
emerged aiming at forecasting biodiversity and species distributions accurately
(see for example ArauNew07; BotkSaxe07a; UrbaBoce16). As ecology enters a
*data-rich* era, where collecting real-time environmental data is becoming
increasingly commun, ecological forecasting is becoming more reliable and
precise [LuoOgle11]. Given that species interactions vary both in space and
time [see for example DupoPadr09; Forr14], and given the role of ecological
networks in ecosystem functioning [see for example PascPasc06; GomePerf11],
forecasting ecological networks is equally as important as forecasting species
diversity. In spite of this, very few forecasting studies on ecological networks
have been conducted. This is in part due to the difficulty to sample species
interactions empirically [Jord16c]. Indeed, sampling a given ecological network
over time is almost never done in practice, as it would require significant time
and (financial) resources. Nevertheless, DeylMay16 measured and modeled the
temporal variety of species interactions in a small American lake, but their
method is not realistically applicable to more complex ecosystems. Another
approach is to predict how ecological networks respond to given changes. For
example, DunnWill02e evaluated the impact of biodiversity loss on food-web
structure by randomly removing species with different functions in their
respective networks. These authors, however, did not make any *prevision* on the
structure of given ecological networks.

Developing a simple yet rigorous model for predicting ecological networks is
essential, especially considering that ecologists currently cannot rely on
species interactions data to conduct many spatial network analyses. Indeed,
sampling species interactions is difficult, and as a result data is very scarce
in network ecology. Species interactions vary spatially [PoisCana12;
PoisStou15], and relying on species interactions databases alone (e.g. the
[Global Biotic Interactions
(GloBI)](https://www.globalbioticinteractions.org/about) database) might be
misleading as many pairwise interactions might not actually occur locally. In
addition, ecological networks are the backbones of ecosystems, and understanding
their structure will help us understand how they function and respond to changes
[PascPasc06]. This is particularly important considering that ecosystems are
changing at an alarming rate [see for example ButcWalp10].

Three information
are needed in order to simulate a food web: the number of species $S$, the
number of links $L = f(S)$, and a food-web model. Because $L$ can be predicted
from $S$, knowing the species richness and having an adequate food-web model are
nevertheless enough. The role of a food-web model is to distribute directed
links among the $S^2$ ordered pairs of species according to an ecological
mechanism [PascDunn05]. Because species identities are unknown in such
simulations, the resulting food webs depict interactions between unlabeled
species. As a matter of fact, it might not be necessary to know exactly who eats
who in a food web when we are interested in a single measure that could describe
a specific aspect of a network. Indeed, knowing the emergent structure of a food
web is necessary and sufficient to tackle various ecological questions,
especially those related to the dynamics, stability, and functions of biological
communities [McCa11; ThomBros12].

Several food-web models have been developed over the years to simulate food webs
using different ecological principles [PascDunn05]. Three of them, that will be
used in this project, are the cascade model [CoheNewm85], the niche model
[WillMart00], and the nested-hierarchy model [CattBers04]. In these three
models, an uniform random value, between 0 and 1, is first assigned to each of
the $S$ species. In the cascade model, this value represents body size
[CoheNewm85]. In this model, species have a probability $P$ of feeding on any
species which have a lower value. Here, cannibalism is not allowed, as well as
feeding on bigger species. In the niche model, this value now represents an
ecological niche [WillMart00]. A species will feed on all species whose value
is within a determined range from its own. In ecological terms, a species
exclusively feeds on the ones that have a similar niche, including itself. The
random values also represent ecological niches under the nested-hierarchy model
[CattBers04]. Without entering into details, this model relaxes the continuity
property of the niche model. It also takes into account, before assigning an
interaction to a species, the feeding links of other species similar to it in
terms of previously assigned preys. Because such food-web models all have
trivial drawbacks, using an ensemble of models might be more appropriate to
simulate food webs.

Because the number of links in a food web can be  predicted from its total
number of species, local food webs could be simulated worldwide using an
appropriate food-web model. Indeed, even if data on interspecific interactions
are particularly scarce, this is not the case for species distribution data.
Data on ecological interactions are indeed orders of magnitude harder to sample
than are data on species occurrence [Jord16; Jord16a]. In fact, species
occurrence is perhaps one of the most highly sampled type of ecological data,
thanks notably to citizen science and the relative easiness to collect them
[BonnCoop09; HochFink12]. Because of the quantity, quality, and openness of
species distribution data, local species richness have been extensively
estimated worldwide, and this could be used in the simulation of local food
webs.

%%%%% Trophique-METE

\section{Vers une théorie de l'entropie maximale des réseaux trophiques (6 pages)}

\subsection{Le principe d'entropie maximale} 

\begin{itemize}
    \item Ce que le principe d'entropie maximale nous permet de connaître
    \item Pourquoi l'utiliser
    \item Comment ça fonctionne, grossièrement
\end{itemize}

At the core of many theories and methods in statistical mechanics, with many
applications in the earth and life sciences [OzawOhmu03; MartSele06], the
principle of maximum entropy (MaxEnt) is a powerful method of inference of
probability distributions. MaxEnt is used to derive the *least-biased*
distribution subject to a given set of constraints on that distribution
[Jayn57; Jayn57a]. In other words, a probability distribution is derived
solely from prior knowledge of a system, without relying on arbitrary or biased
assumptions on the state of that system.

The principle of maximum entropy states that our best guess of a probability
distribution is the one that maximizes entropy given our set of constraints
(e.g. on the range, mean or variance of the distribution). Information entropy
measures the level of information given by the outcome of a stochastic event. According to
Khin13 and Beck09, information entropy $H$ should satisfy four properties
(discrete case): (1) $H$ should be a function of a probability distribution
$p_n$ only; (2) $H$ should be maximized when $p$ is uniform, or constant; (3)
$H$ should not influenced by outcomes with a null probability, and (4) $H$
should be independent on the order of information acquisition. Shannon's entropy
[Shan48]

$$H = -\sum_{n} p_n \log p_n$${eq:shannon}

is the only measure of entropy that satisfies these four properties. Beck09
discusses other information entropy measures, including the Tjallis entropy
[Tsal88], which could also be of interest in ecology according to Hart11 even
though it does not satisfy the forth abovementioned property.

MaxEnt is not a parametrization method, and does not rely on any explicit
mechanisms. Rather, the probability distribution that maximizes eq:shannon
subject to a set of $m$ constraints {$g_1, g_2, ..., g_m$} is directly derived
using the method of Lagrange multipliers:

$$\frac{\partial H}{\partial p_n} = \lambda_1 \frac{\partial g_1}{\partial p_n}
+ \lambda_2 \frac{\partial g_2}{\partial p_n}+...+ \lambda_m \frac{\partial
g_m}{\partial p_n},$${eq:lagrangegeneral}

where {$\lambda_1, \lambda_2, ..., \lambda_m$} are the Lagrange multipliers.
Solving eq:lagrangegeneral analytically or numerically gives us the probability
distribution $p_n$ of maximum entropy. This is the probability distribution that
best represents our prior knowledge on the system. However, maximum entropy
distributions are not necessarily the best fit to empirical data. Deviations
from MaxEnt distributions might suggest that the set of constraints used is not
adequate, or that mechanisms not encoded in those constraints are important
[Hart11].

I will maximize Shannon's entropy $H=-\sum_{k=1}^{S} p_k \log p_k$ subject to
two constraints:

$$g_1 = \sum_{k=1}^{S} p_k = 1,$${eq:g1}
$$g_2 = \langle k \rangle = \sum_{k=1}^{S} k p_k = \frac{2L}{S}.$${eq:g2}

The first constraint $g_1$ is a normalizing constraint, whereas the second
constraint $g_2$ is a constraint on the average of the distribution. 

\subsection{La théorie de l'entropie maximale en écologie} 

\begin{itemize}
    \item Comment le principe d'entropie maximale a été utilisé en écologie
    \item METE / ANSE
\end{itemize}

The principle of maximum entropy has been used to derive many distributions of
interest in ecology. For example, Will11 derived the least-biased degree
distribution of various ecological networks from their total number of species
$S$ and number of interactions $L$. Will11 used the mean degree $L/S$ to
constrain the average of the degree distribution. MaxEnt has also notably been
used in species distribution modelling [SDM; PhilAnde06b; PhilDudi08].

The maximum entropy theory of ecology (METE) is a comprehensive ecological
theory based on the principle of maximum entropy [Hart11]. METE derives many
distributions of interest in macroecology from a set of state variables.
Different versions of this theory can be developed according to the choice of
state variables. The ANSE version [HartZill08; HartNewm14a] relies on four
*state variables*: the area $A_0$, the total number of individuals $N_0$, the
number of species $S_0$, and the total energetic requirement of the community
$E_0$. The core functions of the ANSE version of METE are the ecosystem
structure function $R(n, \epsilon)$ and the intra-specific spatial abundance
distribution $P(n)$ [ISSAD; BrumNewm19]. These two distributions are derived
directly using MaxEnt and are constraints by ratios of the state variables (e.g.
the average species abundance and the average metabolic rate per species). The
ecosystem structure function is the probability distribution that if a species
is chosen at random from the species pool, it will have an abundance of $n$, and
if an individual among those $n$ individuals is chosen at random, it will have a
metabolic rate between $\epsilon$ and $d\epsilon$. ISSAD is the probability that
a given number of individuals of a given species are found in a given area
within $A_0$. Many metrics are derived from the core functions, including the
intra-specific energy distribution (ISED; i.e. the probability density function
that an individual from a given species has a given metabolic energy rate), the
species-abundance distribution (SAD; i.e. the probability that a given species
has a given abundance), and the species-area relationship (SAR; i.e. the
expected number of species found in a given area within $A_0$). The ANSE version
of METE has been tested against empirical data, with promising results [Hart11;
McGlXiao15; but see XiaoMcGl15], especially for less disturbed ecosystems
[NewmWilb20].

HartZill08 derived the SAD, the SAR, the ISSAD, and the ISED, that I will use
to predict food webs. More specifically, the ASNE version of METE predicts that
the species-abundance distribution (SAD) is of the form:

$$\Phi(n) = \frac{1}{ln(\lambda_1^{-1})}\frac{e^{-\lambda_1n}}{n},$${eq:SAD}

where $\Phi(n)$ is the probability that a species has an abundance of $n$, and
$\lambda_1$ is a Lagrange multiplier that satisfies the equation
$\lambda_1ln(\lambda_1^{-1}) \approx \frac{S_o}{N_o}$.

Moreover, the intra-specific spatial abundance distribution (ISSAD) is of the
form:

$$P(n) = \frac{e^{-\lambda_pn}}{\sum_{i=0}^{n_0}e^{-\lambda_pi}},$${eq:ISSAD}

where $P(n)$ is the probability that a species has $n$ individuals in an area
$A$ within $A_0$ if it has $n_0$ individuals in $A_0$, and $\lambda_p$ satisfies
the equation:

$$\frac{n_0A}{A_0}=\frac{1}{1-e^{-\lambda_p(n_0+1)}}\left(\frac{e^{-\lambda_p}}{1-e^{-\lambda_p}}-e^{-\lambda_p(n_0+1)}\left(n_0+\frac{1}{1-e^{-\lambda_p}}\right)\right)$$

which can be solved numerically.

METE also predicts that the species-area relationship (SAR) is of the form:  

$$S(A) =
S_o\sum_{n_0=1}^{N_0}\left[1-\frac{1}{\sum_{n=0}^{n_0}e^{-\lambda_pn}}\right]\frac{1}{ln(\lambda_1^{-1})}\frac{e^{-\lambda_1n_0}}{n_0},$${eq:SAR}

where $S(A)$ is the expected number of species in area $A$ within  $A_0$.

Finally, the intra-specific energy distribution (ISED) is of the form:

$$\Theta(\epsilon)\approx \lambda_2ne^{-\lambda_2n\epsilon},$${eq:ISED}

where $\Theta(\epsilon)$ is the probability density for a species to have an
energetic requirement between $\epsilon$ and $\epsilon+d\epsilon$, and
$\lambda_2 \approx \frac{S_o}{E_o}$.

\subsection{La théorie de l'entropie maximale des réseaux trophiques} 

\begin{itemize}
    \item Étendre la théorie aux réseaux trophiques
    \item On peut faire ça grâce à la variabilité locale des interactions
    \item Avantages comparativement aux autres modèles
    \item Objectifs de la thèse
\end{itemize}

The degree distributions and probabilistic food webs that will be derived in
Chapter 1 may dismiss relevant ecological information easily measurable in many
communities. The total area $A_o$ of a community is such meaningful measure.
Network-area relationships (NARs) describe the spatial scaling of many
ecological network properties [ThomTown05; WoodRuss15a; GaliLurg18a]; and
according to MacDBanv20b, many different NARs are generated from the scaling of
species richness with area and from the scaling of the number of links with
species richness. In Chapter 1, I will have shown how spatial scale impacts
worldwide predictions of food-web structure. Another community measure relevant
to network ecology is the total number of individuals $N_0$, which has been
shown to have a strong effect on the nestedness of mutualistic networks
[KrisJr08]. On a different level, pairwise species interactions could also be a
result of species abundances due to neutral or stochastic processes [Vazq05a;
VazqMeli07; CoelRang18; but see OlitFox15]. Moreover, the total metabolic
energy $E_0$ of a community (or its total biomass) could also impact food-web
structure. For instance, the body mass of two species could predict the strength
of trophic interactions [Anon05; BerlDunn09]. These measures were used in the
ASNE version of the maximum entropy theory of ecology [HartZill08;
HartNewm14a], along with species richness $S_0$, to derive macroecological
metrics of maximum entropy. METE notably derives the species-abundance
distribution (SAD), the species-area relationship (SAR), the intra-specific
spatial abundance distribution (ISSAD), and the intra-specific energy
distribution (ISED). These metrics could provide very useful constraints on the
probability and the weight of pairwise species interactions.

%%%%% Objectifs 

\section{Objectifs généraux (1 page)} 

\begin{itemize}
    \item Décrire cette variabilité et cette incertitude
    \item Trouver une façon de réduire cette incertitude ou d'en tirer profit pour réduire les lacunes dans nos connaissances
    \item Comprendre la variabilité des interactions et l'utiliser pour développer un modèle robuste des réseaux
\end{itemize}

The main objective of my thesis is to provide simple and rigorous predictive
models of food-web structure. My method aims at being as data efficient and as
unbiased as possible. In that respect, it will rely extensively on the principle
of maximum entropy [MaxEnt; Jayn57; Jayn57a], which has repeatedly proven
useful in ecology [see for example PhilAnde06b]. My predictions will however be
highly sensible to the chosen set of *state variables*, i.e. on the handful of
ecological measures fed to my models. I aim at building a particularity
parsimonious model that will rely on a single state variable (i.e. on species
richness), and a more comprehensive model that will rely on more state
variables. The latter will build on the maximum entropy theory of ecology (METE)
of HartZill08. Overall, my research will be a good methodological contribution
to network ecology by making it easier for ecologists to study food webs at
macroecological scales, especially given the scarcity of data on trophic
interactions.

Although I will exclusively analyze food webs in my thesis, my models could be
adapted to any type of ecological networks. Food webs are convenient for reasons
of data availability, ecological significance, and wide literature precedence.
Using my approach with other ecological networks (e.g. parasitism or pollination
networks), while relevant, will not be part of my thesis.

My models will provide snapshots of ecological networks, either in the *present*
or in the future. Following the arguments presented in NewmWilb20, I
hypothesize that my predictions will be more robust for stable communities, i.e.
for communities whose state variables are not changing too rapidly. In contrast,
they should be quite imprecise for more dynamic ecosystems, unless my model is
adapted accordingly (which is unfortunately beyond the scope of my research
project). However, I believe that deviations from my models should remain
ecologically informative even for ecosystems *in transition*.  

My models will be both simple and rigorous. I will make predictions of network
structure using the principle of maximum entropy, which derives least-biased
distributions that respect a set of constraints. The main objective of my thesis
will be to develop and validate these models, and to use them with both actual
and forecasted biodiversity and environmental data. Of course, my predictions
will be of first-order, and will be accurate only if no other ecological
mechanisms, other than the ones encoded in these constraints, is at play. In
that respect, deviations from my predictions will be highly informative. Some of
my predictions could also be used as informative prior distributions in Bayesian
inference studies on ecological networks.

%%%%% Chapitres

\section{Organisation de la thèse (2 pages)}

My thesis will be divided in three related chapters. In the first two chapters,
I will develop and validate my two predictive models of food-web structure. The
first model (Chapter 1) will predict food-web structure using species richness
only, whereas the second model (Chapter 2) will predict food-web structure from
species richness, total abundance, area, and energy requirement. **While the
first model will be used to predict the structure of probabilistic food webs at
very large spatial scales, the second one will be used to make predictions of
weighted food webs for more well-defined communities.** In Chapter 3, these
models will be used to forecast food-web structure using expected changes in the
values of these state variables.

\subsection{Chapitre 1: Décrypter les réseaux d'interactions probabilistes} 

\begin{itemize}
    \item Objectifs
    \item Méthodes 
    \item Principaux résultats
    \item Contribution personnelle (apport original, indépendant et spécifique)
\end{itemize}

\subsection{Chapitre 2: Un modèle d'entropie maximale prédisant la structure des réseaux trophiques} 

\begin{itemize}
    \item Objectifs
    \item Méthodes 
    \item Principaux résultats
    \item Contribution personnelle (apport original, indépendant et spécifique)
\end{itemize}

In Chapter 1, I will derive food-web measures of maximum entropy using two state
variables: the number of species and the number of links. Because the number of
links in food webs can be predicted from their number of species [MacDBanv20b],
I will actually derive food-web properties using data on species richness only.
In contrast to the model of Chapter 2, building a statistical model based on a
single state variable will allow me to make predictions at a large spatial
scale, especially given that species richness is extensively measured and
modeled worldwide [see for example GoteColw01].

This model could be used in different ways. First, it could be used to make
first order approximations of network structure worldwide, which would greatly
facilitate the conduction of spatially-explicit studies on ecological networks.
I will illustrate this potential application in more details in this chapter.
Second, it could be used as a null model to identify ecological mechanisms
leading to observed network structures, by investigating the deviations of
empirical food webs from the predictions of my model. Third, it could also be
used to generate informative prior distributions for network structure, which
could then be updated with empirical data using Bayesian inference methods;
however I will not focus on this application in my thesis project.

In this chapter, I will predict the degree distribution and the joint degree
distribution, two fundamental properties of ecological networks [PoisGrav14a;
DelmBess19], from species richness using the flexible links model
[MacDBanv20b] and the principle of maximum entropy [Will11]. The probabilistic
adjacency matrix, and thus other emerging network properties, will also be
predicted from species richness using simple topological rules on the joint
degree distribution. My model will provide first-order approximations of network
structure at a large spatial scale when used with worldwide estimations of local
species richness. Considering the robustness of METE predictions on more stable
ecological communities [NewmWilb20], I hypothesize that these approximations
will be more precise for stable networks than for networks *in transition*.
Potential discrepancies with empirical data might underline the significance of
other ecological mechanisms not encoded in my model, including dynamical
mechanisms.

Predicting the structure of an ecological network solely from its number of
species is as challenging as it is promising. Given that species richness is
perhaps the most studied measure of local biodiversity worldwide (see for
example GoteColw01), the predictive model developed in this chapter will
greatly facilitate the study of ecological networks at large spatial scales,
including macroecological studies of the functioning and dynamic of ecological
systems [PascPasc06; HeleGarc14]. My model is *very* parsimonious, and as such
can make first-order predictions of food-web structure for all ecosystems on
Earth, as long as their species richness is known. My model will also act as a
null model to identify relevant ecological mechanisms shaping ecological
networks.

In this chapter, food-web structure will be predicted from their species
richness. First, the total number of links will be predicted using the flexible
links model of MacDBanv20b, fitted with all food webs archived on
mangal.io. The degree distribution and the joint
degree distribution will next be derived using the principle of maximum entropy
with the predicted mean degree as a constraint. Probabilistic adjacency matrices
will then be predicted from the joint degree distribution using simple
topological models and simulating annealing algorithms, and various measures of
network structure will be computed from these adjacency matrices. Finally, to
illustrate potential applications of my model, network structure will be mapped
at the global scale using worldwide data on species richness. The proposed
methodology is outlined in fig:conceptual1. Note that although I will only
consider food webs in my research for simplicity purposes, my model could be
adapted and applied to any type of ecological networks.

Many ecological
networks archived on `mangal.io` [PoisBais16] are multilayer networks, i.e. networks that
describe different types of interactions. In contrast to MacDBanv20b, who
considered as food webs any networks having at least one interaction of
predation or herbivory, I will consider as food webs all ecological networks
whose interactions are *mainly* of predation and herbivory types. 

I will use the principle of maximum entropy to derive the least-biased degree
distribution that best represents my prior knowledge of a food web. 

This implies that predicting a food web's degree
distribution from its number of species generates a distribution of degree
distributions when we account for the uncertainty around the predictions of $L$.
On the other hand, in the case where the number of links is known empirically, a
single degree distribution of maximum entropy would be derived. 

I will compare empirical degree distributions of food-webs archived on
`mangal.io` to their predicted degree distributions of maximum entropy using Q-Q
plots or rank curves.

Maximum entropy in-degree and out-degree distributions can be derived in a
similar fashion as the maximum entropy degree distribution. However, although
species *must* have a degree of 1 or higher, they can have an in-degree or an
out-degree of 0. 

I will predict the adjacency matrix using two methods: an heuristic method and a
method based on a null model. The adjacency matrix will first be predicted from
the joint degree distribution using heuristic methods (e.g. simulating
annealing). Using this method, I will obtain a set of plausible networks that
have the desired joint degree distributions, and estimate their structure using
measures presented in DelmBess19. As a result, I will obtain a probability
distribution for each chosen network property and for each value of $S$.

These predictions will be compared to the ones obtained from the second method.
In that case, I will use topological null models to directly predict the
probabilistic adjacency matrix. I will for instance use Type II null models
[BascJord03b; DelmBess19], that estimate the probability of interaction
between two species $i$ and $j$ from their in and out-degrees:

$$P_{i \rightarrow j} =
\frac{1}{2}\left(\frac{k_{in}(j)}{S}+\frac{k_{out}(i)}{S}\right)
$$

Measures that will be
computed include nestedness [StanKopp13b], maximum trophic level, mean trophic
level, generality, and vulnerability [DelmBess19]. My model will be evaluated by plotting its predictions against the observed
structure of empirical webs.

Although my model could be applied to any type of ecological networks, I will
only analyze food webs in my thesis. Food webs indeed have the desirable
property to comprise most species in a biological community. Because all
heterotrophs must eat to survive, the number of nodes in a food web is very well
approximated by local species richness. Furthermore, the scaling of the number
of links with the number of species is very well documented in food webs
(MacDBanv20b). Adapting my method to other types of ecological networks is
beyond the scope of my research project.

\endinput
%%
%% End of file `01_introduction.tex'.
