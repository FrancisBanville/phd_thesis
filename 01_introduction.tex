%%
%% This is file `01_introduction.tex',
%% generated with the docstrip utility.
%%

% Utilisez la macro de langue appropriée.
\francais   
\doublespacing
%ou
%%\anglais
\chapter*{Introduction}


%%%%% Mise en contexte 

\section{Les réseaux d'interactions entre espèces en tant que systèmes complexes}

Les systèmes complexes sont omniprésents autour de nous. Considérons les
transactions effectuées sur les marchés financiers mondiaux
(\cite{Anderson2018Economy}), où une décision prise à Toronto peut avoir des
répercussions à Londres et à Bombay en quelques minutes seulement. Pensons
également au réseau de neurones dans notre cerveau (\cite{Sporns2011Human}), qui
orchestre des milliers de signaux électriques à chaque seconde pour créer notre
expérience subjective du monde. Ou encore, à une colonie de fourmis
(\cite{Bonabeau1999Swarm}), où chaque individu suit des règles simples, mais qui
ensemble génère des comportements complexes tels que la construction de nids et
la défense du territoire. Les systèmes complexes régissent plusieurs phénomènes
importants, mais souvent difficiles à étudier, dans notre quotidien et la
nature. 

Quel que soit leur objet d'étude, les systèmes complexes sont constitués de
nombreux éléments qui interagissent entre eux (\cite{Rind1999Complexity}). Ces
interactions impliquent des échanges directs d'information, de matière ou
d'énergie (\cite{Ladyman2013What}), donnant lieu à des propriétés émergentes de
niveau supérieur (\cite{Foote2007Mathematics}). Caractéristiques fondamentales
des systèmes complexes, ces propriétés émergentes (p. ex. un krach boursier, la
conscience humaine ou le comportement d'un superorganisme) ne sont pas
déductibles des éléments pris individuellement (\cite{Nielsen2000Emergent}). Le
désordre apparent des éléments et de leurs interactions génère donc une
structure ordonnée qui est souvent au cœur des recherches sur les systèmes
complexes. 

Pour faciliter l'étude des systèmes complexes, il est courant de les représenter
sous forme de réseaux, où les éléments forment des nœuds et les interactions,
des liens entre ces nœuds (\cite{Newman2003Structure}). Dans cette thèse, je me
suis intéressé plus particulièrement aux réseaux d'interactions entre espèces,
qui servent d'outil pour mieux comprendre les systèmes écologiques complexes.
Bien ancrée dans la science des réseaux et la théorie des graphes, l'écologie
des réseaux fournit un cadre d'analyse pour étudier le fonctionnement des
écosystèmes et leur résilience face aux changements environnementaux
(\cite{Proulx2005Network}, \cite{McCann2007Protecting}, \cite{McCann2011Food},
\cite{Rooney2012Integrating}, \cite{Valiente-Banuet2019Species}). Elle développe
des méthodes pour mesurer les interactions entre espèces (p. ex.
\cite{Jordano2016Sampling}) et analyser la structure émergente des réseaux
d'interactions (p. ex. \cite{Delmas2019Analysing}), nous permettant ainsi de
mieux comprendre les caractéristiques de ces systèmes complexes. 

Cette thèse jette les bases de la théorie de l'entropie maximale des réseaux
trophiques, qui constitue une approche novatrice en écologie des réseaux
permettant de prédire la structure émergente des réseaux complexes
d'interactions à partir d'un nombre limité d'informations écologiques. Avant
d'exposer cette théorie, il convient de décrire certains concepts clés des
réseaux d'interactions entre espèces afin de mieux comprendre la composition et
la représentation mathématique de ces systèmes écologiques complexes. Les sous-sections
suivantes traitent de la nature et des mesures des interactions entre espèces et
dressent un portrait de la structure émergente des réseaux d'interactions. Ces
descriptions permettront de mieux comprendre les fondements et champs d'application
de cette nouvelle théorie écologique.

\subsection{Qu'est-ce qu'une interaction entre espèces?} 

Une communauté biologique est composée d'espèces différentes interagissant entre
elles dans un espace donné (\cite{Stroud2015Community}). Par exemple, les
milieux forestiers tempérés du Québec sont riches en espèces animales et
végétales. À cause des changements climatiques et de la fragmentation des
habitats naturels, on y retrouve maintenant des tiques \textit{Ixodes
scapularis} infectées par la bactérie \textit{Borrelia burgdorferi}, responsable
de la maladie de Lyme (\cite{Ogden2009Emergence}, \cite{Simon2014Climate}). Les
tiques juvéniles se nourrissent sur des petits mammifères, dont la souris à
pattes blanches (\textit{Peromyscus leucopus}) qui est un réservoir compétent de
la bactérie (\cite{Donahue1987Reservoir}). Les tiques adultes, quant à elles, se
nourrissent et s'accouplent sur les cerfs de Virginie (\textit{Odocoileus
virginianus}, \cite{Lane1991Lyme}). Ces deux espèces (\textit{P. leucopus} et
\textit{O. virginianus}) s'alimentent à leur tour de glands de chêne
(\cite{McShea1993Variablea}, \cite{Elkinton1996Interactions},
\cite{Wolff1996Population}, \cite{McShea2000Influence}), ce qui peut expliquer
pourquoi l'abondance des tiques dépend de celle des glands
(\cite{Ostfeld2006Climate}). Comprendre l'écologie et l'épidémiologie de la
maladie de Lyme nécessite donc une compréhension approfondie des interactions
entre espèces au sein des communautés où la tique \textit{I. scapularis} est
présente. Ces interactions peuvent être représentées 
à différents niveaux d'organisation (p. ex., interactions entre individus, espèces ou clades), 
décrire une variété de processus écologiques (p. ex., interactions hôtes-parasites ou
plantes-herbivores) et couvrir différentes échelles spatiales (local ou régional).

\subsubsection{Une interaction entre individus représentée à l'échelle de l'espèce} 

Bien que les interactions se produisent biologiquement entre des
individus, elles peuvent être représentées à des niveaux d'organisation
supérieures (p. ex., au niveau de l'espèce). Les individus sont à la base des
interactions écologiques puisque ce sont eux qui effectuent les échanges
d'information, de matière ou d'énergie. Par exemple, c'est du sang d'un cerf de
Virginie particulier que s'alimentera une tique adulte. Il va sans dire que les
réseaux d'interactions entre individus sont difficiles à échantillonner, le peu
de données disponibles sur les interactions entre individus étant biaisées en
faveur d'organismes plus simples à observer (\cite{Guimaraes2020Structure}).
Pour cette raison, les interactions entre individus sont le plus souvent
groupées en des niveaux d'organisation supérieures, connectant ainsi différentes
populations, espèces ou clades entre eux (\cite{Elton1927Animal}). Les réseaux
d'interactions entre espèces, qui lient deux espèces lorsque certains de leurs
individus interagissent, ne capturent donc qu'une partie de la complexité d'un
écosystème. Ils constituent malgré tout une représentation adéquate du rôle
écologique des espèces (\cite{Delmas2019Analysing}) et des propriétés émergentes
des écosystèmes (\cite{Loreau2010Populations}, \cite{McCann2011Food},
\cite{Bascompte2013Mutualistic}, \cite{Gonzalez2020Scalingup}), tout en formant
la plupart des réseaux d'interactions échantillonnés
(\cite{Guimaraes2020Structure}). Il n'existe pas de niveau d'organisation
approprié par défaut, celui-ci dépendant de la question posée
(\cite{Niquil2020Shifting}) et des données disponibles.

J'ai testé la théorie de l'entropie maximale des réseaux trophiques sur des
réseaux d'interactions entre espèces. Cependant, elle peut être adaptée à
d'autres niveaux d'organisation taxonomique en sélectionnant les informations
écologiques (entrées du modèle) et mesures prédites (sorties du modèle)
conformément au niveau d'organisation choisi. Dans cette thèse, je propose
également une façon d'étendre la théorie pour qu'elle puisse prédire
\textit{simultanément} les propriétés des réseaux d'interactions entre individus
et espèces. À défaut d'avoir développé une telle théorie élargie, cette thèse
explore comment différents types d'interactions écologiques varient selon le
niveau d'organisation taxonomique, alimentant une réflexion sur le champ
d'application de la théorie au-delà des interactions entre espèces. 

\subsubsection{Une interaction pouvant décrire différents processus écologiques} 

Les interactions entre espèces peuvent avoir des effets positifs, négatifs ou
neutres sur les organismes impliqués, en fonction du type d'interaction. Les
interactions antagonistes, telles que les interactions hôtes-parasites,
prédateurs-proies et plantes-herbivores, profitent à une espèce au détriment de
l'autre. Les interactions hôtes-parasites procurent un habitat et une source
d'alimentation au parasite au prix du \textit{fitness} de l'hôte, alors que les
interactions prédateurs-proies et plantes-herbivores transfèrent l'énergie de la
ressource vers le consommateur, entraînant généralement la mort de la proie,
mais pas nécessairement celle de la plante. Les interactions mutualistes, telles
que les interactions plantes-pollinisateurs, profitent au contraire aux deux
espèces. Le pollinisateur se nourrit du nectar de la plante, qui en retour en
profite pour disperser son pollen à des fins de reproduction. Selon le niveau de
complexité représenté, les réseaux d'interactions entre espèces peuvent être
composés d'un seul type d'interaction (p. ex., les réseaux trophiques composés
exclusivement d'interactions trophiques prédateurs-proies et plantes-herbivores)
ou de plusieurs (p. ex., les réseaux multicouches composés d'interactions
antagonistes et mutualistes, \cite{Pilosof2017Multilayer}).

J'ai formulé les principes de base de la théorie de l'entropie maximale des
réseaux trophiques autour des interactions prédateurs-proies et
plantes-herbivores. Les réseaux trophiques ont une importance historique en
écologie, depuis Charles Sutherland Elton (1900-1991) qui a développé les
concepts de chaînes et de niveaux trophiques (\cite{Elton1927Animal},
\cite{Elton1958Ecology}) il y a près d'un siècle. Plusieurs publications
marquantes en écologie, comme celles de \cite{May1972Will} sur la stabilité des
écosystèmes complexes et de \cite{Lotka1925Elements} et
\cite{Volterra1927Fluctuations} sur la dynamique des populations animales,
portent sur les interactions trophiques. Les interactions prédateurs-proies et
plantes-herbivores déterminent les flux de matière et d'énergie au sein des
écosystèmes. Elles nous permettent de mieux comprendre des phénomènes tels que
la régulation de la productivité primaire par cascade trophique
(\cite{Carpenter1987Regulation}), la stabilité des écosystèmes face à
l'extirpation d'espèces (\cite{Dunne2002Network}, \cite{Srinivasan2007Response},
\cite{Staniczenko2010Structural}) et la capacité des espèces à migrer et
s'adapter aux changements climatiques (\cite{Tylianakis2008Global},
\cite{Gilman2010Framework}) et à la perte d'habitats naturels
(\cite{Evans2013Robustness}). De plus, les données disponibles sur les réseaux
trophiques couvrent une diversité d'habitats à l'échelle mondiale
(\cite{Poisot2021Global}), ce qui facilite la validation des prédictions des
modèles de réseaux. Néanmoins, la théorie de l'entropie maximale des réseaux
trophiques peut être adaptée et testée sur d'autres types d'interactions qui
impliquent un échange direct d'information, de matière ou d'énergie (comme les
interactions hôtes-parasites ou plantes-pollinisateurs).

\subsubsection{Une interaction réalisée localement} 

Cette thèse distingue les interactions réalisées à l'échelle locale de celles
mesurées à l'échelle régionale. Les réseaux locaux
(\cite{Poisot2012Dissimilarity}) sont constitués d'interactions observées ou
réalisées à un endroit et moment particuliers. Deux espèces interagissent
localement si leurs abondances (nombres d'individus) sont suffisamment élevées
pour qu'elles puissent se rencontrer (\cite{Canard2012Emergence},
\cite{Canard2014Empirical}) et que leurs traits (caractéristiques biologiques
telles que la taille corporelle) supportent une interaction lorsqu'elles se
rencontrent (\cite{Bolnick2011Why}, \cite{Gravel2013Inferring},
\cite{Stouffer2011Role}). Par exemple, les habitudes alimentaires du cerf de
Virginie varient selon la disponibilité saisonnière en nourriture
(\cite{Short1975Nutrition}) et sa masse corporelle (\cite{Luna2013Influence}).
Les conditions environnementales, comme la température
(\cite{Angilletta2004Temperature}), les précipitations
(\cite{Woodward2012Climate}) et l'utilisation du territoire
(\cite{Tylianakis2007Habitat}), influent également sur la réalisation des
interactions à l'échelle locale. Ces trois facteurs (abondances, traits et
conditions environnementales) contribuent à la variation spatiale et temporelle
des interactions locales (\cite{Poisot2015Species}). 

Les réseaux régionaux (ou \textit{metaweb}, \cite{Pascual2006Ecological}), quant
à eux, sont constitués d'interactions potentielles, c'est-à-dire d'interactions
biologiquement ou écologiquement faisables sans nécessairement être réalisées
localement. Par exemple, les interactions trophiques régionales du cerf de
Virginie incluent l'ensemble des plantes et champignons dont il peut se nourrir
dans différents contextes. Une interaction régionale indique si deux espèces
possèdent des traits leur permettant d'interagir dans des conditions idéales.
Les réseaux régionaux sont donc un catalogue de l'ensemble des interactions
observées (p. ex., \cite{Maiorano2020Tetraeu}) ou prédites (p. ex.,
\cite{Strydom2022Food}) entre un groupe d'espèces données. Les réseaux locaux,
étant constitués d'un sous-ensemble des espèces et interactions régionales, sont
des sous-réseaux du \textit{metaweb} (\cite{Saravia2022Ecological}).
Contrairement aux réseaux locaux, les réseaux régionaux ne varient pas
spatialement ni temporellement puisqu'ils incluent l'ensemble des interactions
locales. 

Cette distinction entre réseaux locaux et régionaux revêt d'une grande
importance dans l'élaboration de la théorie de l'entropie maximale des réseaux
trophiques, celle-ci étant mieux adaptée aux réseaux locaux. En effet, cette
théorie trouve son fondement dans la complexité des écosystèmes locaux. Les
écosystèmes complexes sont constitués, par définition, d'interactions impliquant
un échange direct d'information, de matière ou d'énergie
(\cite{Ladyman2013What}), ce qui se produit à l'échelle locale. Les réseaux
régionaux décrivent plutôt le potentiel que de tels échanges aient lieu, sans
nécessairement être réalisés. De plus, la variabilité des interactions locales
introduit une incertitude qui est non réductible (c.-à-d. qui ne diminue pas
avec davantage de données). Cela a pour conséquence de diminuer la prédictibilité des
interactions locales, rendant possiblement les réseaux locaux plus complexes. La
théorie de l'entropie maximale des réseaux trophiques vise précisément à
identifier la structure la plus complexe possible des réseaux locaux, dans la
limite de contraintes écologiques. Un chapitre complet de cette thèse est dédié
aux caractéristiques et sources d'incertitude des réseaux locaux et régionaux.

\subsection{Comment mesurer une interaction entre espèces?} 

Construire un réseau demande de faire des choix méthodologiques concernant les
interactions entre espèces, choix qui impacteront notre interprétation des
propriétés émergentes du réseau. Étant donné le nombre élevé d'organismes et la
diversité de leurs interactions, il est souvent nécessaire d'opter pour une
représentation simplifiée des interactions au sein d'une communauté biologique.
Cette représentation simplifiée doit contenir les informations écologiques
pertinentes au sujet d'étude, dans les limites de ce qui peut être observé. 
Ces décisions dépendent autant des objectifs de recherche que des données
disponibles.

\subsubsection{Évaluer la présence ou la valeur d'une interaction} 

\subsubsection{Évaluer l'incertitude d'une interaction} 


\subsection{Quelle est la structure émergente d'un réseau d'interactions entre espèces?} 

\subsubsection{Une structure représentée par des valeurs uniques} 

\subsubsection{Une structure représentée par des distributions} 


%%%%% La théorie de l'entropie maximale des réseaux trophiques

\section{Vers une théorie de l'entropie maximale des réseaux trophiques}

\subsection{Pourquoi avons-nous besoin d'une théorie des réseaux trophiques?} 

\subsection{Quels sont les fondements de la théorie?} 

\subsection{Que permet-elle de prédire?} 

\subsection{Quel est son champ d'application?} 


%%%%% Objectifs 

\section{Objectifs généraux de la thèse} 


%%%%% Chapitres

\section{Organisation de la thèse}

\subsection{Chapitre 1: Décrypter les réseaux d'interactions probabilistes} 

\subsection{Chapitre 2: Des modèles d'entropie maximale prédisant la structure des réseaux trophiques} 

\endinput
%%
%% End of file `01_introduction.tex'.
