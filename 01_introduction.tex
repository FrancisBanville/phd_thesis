%%
%% This is file `01_introduction.tex',
%% generated with the docstrip utility.
%%

% Utilisez la macro de langue appropriée.
\francais   
\doublespacing
%ou
%%\anglais
\chapter*{Introduction}


%%%%% Mise en contexte 

\section{L'écologie des réseaux pour mieux comprendre les écosystèmes complexes}

\subsection{Pourquoi étudier les interactions écologiques} 

\begin{itemize}
    \item Différents types d'interactions (trophique, parasitisme, pollinisation)
    \item Écologie des interactions (flux d'énergie, épidémie, sécurité alimentaire)
    \item Écologie des réseaux (stabilité des écosystèmes, résilience, fonctions)
\end{itemize}
  

\subsection{Différents types de réseaux d'interactions écologiques} 

\begin{itemize}
    \item Étudier les réseaux demande de faire des choix 
    \item Ces choix dépendent de ce qu'on veut étudier et de notre capacité d'observation
    \item Différents processus écologiques seront pris en compte selon le type de réseau
\end{itemize}

\subsubsection{Réseaux d'individus ou d'espèces: une question d'organisation} 

\begin{itemize}
    \item Les individus sont à la base des interactions
    \item On peut vouloir grouper les individus en populations ou espèces
    \item On peut aussi les grouper à des niveaux d'organisation supérieurs (p.ex. espèces trophiques)
\end{itemize}

\subsubsection{Réseaux binaires ou quantitatifs: une question de mesure} 

\begin{itemize}
    \item Les réseaux binaires nous disent si l'interaction a lieu
    \item Les réseaux quantitatifs peuvent vouloir dire différentes choses (flux d'énergie, impact d'une espèce sur une autre)
\end{itemize}

\subsubsection{Réseaux locaux ou régionaux: une question d'échelle spatiale} 

\begin{itemize}
    \item Les réseaux locaux nous disent si l'interaction est réalisé 
    \item Les réseaux régionaux nous disent si l'interaction a un potentiel d'être réalisé
\end{itemize}

\subsection{Structure des réseaux d'interactions entre espèces} 

\begin{itemize}
    \item Il existe plusieurs mesures sur les réseaux 
\end{itemize}

\subsubsection{Propriétés des espèces} 

\begin{itemize}
    \item Peuvent nous informer du rôle des espèces au sein des réseaux (distribution de degrés)
\end{itemize}

\subsubsection{Propriétés émergentes} 

\begin{itemize}
    \item Peuvent nous informer de la stabilité et résilience des écosystèmes (modularité, connectance)
    \item Peuvent nous informer du fonctionnement des écosystèmes 
    \item Complexité 
\end{itemize}



%%%%% Stochasticité des interactions 

\section{Incertitude et variabilité des interactions entre espèces}

\begin{itemize}
    \item Le défi principal en écologie des réseaux est le manque de données
\end{itemize}

\subsection{Incertitude des interactions: La peur de manquer quelque chose d'important} 

\begin{itemize}
    \item On ne peut pas échantillonner toutes les interactions (faux positifs et négatifs)
    \item On n'est pas certain des conditions permettant aux interactions d'être réalisés
\end{itemize}

\subsection{Variabilité des interactions: Quand le lion s'endort} 

\begin{itemize}
    \item Deux espèces qui co-occurrent ne vont pas nécessairement interagir
    \item Conditions pour qu'une interaction locale ait lieu 
    \item Exemples de variation dans le temps et l'espace due aux conditions environmentales 
\end{itemize}

\subsection{Représentation probabiliste des interactions entre espèces} 

\begin{itemize}
    \item Nous pouvons représenter la variabilité et l'incertitude de manière probabiliste
    \item Par contre, cette probabilité est difficile à interpréter 
    \item C'est important de bien connaître la source d'incertitude puisque ça nous permet de savoir comment réduire cette incertitude
    \item C'est important de bien connaître la source d'incertitude puisque ça nous informe sur les processus écologiques 
\end{itemize}



%%%%% Modèles prédictifs 

\section{De la nécessité de développer des modèles prédictifs} 

\begin{itemize}
    \item Nous avons besoin de modèles prédictifs pour combler le manque de données 
    \item Ces modèles prédictifs doivent être explicites sur l'incertitude des interactions
    \item Il existe plusieurs modèles pour prédire les interactions et les réseaux, mais ces modèles peuvent être complexes
\end{itemize}


%%%%% Trophique-METE

\section{Vers une théorie de l'entropie maximale des réseaux trophiques}

\subsection{Le principe d'entropie maximale} 

\begin{itemize}
    \item Ce que le principe d'entropie maximale nous permet de connaître
    \item Pourquoi l'utiliser
    \item Comment ça fonctionne, grossièrement
\end{itemize}

\subsection{La théorie de l'entropie maximale en écologie} 

\begin{itemize}
    \item Comment le principe d'entropie maximale a été utilisé en écologie
    \item METE / ANSE
\end{itemize}


\subsection{La théorie de l'entropie maximale des réseaux trophiques} 

\begin{itemize}
    \item Étendre la théorie aux réseaux trophiques
    \item On peut faire ça grâce à la variabilité locale des interactions
    \item Avantages comparativement aux autres modèles
    \item Objectifs de la thèse
\end{itemize}


%%%%% Objectifs 

\section{Objectifs généraux} 

\begin{itemize}
    \item Décrire cette variabilité et cette incertitude
    \item Trouver une façon de réduire cette incertitude ou d'en tirer profit pour réduire les lacunes dans nos connaissances
    \item Comprendre la variabilité des interactions et l'utiliser pour développer un modèle robuste des réseaux
\end{itemize}


%%%%% Chapitres

\section{Organisation de la thèse}

\subsection{Chapitre 1: Décrypter les réseaux d'interactions probabilistes} 

\begin{itemize}
    \item Objectifs
    \item Méthodes 
    \item Principaux résultats
    \item Contribution personnelle (apport original, indépendant et spécifique)
\end{itemize}

\subsection{Chapitre 2: Un modèle d'entropie maximale prédisant la structure des réseaux trophiques} 

\begin{itemize}
    \item Objectifs
    \item Méthodes 
    \item Principaux résultats
    \item Contribution personnelle (apport original, indépendant et spécifique)
\end{itemize}

\endinput
%%
%% End of file `01_introduction.tex'.