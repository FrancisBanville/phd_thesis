%%
%% This is file `01_introduction.tex',
%% generated with the docstrip utility.
%%

% Utilisez la macro de langue appropriée.
\francais   
\doublespacing
%ou
%%\anglais
\chapter*{Introduction}


%%%%% Mise en contexte 

\section{Les réseaux d'interactions entre espèces en tant que systèmes complexes}

Les systèmes complexes façonnent le monde qui nous entoure. Considérons par
exemple les transactions effectuées sur les marchés financiers mondiaux
(\cite{Anderson2018Economy}), où une décision prise à Toronto peut avoir des
répercussions à Londres et à Bombay en quelques minutes seulement. Pensons
également au réseau de neurones dans notre cerveau (\cite{Sporns2011Human}), qui
orchestre des milliers de signaux électriques à chaque seconde pour créer notre
expérience subjective du monde. Ou encore, à une colonie de fourmis
(\cite{Bonabeau1999Swarm}), où chaque individu suit des règles simples, mais qui
ensemble génère des comportements complexes tels que la construction de nids et
la défense du territoire. Les systèmes complexes régissent plusieurs phénomènes
importants, mais souvent difficiles à étudier de par leur faible prédictibilité,
dans notre quotidien et la nature. 

Quel que soit leur objet d'étude, les systèmes complexes sont constitués de
nombreux éléments qui interagissent entre eux (\cite{Rind1999Complexity}). Ces
interactions impliquent des échanges directs d'information, de matière ou
d'énergie (\cite{Ladyman2013What}), donnant lieu à des propriétés émergentes de
niveau supérieur (\cite{Foote2007Mathematics}). Caractéristiques fondamentales
des systèmes complexes, ces propriétés émergentes (p. ex. un krach boursier, la
conscience humaine ou le comportement d'un superorganisme) ne sont pas
déductibles des éléments pris individuellement (\cite{Nielsen2000Emergent}). Le
désordre apparent des éléments et de leurs interactions génère donc une
structure ordonnée qui est souvent au cœur des recherches sur les systèmes
complexes. 

Pour faciliter l'étude des systèmes complexes, il est courant de les représenter
sous forme de réseaux, où les éléments forment des nœuds et les interactions,
des liens entre ces nœuds (\cite{Newman2003Structure}). Dans cette thèse, je me
suis intéressé plus particulièrement aux réseaux d'interactions entre espèces,
qui servent d'outil pour mieux comprendre les systèmes écologiques complexes.
Bien ancrée dans la science des réseaux et la théorie des graphes, l'écologie
des réseaux fournit un cadre d'analyse pour étudier le fonctionnement des
écosystèmes et leur résilience face aux changements environnementaux
(\cite{Proulx2005Network}, \cite{McCann2007Protecting}, \cite{McCann2011Food},
\cite{Rooney2012Integrating}, \cite{Valiente-Banuet2019Species}). Elle y
parvient en développant des méthodes pour mesurer les interactions entre espèces
(p. ex. \cite{Jordano2016Sampling}) et analyser la structure émergente des
réseaux d'interactions (p. ex. \cite{Delmas2019Analysing}), nous permettant
ainsi de mieux comprendre les caractéristiques de ces systèmes complexes. 

Cette thèse jette les bases de la théorie de l'entropie maximale des réseaux
trophiques, qui constitue une approche novatrice en écologie des réseaux
permettant de prédire la structure émergente des réseaux complexes
d'interactions à partir d'un nombre limité d'informations écologiques. Avant
d'exposer cette théorie, il convient de décrire certains concepts clés des
réseaux d'interactions entre espèces afin de mieux comprendre la composition et
la représentation mathématique de ces systèmes écologiques complexes. Les sous-sections
suivantes traitent de la nature et des mesures des interactions entre espèces et
dressent un portrait de la structure émergente des réseaux d'interactions. Ces
descriptions permettront de mieux comprendre les fondements et champs d'application
de cette nouvelle théorie écologique.

\subsection{Qu'est-ce qu'une interaction entre espèces?} 

Une communauté biologique est composée d'espèces différentes interagissant entre
elles dans un espace donné (\cite{Stroud2015Community}). Par exemple, les
milieux forestiers tempérés du Québec sont riches en espèces animales et
végétales. À cause des changements climatiques et de la fragmentation des
habitats naturels, on y retrouve maintenant des tiques \textit{Ixodes
scapularis} infectées par la bactérie \textit{Borrelia burgdorferi}, responsable
de la maladie de Lyme (\cite{Ogden2009Emergence}, \cite{Simon2014Climate}). Les
tiques juvéniles se nourrissent sur des petits mammifères, dont la souris à
pattes blanches (\textit{Peromyscus leucopus}) qui est un réservoir compétent de
la bactérie (\cite{Donahue1987Reservoir}). Les tiques adultes, quant à elles, se
nourrissent et s'accouplent sur les cerfs de Virginie (\textit{Odocoileus
virginianus}, \cite{Lane1991Lyme}). Ces deux espèces (\textit{P. leucopus} et
\textit{O. virginianus}) s'alimentent à leur tour de glands de chêne
(\cite{McShea1993Variablea}, \cite{Elkinton1996Interactions},
\cite{Wolff1996Population}, \cite{McShea2000Influence}), ce qui peut expliquer
pourquoi l'abondance des tiques dépend fréquemment de celle des glands
(\cite{Ostfeld2006Climate}). Comprendre l'écologie et l'épidémiologie de la
maladie de Lyme nécessite une compréhension approfondie des interactions entre
espèces au sein des communautés où la tique \textit{I. scapularis} est présente.
De manière plus générale, les interactions écologiques peuvent être représentées
à différents niveaux d'organisation (p. ex. interactions entre individus ou
espèces), décrire une variété de processus écologiques (p. ex. interactions
hôtes-parasites ou plantes-herbivores) et couvrir différentes échelles spatiales
(local ou régional).

\subsubsection{Une interaction entre individus représentée à l'échelle de l'espèce} 

Bien que les interactions se produisent biologiquement entre des individus,
elles peuvent être représentées à des niveaux d'organisation supérieures (p. ex.
au niveau de l'espèce). Les individus sont à la base des interactions
écologiques puisque ce sont eux qui effectuent les échanges d'information, de
matière ou d'énergie. Par exemple, c'est du sang d'un cerf de Virginie
particulier que s'alimentera une tique adulte. Il va sans dire que les réseaux
d'interactions entre individus sont difficiles à échantillonner, le peu de
données disponibles sur les interactions entre individus étant biaisées en
faveur d'organismes plus simples à observer (\cite{Guimaraes2020Structure}).
Pour cette raison, les interactions entre individus sont le plus souvent
groupées en des niveaux d'organisation supérieures, connectant ainsi différentes
populations, espèces ou clades entre eux (\cite{Elton1927Animal}). Les réseaux
d'interactions entre espèces, qui lient deux espèces lorsque certains de leurs
individus interagissent, ne capturent donc qu'une partie de la complexité d'un
écosystème. Ils constituent malgré tout une représentation adéquate du rôle
écologique des espèces (\cite{Delmas2019Analysing}) et des propriétés émergentes
des écosystèmes (\cite{Loreau2010Populations}, \cite{McCann2011Food},
\cite{Bascompte2013Mutualistic}, \cite{Gonzalez2020Scalingup}), tout en formant
la plupart des réseaux d'interactions échantillonnés
(\cite{Guimaraes2020Structure}). Cependant, il n'existe pas de niveau
d'organisation approprié par défaut, celui-ci dépendant de la question posée
(\cite{Niquil2020Shifting}) et des données disponibles.

\subsubsection{Une interaction pouvant décrire différents processus écologiques} 

Les interactions entre espèces peuvent avoir des effets positifs, négatifs ou
neutres sur les organismes impliqués, en fonction du type d'interaction. Les
interactions antagonistes, telles que les interactions hôtes-parasites,
prédateurs-proies et plantes-herbivores, profitent à une espèce au détriment de
l'autre. Les interactions hôtes-parasites procurent un habitat et une source
d'alimentation au parasite au prix du \textit{fitness} de l'hôte, alors que les
interactions prédateurs-proies et plantes-herbivores transfèrent l'énergie de la
ressource vers le consommateur, entraînant généralement la mort de la proie,
mais pas nécessairement celle de la plante. Les interactions mutualistes, telles
que les interactions plantes-pollinisateurs, profitent au contraire aux deux
espèces. Le pollinisateur se nourrit du nectar de la plante, qui en retour en
profite pour disperser son pollen à des fins de reproduction. Selon le niveau de
complexité représenté, les réseaux d'interactions entre espèces peuvent être
composés d'un seul type d'interaction (p. ex. les réseaux trophiques composés
exclusivement d'interactions trophiques prédateurs-proies et plantes-herbivores)
ou de plusieurs (p. ex. les réseaux multicouches composés d'interactions
antagonistes et mutualistes, \cite{Pilosof2017Multilayer}).

Les interactions prédateurs-proies et plantes-herbivores ont une importance
particulière et historique en écologie. Les réseaux trophiques ont été
introduits il y a près d'un siècle, lorsque Charles Sutherland Elton (1900-1991)
a développé les concepts de chaînes et de niveaux trophiques
(\cite{Elton1927Animal}, \cite{Elton1958Ecology}). Plusieurs publications
marquantes en écologie, comme celles de \cite{May1972Will} sur la stabilité des
écosystèmes complexes et de \cite{Lotka1925Elements} et
\cite{Volterra1927Fluctuations} sur la dynamique des populations animales,
portent sur les interactions trophiques. Les interactions prédateurs-proies et
plantes-herbivores déterminent les flux de matière et d'énergie au sein des
écosystèmes. Elles nous permettent de mieux comprendre des phénomènes tels que
la régulation de la productivité primaire par cascade trophique
(\cite{Carpenter1987Regulation}), la stabilité des écosystèmes face à
l'extirpation d'espèces (\cite{Dunne2002Network}, \cite{Srinivasan2007Response},
\cite{Staniczenko2010Structural}) et la capacité des espèces à migrer et
s'adapter aux changements climatiques (\cite{Tylianakis2008Global},
\cite{Gilman2010Framework}) et à la perte d'habitats naturels
(\cite{Evans2013Robustness}). De plus, les données disponibles sur les réseaux
trophiques couvrent une diversité d'habitats à l'échelle mondiale
(\cite{Poisot2021Global}), ce qui nous permet de tester une variété de modèles
statistiques et d'hypothèses écologiques sur ces interactions. C'est pourquoi
les interactions trophiques occuperont une place centrale dans cette thèse.

\subsubsection{Une interaction réalisée localement mais cataloguée régionalement} 

Cette thèse distingue les interactions réalisées à l'échelle locale de celles
cataloguées à l'échelle régionale. Les réseaux locaux
(\cite{Poisot2012Dissimilarity}) sont constitués d'interactions observées ou
réalisées à un endroit et moment particuliers. Ce sont eux qui décrivent les
échanges directs d'information, de matière ou d'énergie ayant lieu à l'échelle
locale. Deux espèces interagissent localement si leurs abondances (nombres
d'individus) sont suffisamment élevées pour qu'elles puissent se rencontrer
(\cite{Canard2012Emergence}, \cite{Canard2014Empirical}) et que leurs traits
(caractéristiques biologiques telles que la taille corporelle) supportent une
interaction lorsqu'elles se rencontrent (\cite{Bolnick2011Why},
\cite{Gravel2013Inferring}, \cite{Stouffer2011Role}). Par exemple, les habitudes
alimentaires du cerf de Virginie varient localement selon la disponibilité
saisonnière en nourriture (\cite{Short1975Nutrition}) et sa masse corporelle
(\cite{Luna2013Influence}). Les conditions environnementales, comme la
température (\cite{Angilletta2004Temperature}), les précipitations
(\cite{Woodward2012Climate}) et l'utilisation du territoire
(\cite{Tylianakis2007Habitat}), influent également sur la réalisation des
interactions à l'échelle locale. Ces trois facteurs (abondances, traits et
conditions environnementales) contribuent à la variation spatiale et temporelle
des interactions locales (\cite{Poisot2015Species}). 

Les réseaux régionaux (ou \textit{metaweb}, \cite{Pascual2006Ecological}), quant
à eux, sont constitués d'interactions potentielles, c'est-à-dire d'interactions
biologiquement ou écologiquement faisables sans nécessairement être réalisées
localement (\cite{Tylianakis2017Ecological}). Par exemple, les interactions
trophiques régionales du cerf de Virginie incluent l'ensemble des plantes et
champignons dont il peut se nourrir dans différents contextes. Une interaction
régionale indique si deux espèces possèdent des traits leur permettant
d'interagir dans des conditions idéales. Les réseaux régionaux sont donc un
catalogue de l'ensemble des interactions observées (p. ex.
\cite{Maiorano2020Tetraeu}) ou prédites (p. ex. \cite{Strydom2022Food}) entre un
groupe d'espèces données. Les réseaux locaux, étant constitués d'un
sous-ensemble des espèces et de leurs interactions régionales, sont des
sous-réseaux du \textit{metaweb} (\cite{Saravia2022Ecological}). Contrairement
aux réseaux locaux, les réseaux régionaux ne varient pas spatialement ni
temporellement puisqu'ils incluent l'ensemble des interactions locales entre un
groupe d'espèces. 

\subsection{Comment mesurer une interaction entre espèces?} 

Outre le type d'interaction, la manière dont nous mesurons les interactions
entre espèces affecte également notre représentation des réseaux et notre
interprétation de leurs propriétés. Les mesures des interactions varient selon
leur niveau d'information écologique. Alors que les interactions binaires et
quantitatives évaluent respectivement la présence et le poids d'une interaction,
les interactions probabilistes mesurent son incertitude. Le choix d'une mesure
d'interaction, qui dépend des questions posées et de la qualité des données
disponibles, détermine les aspects de la complexité écologique représentés dans
un réseau. Les propriétés des réseaux sont généralement calculées à partir d'une
matrice d'adjacence (représentation matricielle d'un réseau) contenant la valeur
(binaire, quantitative ou probabiliste) des interactions entre chaque paire
d'espèces (\cite{Delmas2019Analysing}).

\subsubsection{Évaluer la présence ou le poids d'une interaction} 

Les interactions binaires indiquent si un lien est présent ou non dans un
réseau. La présence d'une interaction locale peut signifier que deux espèces ont
été observées en train d'interagir à l'intérieur des frontières du réseau. Par
exemple, observer une tique \textit{I. scapularis} sur le pelage d'un cerf de
Virginie \textit{O. virginianus} dans une forêt donnée témoigne d'une
interaction hôte-parasite entre ces deux espèces. Ne pas détecter cette
interaction suggère qu'elle ne se produit pas à cet emplacement particulier.
Cependant, une absence d'observation peut également être un faux négatif
(\cite{Bluthgen2010Why}, \cite{Chacoff2012Evaluating}, \cite{Stock2017Linear}),
c'est-à-dire une interaction non observée qui est en réalité réalisée. Étant
donné la difficulté d'échantillonner l'ensemble des interactions entre espèces
dans un lieu donné (\cite{Jordano2016Sampling}), ces faux négatifs sont
fréquents dans les jeux de données d'interactions entre espèces. Les méthodes
prédictives des interactions entre espèces (\cite{Strydom2021Roadmapa}) pallient
cette difficulté d'observation en prédisant lesquelles sont réalisées à
l'échelle locale. Il est cependant nécessaire de tester ces méthodes sur des
données empiriques fiables de présence-absence pour pouvoir utiliser les réseaux
prédits en toute confiance (\cite{Brimacombe2024Applying}). Bien qu'elles
contiennent peu d'information écologique comparativement aux autres mesures
d'interaction, les interactions binaires ont été plus fréquemment étudiées en
écologie (\cite{Pascual2006Ecological}, \cite{Delmas2019Analysing}), notamment
en raison de leur plus grande facilité d'échantillonnage
(\cite{Jordano2016Sampling}) et de prédiction (\cite{Strydom2021Roadmapa}).

Les interactions quantitatives mesurent quant à elles le poids d'une interaction
(\cite{Berlow2004Interaction}). Ce poids peut représenter le flux d'énergie ou
de biomasse (p. ex. \cite{Benke2001Food}, \cite{Post2002Long},
\cite{Bersier2002Quantitative}, \cite{Borrett2019Walk}), l'impact démographique
(p. ex. \cite{Paine1992Foodweb}, \cite{Kokkoris2002Variability},
\cite{Emmerson2004Predatorprey}) ou la fréquence d'interaction (p. ex.
\cite{Herrera1989Pollinator}, \cite{Montoya2003Food}) entre deux espèces. Par
exemple, le taux de cerfs de Virginie infectés par \textit{I. scapularis} est
une mesure du poids d'une interaction locale. Quel que soit l'effet mesuré, le
poids d'une interaction est souvent difficile à évaluer, surtout lorsque cet
effet est variable dans le temps ou qu'il suit une fonction non linéaire ou
densité-dépendante (\cite{Wootton2005Measurement}). De plus, puisque les
interactions binaires peuvent être déduites des interactions quantitatives, il
est nettement plus ardu d'échantillonner et de prédire ces dernières. Toutefois,
lorsqu'ils sont bien mesurés, les réseaux d'interactions quantitatives décrivant
les flux d'énergie ou de biomasse entre espèces semblent être une bonne
représentation des échanges directs se produisant au sein des systèmes
écologiques complexes. 

\subsubsection{Évaluer l'incertitude d'une interaction} 

Que nous ayons évalué la présence ou le poids d'une interaction, il est
important de quantifier et de bien comprendre l'incertitude de nos estimations
(c.-à-d. notre niveau de croyance quant à la présence ou valeur d'une
interaction). En raison des défis que posent l'échantillonnage des interactions
entre espèces (\cite{Jordano2016Sampling}), celles-ci sont souvent incertaines,
surtout quand elles n'ont pas été directement observées ou qu'elles ont été
prédites à l'aide d'une méthode numérique. Cette incertitude peut avoir
plusieurs sources, incluant les erreurs de mesure des données servant à prédire
une interaction (p. ex. données de taille corporelle,
\cite{Gravel2013Inferring}), l'intervalle de confiance des paramètres du modèle
et l'existence de modèles prédictifs alternatifs (\cite{Simmonds2022Insights},
\cite{Simmonds2024Recommendations}). La variabilité des interactions locales
dans le temps et l'espace (\cite{Poisot2015Species}) introduit une source
additionnelle d'incertitude en limitant le transfert d'information d'un réseau à
un autre, par exemple lors de l'inférence d'interactions locales à partir d'un
réseau régional (\cite{Dansereau2023Spatially}). Documenter l'incertitude des
interactions entre espèces offre une vision plus juste de la fiabilité des
données d'interactions. 

Les interactions probabilistes mesurent la probabilité qu'une interaction soit
présente (\cite{Poisot2016Structure}). À l'échelle locale, elles représentent la
probabilité qu'une interaction soit réalisée à un endroit et moment
particuliers. Par exemple, si nous n'avons pas observé d'interaction entre un
cerf de Virginie et \textit{I. scapularis} dans une forêt donnée, nous pouvons
évaluer la probabilité que ces deux espèces interagissent localement à partir de
leurs abondances relatives (\cite{Canard2014Empirical}) ou de nos connaissances
sur le cycle de vie du parasite. À l'échelle régionale, les interactions
probabilistes représentent plutôt la probabilité qu'une interaction soit
biologiquement ou écologiquement faisable (\cite{Strydom2023Grapha}). Cette
probabilité d'interaction régionale est particulièrement utile lorsque deux
espèces n'ont jamais été observées en train d'interagir, mais qu'elles
pourraient possiblement le faire dans les bonnes circonstances. Qu'elles soient
locales ou régionales, les interactions probabilistes sont généralement évaluées
de manière indépendante. Cela peut introduire des biais lorsque nous propageons
l'incertitude des interactions locales vers la structure du réseau
(\cite{Poisot2016Structure}), à cause de l'impact que les interactions peuvent
exercer les unes sur les autres au sein d'un réseau local
(\cite{Golubski2011Modifying}, \cite{Ims2013Indirect}).


\subsection{Quelle est la structure émergente d'un réseau d'interactions entre espèces?} 

Les propriétés émergentes d'un système écologique complexe peuvent être évaluées
en mesurant la structure de son réseau d'interactions, c'est-à-dire en calculant
les propriétés de la matrice d'adjacence. Plusieurs mesures ont été proposées
pour analyser la structure des réseaux d'interactions binaires
(\cite{Delmas2019Analysing}), quantitatives (\cite{Bersier2002Quantitative}) et
probabilistes (\cite{Poisot2016Structure}). Ces mesures peuvent décrire des
valeurs uniques (p. ex. nombre d'espèces, nombre d'interactions, connectance,
modularité) et des distributions des caractéristiques des espèces (p. ex.
distribution de degrés) ou des interactions (p. ex. distribution des poids
d'interaction). La structure des réseaux est associée à la dynamique
(\cite{Pascual2006Ecologicala}) et au fonctionnement des écosystèmes, qui joue
un rôle crucial dans les flux d'énergie, le cycle des nutriments et la
régulation des populations (\cite{McCann2011Food}, \cite{Thompson2012Food}).
Ci-dessous sont décrites plusieurs mesures centrales à cette thèse,
caractérisant la structure des réseaux d'interactions binaires entre espèces.

\subsubsection{Une structure représentée par des valeurs uniques} 

Le nombre d'espèces (nœuds) et le nombre d'interactions (liens) sont de bons
premiers descripteurs de la structure d'un réseau, décrivant respectivement son
ordre et sa taille (\cite{Delmas2019Analysing}). Un réseau contenant plus
d'espèces réalise un plus grand nombre d'interactions, ces deux mesures étant
liées par une loi de puissance (\cite{Brose2004Unified},
\cite{Riede2010Chapter}) ou des principes écologiques contraignants (comme
l'obligation pour les espèces de se nourrir, \cite{MacDonald2020Revisiting}). Le
modèle des liens flexibles développé par \cite{MacDonald2020Revisiting} prédit
de manière réaliste le nombre d'interactions dans les réseaux trophiques à
partir du nombre d'espèces, ce qui constitue une bonne première étape vers la
reconstitution des réseaux trophiques à partir de la richesse en espèces lorsque
les données d'interactions sont rares (\cite{Strydom2021Roadmapa}). 

Ces deux mesures déterminent la connectance d'un réseau, soit la proportion des
interactions possibles qui sont réalisées. La connectance est un descripteur clé
de la structure des réseaux d'interactions entre espèces
(\cite{Martinez1992Constant}), une connectance élevée indiquant que les espèces
sont davantage connectées. Elle a été associée à plusieurs propriétés émergentes
des écosystèmes, notamment à l'invasibilité des habitats par des espèces
envahissantes, qui ont plus de difficulté à intégrer un réseau fortement
connecté (\cite{Smith-Ramesh2017Global}). La stabilité des réseaux trophiques
face à l'extirpation d'espèces augmente également avec la connectance puisqu'un
prédateur peut plus facilement s'adapter en cherchant d'autres sources de
nourriture après la disparition d'une de ses proies (\cite{Dunne2002Network}).
Toutefois, on observe la relation inverse dans les réseaux mutualistes lorsque
l'on tient compte de la dépendance mutuelle des espèces et de l'importance
relative de chaque interaction (\cite{Vieira2015Simple}). La connectance est un
des indicateurs de la complexité des réseaux d'interactions entre espèces
(\cite{Landi2018Complexity}, \cite{Strydom2021Svd}) et covarie avec plusieurs
autres mesures de la structure des réseaux (p. ex. la distribution de degrés,
\cite{Poisot2014When}).

La structure des réseaux d'interactions peut décrire une variété d'autres
processus écologiques. Par exemple, le niveau trophique maximal est le nombre
maximal d'espèces séparant un producteur primaire d'un prédateur apical le long
de chaînes trophiques (\cite{Cohen1978Food}). Il représente une limite
biologique au transfert d'énergie dans les réseaux trophiques
(\cite{Williams2004Limits}). D'autre part, l'emboîtement décrit à quel point les
diètes des espèces spécialistes sont un sous-ensemble de celles des espèces
généralistes (\cite{Staniczenko2013Ghost}). L'emboîtement est fortement corrélée
à la modularité (c.-à-d. au niveau de division des espèces en modules
étroitement connectées) dans les réseaux antagonistes et mutualistes
(\cite{Fortuna2010Nestedness}) et diminue la stabilité des écosystèmes
(\cite{Okuyama2008Network}, \cite{Bastolla2009Architecture},
\cite{Thebault2010Stability}). Plusieurs propriétés des réseaux d'interactions
sont corrélées les unes avec les autres, ce qui suggère que la structure des
réseaux est influencée par un nombre plus restreint de variables écologiques.

\subsubsection{Une structure représentée par des distributions} 

Les distributions des caractéristiques des espèces peuvent nous informer de 
leur rôle écologique. Le degré d'une espèce est le nombre d'interactions qu'elle réalise avec d'autres
espèces dans le réseau, et la distribution des degrés est la distribution de
probabilité de ces degrés. De manière similaire, la distribution conjointe des
degrés est la distribution de probabilité conjointe pour les degrés entrants
(nombre d'interactions entrantes) et sortants (nombre d'interactions sortantes).

%%%%% La théorie de l'entropie maximale des réseaux trophiques

\section{Vers une théorie de l'entropie maximale des réseaux trophiques}

Cette thèse jette les bases d'une théorie permettant de reconstituer la
structure émergente des réseaux d'interactions entre espèces à partir d'un
nombre limité d'informations écologiques. Cette théorie repose sur la prémisse
selon laquelle les interactions entre espèces 

\subsection{Pourquoi avons-nous besoin de cette théorie?} 

- Prédire la structure de différents types de réseaux (écologie comme science prédictive)
- Fournir une explication quant à leur structure (compréhension approfondie des réseaux)
- Extension potentielle de METE

\subsection{Quels sont les fondements de la théorie?} 

- Les réseaux d'interactions entre espèces sont des systèmes complexes
- Complexité interne des réseaux 
- Les interactions entre espèces dépendent d'une multitude de mécanismes biologiques qui peuvent être ignorés - peut-être qu'il n'y a pas de mécanismes dominants
- Les interactions entre espèces sont probabilistes (et donc la structure aussi) - vision probabiliste des réseaux
- Non-indépendance des interactions nous incite à développer des méthodes pour prédire la structure sans?
- La structure des réseaux est contrainte par un ensemble restreint de variables
- Les mesures de la structure sont corrélées entre elles
- Principe d'entropie maximale qui maximise l'incertitude ou la complexité (prédictions non-biaisées) 
- Mécanismes sont dans les variables d'état (mécanismes implicites)

\subsection{Que permet-elle de prédire?} 

- Différentes réalisations (modèles) de la théorie 

\subsection{Quel est son champ d'application?} 

J'ai testé la théorie de l'entropie maximale des réseaux trophiques sur des
réseaux d'interactions entre espèces. Cependant, elle peut être adaptée à
d'autres niveaux d'organisation taxonomique en sélectionnant les informations
écologiques (entrées du modèle) et mesures prédites (sorties du modèle)
conformément au niveau d'organisation choisi. Dans cette thèse, je propose
également une façon d'étendre la théorie pour qu'elle puisse prédire
\textit{simultanément} les propriétés des réseaux d'interactions entre individus
et espèces. À défaut d'avoir développé une telle théorie élargie, cette thèse
explore comment différents types d'interactions écologiques varient selon le
niveau d'organisation taxonomique, alimentant une réflexion sur le champ
d'application de la théorie au-delà des interactions entre espèces. 

J'ai formulé les principes de base de la théorie de l'entropie maximale des
réseaux trophiques autour des interactions prédateurs-proies et
plantes-herbivores. Néanmoins, la théorie de l'entropie maximale des réseaux
trophiques peut être adaptée et testée sur d'autres types d'interactions qui
impliquent un échange direct d'information, de matière ou d'énergie (comme les
interactions hôtes-parasites ou plantes-pollinisateurs).

Cette distinction entre réseaux locaux et régionaux revêt d'une grande
importance dans l'élaboration de la théorie de l'entropie maximale des réseaux
trophiques, celle-ci étant mieux adaptée aux réseaux locaux. En effet, cette
théorie trouve son fondement dans la complexité des écosystèmes locaux. Les
écosystèmes complexes sont constitués, par définition, d'interactions impliquant
un échange direct d'information, de matière ou d'énergie
(\cite{Ladyman2013What}), ce qui se produit à l'échelle locale. Les réseaux
régionaux décrivent plutôt le potentiel que de tels échanges aient lieu, sans
nécessairement être réalisés. De plus, la variabilité des interactions locales
introduit une incertitude qui est non réductible (c.-à-d. qui ne diminue pas
avec davantage de données). Cela a pour conséquence de diminuer la prédictibilité des
interactions locales, rendant possiblement les réseaux locaux plus complexes. La
théorie de l'entropie maximale des réseaux trophiques vise précisément à
identifier la structure la plus complexe possible des réseaux locaux, dans la
limite de contraintes écologiques. Un chapitre complet de cette thèse est dédié
aux caractéristiques et sources d'incertitude des réseaux locaux et régionaux.


%%%%% Objectifs 

\section{Objectifs généraux de la thèse} 


%%%%% Chapitres

\section{Organisation de la thèse}

\subsection{Chapitre 1: Décrypter les réseaux d'interactions probabilistes} 

\subsection{Chapitre 2: Des modèles d'entropie maximale prédisant la structure des réseaux trophiques} 

\endinput
%%
%% End of file `01_introduction.tex'.
