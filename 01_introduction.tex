%%
%% This is file `01_introduction.tex',
%% generated with the docstrip utility.
%%

% Utilisez la macro de langue appropriée.
\francais   
\doublespacing
%ou
%%\anglais
\chapter*{Introduction}


%%%%% Mise en contexte 

\section{Les réseaux d'interactions entre espèces en tant que systèmes complexes}

Plusieurs phénomènes naturels sont régis par des systèmes complexes, qui forment
un ensemble cohérent d'éléments interagissant entre eux
(\cite{Rind1999Complexity}). Ces interactions impliquent des échanges directs
d'information, de matière ou d'énergie (\cite{Ladyman2013What}) dont l'effet
conjoint génère des propriétés émergentes de niveau supérieur
(\cite{Foote2007Mathematics}). Par exemple, le réseau de neurones dans un
cerveau humain orchestre des milliers de signaux électriques à chaque seconde
pour créer une expérience subjective du monde (\cite{Sporns2011Human}). De même,
une colonie de fourmis produit des comportements complexes tels que la
construction de nids et la défense du territoire par le biais d'individus
coopératifs qui s'échangent ressources et informations
(\cite{Bonabeau1999Swarm}). Caractéristiques fondamentales des systèmes
complexes, ces propriétés émergentes ne sont pas déductibles des éléments pris
individuellement (\cite{Nielsen2000Emergent}). Du désordre apparent des éléments
et de leurs interactions émerge donc une structure ordonnée qui est au cœur des
recherches sur les systèmes complexes. 

Pour faciliter l'étude des systèmes complexes, il est courant de les représenter
sous forme de réseaux, où les éléments forment des nœuds et les interactions,
des liens entre ces nœuds (\cite{Newman2003Structure}). Dans cette thèse, je me
suis intéressé aux réseaux d'interactions entre espèces, qui servent d'outil
pour mieux comprendre les systèmes écologiques complexes. Ces réseaux lient
chaque paire d'espèces qui interagissent entre elles, comme un prédateur et sa
proie, un parasite et son hôte ou une plante et son pollinisateur. Bien ancrée
dans la science des réseaux et la théorie des graphes, l'écologie des réseaux
fournit un cadre d'analyse pour étudier les propriétés émergentes des
écosystèmes, comme leur fonctionnement (p. ex. productivité primaire et cycle
des nutriments) et résilience face aux changements environnementaux
(\cite{Proulx2005Network}, \cite{McCann2007Protecting}, \cite{McCann2011Food},
\cite{Rooney2012Integrating}, \cite{Valiente-Banuet2019Species}). Pour ce faire,
elle a recours à des méthodes pour mesurer les interactions entre espèces (p.
ex. \cite{Jordano2016Sampling}) et analyser la structure émergente des réseaux
d'interactions (p. ex. \cite{Delmas2019Analysing}), nous permettant ainsi de
mieux comprendre les caractéristiques de ces systèmes complexes. 

Cette thèse jette les bases de la théorie de l'entropie maximale des réseaux
trophiques, qui constitue une approche novatrice en écologie des réseaux
permettant de prédire la structure émergente des réseaux complexes
d'interactions à partir d'un nombre limité d'informations écologiques. Avant
d'exposer cette théorie, il convient de passer en revue quelques
caractéristiques importantes des systèmes écologiques complexes, notamment en ce
qui a trait aux mécanismes régissant les interactions entre espèces et aux
mesures des interactions et de la structure des réseaux. Plus spécifiquement,
j'explique, dans les sous-sections suivantes, que les interactions entre espèces
sont le résultat de nombreux mécanismes écologiques souvent difficiles à
départager, que les interactions entre espèces sont intrinsèquement
probabilistes et que la structure émergente des réseaux d'interactions est
écologiquement et statistiquement contrainte. Ces trois observations
sous-tendent la théorie proposée en décrivant plusieurs aspects fondamentaux de
la complexité écologique.

\subsection{Quels mécanismes écologiques sous-tendent les interactions entre espèces?} 

Une communauté biologique est composée d'espèces différentes interagissant entre
elles dans un espace donné (\cite{Stroud2015Community}) lorsque les conditions
environnementales le permettent. Par exemple, les milieux forestiers tempérés du
Québec sont riches en espèces animales et végétales. À cause des changements
climatiques et de la fragmentation des habitats naturels, on y retrouve
maintenant des tiques \textit{Ixodes scapularis} infectées par la bactérie
\textit{Borrelia burgdorferi}, responsable de la maladie de Lyme
(\cite{Ogden2009Emergence}, \cite{Simon2014Climate}). Les tiques juvéniles se
nourrissent sur des petits mammifères, dont la souris à pattes blanches
(\textit{Peromyscus leucopus}) qui est un réservoir compétent de la bactérie
(\cite{Donahue1987Reservoir}). Les tiques adultes, quant à elles, se nourrissent
et s'accouplent sur les cerfs de Virginie (\textit{Odocoileus virginianus},
\cite{Lane1991Lyme}). Ces deux espèces (\textit{P. leucopus} et \textit{O.
virginianus}) s'alimentent à leur tour de glands de chêne
(\cite{McShea1993Variablea}, \cite{Elkinton1996Interactions},
\cite{Wolff1996Population}, \cite{McShea2000Influence}), ce qui peut expliquer
l'abondance élevée de tiques dans les chênaies (\cite{Ostfeld2006Climate}).
Comprendre l'écologie et l'épidémiologie de la maladie de Lyme nécessite une
compréhension approfondie des interactions entre espèces au sein des communautés
où la tique \textit{I. scapularis} est présente. Plus généralement, il est
important d'identifier les mécanismes écologiques sous-tendant les interactions
entre espèces pour mieux comprendre les conditions favorisant leur réalisation. 

\subsubsection{Les interactions impliquent différents types d'échange} 

Les interactions entre espèces peuvent avoir des effets positifs, négatifs ou
neutres sur les organismes impliqués, en fonction du type d'interaction. Les
interactions antagonistes, telles que les interactions hôtes-parasites,
prédateurs-proies et plantes-herbivores, profitent à une espèce au détriment de
l'autre. Les interactions hôtes-parasites procurent un habitat et une source
d'alimentation au parasite au prix du \textit{fitness} de l'hôte, alors que les
interactions prédateurs-proies et plantes-herbivores transfèrent l'énergie de la
ressource vers le consommateur, entraînant généralement la mort de la proie,
mais pas nécessairement celle de la plante. Les interactions mutualistes, telles
que les interactions plantes-pollinisateurs, profitent au contraire aux deux
espèces. Le pollinisateur se nourrit du nectar de la plante, qui en retour en
profite pour disperser son pollen à des fins de reproduction. Selon le niveau de
complexité représenté, les réseaux d'interactions entre espèces peuvent être
composés d'un seul type d'interaction (p. ex. les réseaux trophiques composés
exclusivement d'interactions trophiques prédateurs-proies et plantes-herbivores)
ou de plusieurs (p. ex. les réseaux multicouches composés d'interactions
antagonistes et mutualistes, \cite{Pilosof2017Multilayer}). 

Les interactions prédateurs-proies et plantes-herbivores ont une importance
particulière et historique en écologie. Les réseaux trophiques ont été
introduits il y a près d'un siècle, lorsque Charles Sutherland Elton (1900-1991)
a développé les concepts de chaînes et de niveaux trophiques
(\cite{Elton1927Animal}, \cite{Elton1958Ecology}). Plusieurs publications
marquantes en écologie, comme celles de \cite{May1972Will} sur la stabilité des
écosystèmes complexes et de \cite{Lotka1925Elements} et
\cite{Volterra1927Fluctuations} sur la dynamique des populations animales,
portent sur les interactions trophiques. Les interactions prédateurs-proies et
plantes-herbivores déterminent les flux de matière et d'énergie au sein des
écosystèmes. Elles nous permettent de mieux comprendre des phénomènes tels que
la régulation de la productivité primaire par cascade trophique
(\cite{Carpenter1987Regulation}), la stabilité des écosystèmes face à
l'extirpation d'espèces (\cite{Dunne2002Network}, \cite{Srinivasan2007Response},
\cite{Staniczenko2010Structural}) et la capacité des espèces à migrer et
s'adapter aux changements climatiques (\cite{Tylianakis2008Global},
\cite{Gilman2010Framework}) et à la perte d'habitats naturels
(\cite{Evans2013Robustness}). De plus, les données disponibles sur les réseaux
trophiques couvrent une diversité d'habitats à l'échelle mondiale
(\cite{Poisot2021Global}), ce qui nous permet de tester une variété de modèles
statistiques et d'hypothèses écologiques sur ces interactions. C'est pourquoi 
les réseaux trophiques occupent une place centrale dans cette thèse. 

Les mécanismes écologiques qui sous-tendent différents types d'interactions sont
essentiellement les mêmes, bien que leurs spécificités puissent varier. Par
exemple, toutes les interactions directes reposent sur des règles de
correspondance des traits (\cite{Poisot2015Species}). Cependant, les traits
pertinents varient d'un type d'interaction à un autre (p. ex. dentition d'un
prédateur pour une interaction trophique ou longueur de l'étamine pour une
interaction plante-pollinisateur). L'ensemble des interactions écologiques
directes est réalisé localement par des individus, permettant ainsi de discuter
communément de leurs mécanismes écologiques.

\subsubsection{Les interactions sont réalisées par des individus} 

Les individus sont à la base des interactions entre espèces puisque ce sont eux
qui effectuent les échanges d'information, de matière ou d'énergie. Par exemple,
c'est du sang d'un cerf de Virginie particulier que s'alimentera une tique
adulte. Les mécanismes sous-tendant les interactions entre espèces sont donc
basés sur ceux agissant à l'échelle individuelle (p. ex. le comportement d'une
proie qui détermine sa vulnérabilité à un prédateur,
\cite{Choh2012Predatorprey}). Il va sans dire que les réseaux d'interactions
entre individus (réseaux dont les nœuds sont des individus) représentent plus
exhaustivement les échanges ayant lieu au sein des écosystèmes. Ils sont
cependant difficiles à échantillonner, le peu de données disponibles sur les
interactions à cette échelle étant biaisées en faveur d'organismes plus simples
à observer (\cite{Guimaraes2020Structure}). Pour cette raison, les interactions
entre individus sont le plus souvent groupées en des niveaux d'organisation
supérieures, connectant différentes populations, espèces ou clades entre eux
(\cite{Elton1927Animal}). Les réseaux d'interactions entre espèces, qui lient
deux espèces lorsque certains de leurs individus interagissent, ne capturent
donc qu'une partie de la complexité écologique. Ils constituent néanmoins une
représentation adéquate du rôle écologique des espèces
(\cite{Delmas2019Analysing}) et des propriétés émergentes des écosystèmes
(\cite{Loreau2010Populations}, \cite{McCann2011Food},
\cite{Bascompte2013Mutualistic}, \cite{Gonzalez2020Scalingup}). Comprendre
comment les processus écologiques agissant à l'échelle individuelle se combinent
pour supporter les interactions entre espèces nous aide à mieux saisir les
déterminants des réseaux d'interactions entre espèces.

\subsubsection{Les interactions locales résultent de plusieurs mécanismes écologiques} 

Cette thèse distingue les interactions cataloguées à l'échelle régionale de
celles réalisées à l'échelle locale. Les réseaux régionaux (ou \textit{metaweb},
\cite{Pascual2006Ecological}) sont constitués d'interactions potentielles,
c'est-à-dire d'interactions biologiquement ou écologiquement faisables sans
nécessairement être réalisées localement (\cite{Tylianakis2017Ecological}). Par
exemple, les interactions trophiques régionales du cerf de Virginie incluent
l'ensemble des plantes et champignons dont il peut se nourrir dans différents
contextes. Une interaction régionale indique si deux espèces possèdent des
traits leur permettant d'interagir dans des conditions idéales. Les réseaux
régionaux sont donc un catalogue de l'ensemble des interactions observées (p.
ex. \cite{Maiorano2020Tetraeu}) ou prédites (p. ex. \cite{Strydom2022Food})
entre un groupe d'espèces données. Contrairement aux réseaux locaux, les réseaux
régionaux ne varient pas spatialement ni temporellement puisqu'ils incluent
l'ensemble des interactions locales entre un groupe d'espèces. Les mécanismes
biologiques sous-tendant les interactions régionales sont relativement peu
nombreux et plus facile à identifier, reposant presqu'exclusivement sur les
traits des espèces. 

Les réseaux locaux (\cite{Poisot2012Dissimilarity}), quant à eux, sont
constitués d'interactions observées ou réalisées à un endroit et moment
particuliers. Ce sont eux qui décrivent les échanges directs d'information, de
matière ou d'énergie réalisés par les individus à l'échelle locale. Étant donné
qu'une interaction locale ne peut être réalisée que si elle est biologiquement
faisable, les réseaux locaux sont des sous-réseaux du \textit{metaweb}
(\cite{Saravia2022Ecological}). Deux espèces interagissent localement si leurs
abondances (nombres d'individus) sont suffisamment élevées pour qu'elles
puissent se rencontrer (\cite{Canard2012Emergence}, \cite{Canard2014Empirical})
et que leurs traits (caractéristiques biologiques telles que la taille
corporelle) supportent une interaction lorsqu'elles se rencontrent
(\cite{Bolnick2011Why}, \cite{Gravel2013Inferring}, \cite{Stouffer2011Role}).
Par exemple, les habitudes alimentaires du cerf de Virginie varient localement
selon la disponibilité saisonnière en nourriture (\cite{Short1975Nutrition}) et
sa masse corporelle (\cite{Luna2013Influence}). Les conditions
environnementales, comme la température (\cite{Angilletta2004Temperature}), les
précipitations (\cite{Woodward2012Climate}) et l'utilisation du territoire
(\cite{Tylianakis2007Habitat}), influent également sur la réalisation des
interactions à l'échelle locale. Ces trois facteurs (abondances, traits,
conditions environnementales) contribuent à la variation spatiale et temporelle
des interactions locales (\cite{Poisot2015Species}). Les mécanismes sous-tendant
les interactions locales sont nombreux, et il peut être difficile d'identifier
lesquels contribuent le plus à la réalisation des interactions. 

\subsection{Comment mesurer une interaction entre espèces?} 

Outre le type d'interaction, la manière dont nous mesurons les interactions
entre espèces affecte également notre représentation des réseaux et notre
interprétation de leurs propriétés. Les mesures des interactions varient selon
leur niveau d'information écologique. Alors que les interactions binaires et
quantitatives évaluent respectivement la présence et le poids d'une interaction,
les interactions probabilistes mesurent son incertitude. Étant donné qu'il y
aura toujours une incertitude dans notre évaluation des interactions, celles-ci
sont intrinsèquement probabilistes. Cependant, le choix d'une mesure
d'interaction dépend des questions posées, de la qualité des données disponibles
et de notre capacité de quantifier cette incertitude. Les propriétés des réseaux
sont généralement calculées à partir d'une matrice d'adjacence (représentation
matricielle d'un réseau) contenant la valeur (binaire, quantitative ou
probabiliste) des interactions entre chaque paire d'espèces
(\cite{Delmas2019Analysing}), d'où l'importance du choix de mesure.

\subsubsection{Les interactions mesurent différents aspects des échanges de matière et d'énergie} 

Les interactions binaires indiquent si un lien est présent ou non dans un
réseau. La présence d'une interaction locale peut signifier que deux espèces ont
été observées en train d'interagir à l'intérieur des frontières du réseau. Par
exemple, observer une tique \textit{I. scapularis} sur le pelage d'un cerf de
Virginie \textit{O. virginianus} dans une forêt donnée témoigne d'une
interaction hôte-parasite entre ces deux espèces. Ne pas détecter cette
interaction suggère qu'elle ne se produit pas à cet emplacement particulier.
Cependant, une absence d'observation peut également être un faux négatif
(\cite{Bluthgen2010Why}, \cite{Chacoff2012Evaluating}, \cite{Stock2017Linear}),
c'est-à-dire une interaction non observée qui est en réalité réalisée. Étant
donné la difficulté d'échantillonner l'ensemble des interactions entre espèces
dans un lieu donné (\cite{Jordano2016Sampling}), ces faux négatifs sont
fréquents dans les jeux de données d'interactions entre espèces. Les méthodes
prédictives des interactions entre espèces (\cite{Strydom2021Roadmapa}) pallient
cette difficulté d'observation en prédisant lesquelles sont réalisées à
l'échelle locale. Il est cependant nécessaire de tester ces méthodes sur des
données empiriques fiables de présence-absence pour pouvoir utiliser les réseaux
prédits en toute confiance (\cite{Brimacombe2024Applying}). Bien qu'elles
contiennent peu d'information écologique comparativement aux autres mesures
d'interaction, les interactions binaires ont été plus fréquemment étudiées en
écologie (\cite{Pascual2006Ecological}, \cite{Delmas2019Analysing}), notamment
en raison de leur plus grande facilité d'échantillonnage
(\cite{Jordano2016Sampling}) et de prédiction (\cite{Strydom2021Roadmapa}).

Les interactions quantitatives mesurent quant à elles le poids d'une interaction
(\cite{Berlow2004Interaction}). Ce poids peut représenter le flux d'énergie ou
de biomasse (p. ex. \cite{Benke2001Food}, \cite{Post2002Long},
\cite{Bersier2002Quantitative}, \cite{Borrett2019Walk}), l'impact démographique
(p. ex. \cite{Paine1992Foodweb}, \cite{Kokkoris2002Variability},
\cite{Emmerson2004Predatorprey}) ou la fréquence d'interaction (p. ex.
\cite{Herrera1989Pollinator}, \cite{Montoya2003Food}) entre deux espèces. Par
exemple, le taux de cerfs de Virginie infectés par \textit{I. scapularis} est
une mesure du poids d'une interaction locale. Quel que soit l'effet mesuré, le
poids d'une interaction est souvent difficile à évaluer, surtout lorsque cet
effet est variable dans le temps ou qu'il suit une fonction non linéaire ou
densité-dépendante (\cite{Wootton2005Measurement}). De plus, puisque les
interactions binaires peuvent être déduites des interactions quantitatives, il
est nettement plus ardu d'échantillonner et de prédire ces dernières. Toutefois,
lorsqu'ils sont bien mesurés, les réseaux d'interactions quantitatives décrivant
les flux d'énergie ou de biomasse entre espèces semblent être une bonne
représentation des échanges directs se produisant au sein des systèmes
écologiques complexes. 

\subsubsection{Les interactions sont intrinsèquement probabilistes} 

Que nous ayons évalué la présence ou le poids d'une interaction, il est
important de quantifier et de bien comprendre l'incertitude de nos estimations
(c.-à-d. notre niveau de croyance quant à la présence ou valeur d'une
interaction). En raison des défis que posent l'échantillonnage des interactions
entre espèces (\cite{Jordano2016Sampling}), celles-ci possèdent presque toujours
une incertitude, surtout quand elles n'ont pas été directement observées ou
qu'elles ont été prédites à l'aide d'une méthode numérique. Cette incertitude
peut avoir plusieurs sources, incluant les erreurs de mesure des données servant
à prédire une interaction (p. ex. données de taille corporelle,
\cite{Gravel2013Inferring}), l'intervalle de confiance des paramètres du modèle
et l'existence de modèles prédictifs alternatifs (\cite{Simmonds2022Insights},
\cite{Simmonds2024Recommendations}). La variabilité des interactions locales
dans le temps et l'espace (\cite{Poisot2015Species}) introduit une source
additionnelle d'incertitude en limitant le transfert d'information d'un réseau à
un autre, par exemple lors de l'inférence d'interactions locales à partir d'un
réseau régional (\cite{Dansereau2023Spatially}). Documenter l'incertitude des
interactions entre espèces offre une vision plus juste de la fiabilité des
données d'interactions. 

Les interactions probabilistes mesurent la probabilité qu'une interaction soit
présente (\cite{Poisot2016Structure}). À l'échelle locale, elles représentent la
probabilité qu'une interaction soit réalisée à un endroit et moment
particuliers. Par exemple, si nous n'avons pas observé d'interaction entre un
cerf de Virginie et \textit{I. scapularis} dans une forêt donnée, nous pouvons
évaluer la probabilité que ces deux espèces interagissent localement à partir de
leurs abondances relatives (\cite{Canard2014Empirical}) ou de nos connaissances
sur le cycle de vie du parasite. À l'échelle régionale, les interactions
probabilistes représentent plutôt la probabilité qu'une interaction soit
biologiquement ou écologiquement faisable (\cite{Strydom2023Grapha}). Cette
probabilité d'interaction régionale est particulièrement utile lorsque deux
espèces n'ont jamais été observées en train d'interagir, mais qu'elles
pourraient possiblement le faire dans les bonnes circonstances. Qu'elles soient
locales ou régionales, les interactions probabilistes sont généralement évaluées
de manière indépendante. Cela peut introduire des biais lorsque nous propageons
l'incertitude des interactions locales vers la structure du réseau
(\cite{Poisot2016Structure}), à cause de l'impact que les interactions peuvent
exercer les unes sur les autres au sein d'un réseau local
(\cite{Golubski2011Modifying}, \cite{Ims2013Indirect}). Néanmoins, les
interactions probabilistes demeurent le moyen le plus courant de représenter
l'incertitude inhérente aux interactions entre espèces.

\subsection{Quelle est la structure émergente d'un réseau d'interactions entre espèces?} 

Les propriétés émergentes d'un système écologique complexe peuvent être évaluées
en mesurant la structure de son réseau d'interactions, c'est-à-dire en calculant
les propriétés de la matrice d'adjacence. Plusieurs mesures ont été proposées
pour analyser la structure des réseaux d'interactions binaires
(\cite{Delmas2019Analysing}), quantitatives (\cite{Bersier2002Quantitative}) et
probabilistes (\cite{Poisot2016Structure}). Plusieurs de ces mesures sont de
bons descripteurs de la complexité des écosystèmes (\cite{Landi2018Complexity}).
Elles peuvent décrire des valeurs uniques (p. ex. nombre d'espèces, nombre
d'interactions, connectance) ou des distributions relatives aux espèces (p. ex.
distribution de degrés) ou aux interactions (p. ex. distribution des poids
d'interaction). La structure des réseaux est associée à la dynamique
(\cite{Pascual2006Ecologicala}) et au fonctionnement des écosystèmes, qui
détermine les flux d'énergie, le cycle des nutriments et la régulation des
populations (\cite{McCann2011Food}, \cite{Thompson2012Food}). Malgré le nombre
élevé de mesures développées, plusieurs covarient entre elles ou sont
contraintes par des variables écologiques. Cela suggère que la structure des
réseaux puisse être déterminée par un nombre limité de variables. Ici, je
propose un aperçu des mesures de la structure d'un réseau, en mettant l'accent
sur celles décrivant la complexité des réseaux d'interactions entre espèces. 

\subsubsection{Plusieurs mesures décrivent la complexité écologique} 

Le niveau de complexité d'un écosystème peut être évalué à l'aide de différents
descripteurs (\cite{Landi2018Complexity}). Les plus simples d'entre eux sont le
nombre d'espèces (p. ex. \cite{May1972Will}) et le nombre d'interactions (p. ex.
\cite{Okuyama2008Network}). Ces deux mesures déterminent la connectance d'un
réseau, soit la proportion des interactions possibles qui sont réalisées. La
connectance est un descripteur clé de la structure des réseaux d'interactions
entre espèces (\cite{Martinez1992Constant}) et de sa complexité (p. ex.
\cite{Rozdilsky2001Complexity}), une connectance élevée indiquant que les
espèces sont davantage connectées. Elle a été associée à plusieurs propriétés
émergentes des écosystèmes, notamment à l'invasibilité des habitats par des
espèces envahissantes, qui ont plus de difficulté à intégrer un réseau fortement
connecté (\cite{Smith-Ramesh2017Global}). La stabilité des réseaux trophiques
face à l'extirpation d'espèces augmente également avec la connectance puisqu'un
prédateur peut plus facilement s'adapter en cherchant d'autres sources de
nourriture après la disparition d'une de ses proies (\cite{Dunne2002Network}).
Toutefois, on observe la relation inverse dans les réseaux mutualistes lorsque
l'on tient compte de la dépendance mutuelle des espèces et de l'importance
relative de chaque interaction (\cite{Vieira2015Simple}). Ces descripteurs
capturent tous un aspect important de la complexité écologique, soit sa faible
prédictibilité (\cite{Strydom2021Svd}). 

Un autre descripteur de la complexité d'un écosystème est la distribution de
degrés (\cite{Landi2018Complexity}). Le degré d'une espèce est le nombre
d'interactions qu'elle réalise avec les autres espèces dans son réseau, et la
distribution de degrés est la distribution de probabilité de ces degrés. Dans
les réseaux dirigés, où une interaction se fait dans un sens particulier (p. ex.
prédateurs vers proies), la distribution conjointe des degrés est la
distribution de probabilité conjointe décrivant les degrés entrants (nombre
d'interactions entrantes) et sortants (nombre d'interactions sortantes). La
distribution de degrés est une mesure écologiquement informative, nous éclairant
sur le rôle écologique des espèces (\cite{Sole2001Complexity},
\cite{Dunne2002Network}, \cite{Memmott2004Tolerance}), les mécanismes
sous-jacents aux interactions entre espèces (\cite{Williams2011Biology}) et
l'assemblage des réseaux (\cite{Vazquez2005Degree}).

D'autres mesures ont été proposées pour décrire la complexité d'un réseau
écologique. Par exemple, \cite{Strydom2021Svd} suggèrent l'utilisation de
l'entropie SVD (de l'anglais \textit{Singular Value Decomposition}) comme
descripteur de la complexité interne d'un réseau. Cette mesure, basée sur
l'entropie de Shannon (\cite{Shannon1948Mathematical}), quantifie l'information
contenue dans les rangs de la matrice d'adjacence. Elle représente une façon
plus robuste de quantifier la complexité d'un réseau que les autres mesures
mentionnées. L'entropie SVD a été utilisée comme mesure de complexité dans
l'application de la théorie de l'entropie maximale des réseaux trophiques
présentée dans cette thèse. 

\subsubsection{Les mesures de la structure sont corrélées entre elles} 

La structure des réseaux est caractérisée par des mesures qui covarient
fortement les unes avec les autres, limitant ainsi l'information écologique
unique qu'elles procurent. Par exemple, un réseau contenant plus d'espèces
réalise un plus grand nombre d'interactions, ces deux mesures ayant été liées
par une loi de puissance (\cite{Brose2004Unified}, \cite{Riede2010Chapter}).
Également, la connectance détermine plusieurs autres mesures de la structure des
réseaux, comme la distribution de degrés (\cite{Poisot2014When}). D'autres
mesures de la structure des réseaux covarient entre elles, comme l'emboîtement
et la modularité dans les réseaux antagonistes et mutualistes
(\cite{Fortuna2010Nestedness}). L'emboîtement décrit à quel point les diètes des
espèces spécialistes sont un sous-ensemble de celles des espèces généralistes
(\cite{Staniczenko2013Ghost}), alors que la modularité représente le niveau de
division des espèces en modules étroitement connectées. Un réseau fortement
emboîté, et donc peu modulaire, a tendance à être moins stable écologiquement
(\cite{Okuyama2008Network}, \cite{Bastolla2009Architecture},
\cite{Thebault2010Stability}). Plusieurs propriétés des réseaux d'interactions
sont corrélées les unes avec les autres, ce qui suggère que la structure des
réseaux puisse être déterminée par quelques mesures seulement. Une théorie
permettant d'unifier et de réconcilier ces différentes mesures dans un cadre
commun fait actuellement défaut.

\subsubsection{La structure est contrainte par des variables écologiques} 

Au-delà des corrélations observées entre les propriétés des réseaux, des
principes écologiques contraignent la structure des réseaux. Ces valeurs
imposent une limite biologique à la configuration que peut prendre un réseau
d'interactions. Par exemple, l'obligation pour les espèces de se nourrir impose
une limite inférieure au nombre d'interactions réalisées dans un réseau
(\cite{MacDonald2020Revisiting}). Le modèle des liens flexibles développé par
\cite{MacDonald2020Revisiting} prédit de manière réaliste le nombre
d'interactions dans les réseaux trophiques en utilisant le nombre d'espèces
comme contrainte écologique. Prédire le nombre d'interactions constitue une
bonne première étape vers la reconstitution des réseaux trophiques à partir de
la richesse en espèces lorsque les données d'interactions sont rares
(\cite{Strydom2021Roadmapa}). En effet, le nombre d'espèces et le nombre
d'interactions peuvent à leur tour être utilisés comme contraintes pour dériver
la distribution de degrés avec le principe d'entropie maximale
(\cite{Williams2011Biology}). Également, le niveau trophique maximal est le
nombre maximal d'espèces séparant un producteur primaire d'un prédateur apical
le long de chaînes trophiques (\cite{Cohen1978Food}). Il représente une limite
biologique au transfert d'énergie dans les réseaux trophiques
(\cite{Williams2004Limits}). Tenir compte de ces contraintes peut faciliter la
prédiction de la structure des réseaux d'interactions en réduisant le nombre de
configurations écologiquement possibles.  

%%%%% La théorie de l'entropie maximale des réseaux trophiques

\section{Vers une théorie de l'entropie maximale des réseaux trophiques}

Cette thèse pose les fondements d'une théorie des réseaux trophiques prédisant
leur structure émergente à partir d'un nombre limité de variables d'état
(variables écologiques caractérisant la structure des réseaux). Cette théorie
utilise ces variables écologiques pour contraindre des distributions de
probabilité et identifier celle qui est la moins biaisée (c.-à-d. celle qui ne
fait aucune supposition sur la forme de la distribution au-delà des informations
fournies par les contraintes écologiques). Ces distributions peuvent
caractériser différents aspects de la structure des réseaux trophiques (p. ex.
leur distribution de degrés). Selon les variables utilisées et les mesures
prédites, différentes versions de la théorie peuvent être développées, celles
présentant le plus fort soutien empirique étant retenues au détriment des
autres.

\subsection{Quels sont les fondements de la théorie?} 

\subsubsection{Une théorie basée sur le principe d'entropie maximale} 

Le principe d'entropie maximale (MaxEnt) est une méthode d'inférence permettant
d'identifier la distribution de probabilité la moins biaisée possible
considérant un ensemble de contraintes sur la forme de la distribution
(\cite{Jaynes1957Informationa}, \cite{Jaynes1957Information}). MaxEnt stipule
que notre meilleure estimation d'une distribution de probabilité est celle qui
maximise son entropie d'information (c.-à-d. une mesure de l'incertitude d'un
évènement aléatoire comme l'entropie de Shannon,
\cite{Shannon1948Mathematical}), compte tenu des contraintes utilisées. Par
exemple, si nous ne connaissons que la moyenne d'une distribution, nous pouvons
montrer que la distribution exponentielle est celle qui maximise son entropie
(\cite{Frank2011Simple}). Cette distribution unique est dérivée uniquement à
partir de nos connaissances préalables sur le système étudié, sans faire appel à
des suppositions additionnelles sur la forme de la distribution. MaxEnt ne
repose sur aucun mécanisme explicite et ne nécessite pas d'ajustement aux
données. La distribution de probabilité d'entropie maximale est plutôt dérivée
directement à l'aide de la méthode des multiplicateurs de Lagrange (une méthode
d'optimisation mathématique), dont le fonctionnement est décrit plus loin. Bien
qu'il n'y ait aucune garantie que cette distribution soit la mieux soutenue
empiriquement, elle est celle qui représente le plus fidèlement nos
connaissances préalables (c.-à-d. celle qui minimise les biais causés par des
suppositions infondées). 

\subsubsection{La théorie de l'entropie maximale de l'écologie appliquée aux réseaux} 

La théorie de l'entropie maximale de l'écologie (METE) prédit de nombreuses
distributions d'intérêt en macroécologie à l'aide de MaxEnt
(\cite{Harte2011Maximum}). Différentes versions de cette théorie peuvent être
développées en fonction des variables d'état choisies. La version ANSE
(\cite{Harte2008Maximum}, \cite{Harte2014Maximuma}) repose sur quatre variables
d'état : la superficie $A_0$, le nombre total d'individus $N_0$, le nombre
d'espèces $S_0$ et l'énergie métabolique totale de la communauté $E_0$. Les
ratios entre ces quatre variables (p. ex. l'abondance moyenne des espèces)
contraignent les distributions prédites par la théorie (p. ex. la distribution
de l'abondance des espèces, \cite{Brummer2019Derivations}). Les prédictions de
la version ANSE de METE sont en grande partie conformes aux données empiriques
(\cite{Harte2011Maximum}, \cite{McGlinn2015Exploring}), surtout pour les
écosystèmes moins perturbés (\cite{Newman2020Disturbance}). 

La théorie de l'entropie maximale des réseaux trophiques est une application de
METE aux réseaux d'interactions entre espèces. Elle permet de prédire plusieurs
mesures de la structure des réseaux à partir d'un nombre limité de variables
d'état les contraignant. Un modèle MaxEnt, en phase avec les objectifs et
méthodes de cette théorie, a été développé par \cite{Williams2011Biology} pour
prédire la distribution de degrés de réseaux mutualistes et antagonistes. Pour
ce faire, ces auteurs ont utilisé le nombre total d'espèces et d'interactions
comme variables d'état, contraignant la distribution de degrés par le nombre
moyen d'interactions par espèces. La théorie présentée dans cette thèse applique
ces mêmes principes aux réseaux d'interactions trophiques, tout en étendant la
portée des prédictions fournies par MaxEnt.

\subsubsection{Une théorie en plein développement} 

Elle repose sur la prémisse selon
laquelle la structure des réseaux (propriétés macroscopiques des systèmes
complexes), résulte d'une agrégation des processus écologiques se produisant à
l'échelle de l'interaction (\cite{Frank2009Common}). 


est enracinée dans la vision
probabiliste des réseaux d'interactions entre espèces. 


La distribution la moins biaisée est trouvée à l'aide d'une méthode
mathématique qui ne requiert pas d'ajustement aux données. 


\subsection{Pourquoi avons-nous besoin de cette théorie?} 

La théorie de l'entropie maximale des réseaux trophiques explique la structure
des réseaux d'interactions entre espèces en montrant qu'elle est écologiquement
et statistiquement contrainte. À l'aide de principes statistiques simples, elle
permet de prédire quantitativement et de manière cohérente différents aspects de
la structure des réseaux, sans avoir recours à des méthodes d'ajustement de
modèles aux données. Ces prédictions dépendent des variables écologiques
utilisées pour contraindre la structure des réseaux. Identifier les variables
qui génèrent les meilleures prédictions nous permet de mieux comprendre les
facteurs qui déterminent sur la structure des réseaux. Ainsi, en tant que théorie
scientifique, elle permet à la fois de générer des prédictions vérifiables sur
la structure des réseaux et d'identifier les mécanismes écologiques qui les
sous-tendent le plus fortement. 

\subsubsection{Prédire la structure des réseaux trophiques} 

L'écologie doit s'élever au rang d'une science prédictive pour que nous
puissions mieux anticiper les effets des changements environnementaux à venir
(\cite{Evans2012Predictive}) et informer adéquatement les processus de prises de
décision environnementale (\cite{Clark2001Ecological}). Étant donné la
difficulté d'échantillonner localement les interactions entre espèces
(\cite{Jordano2016Sampling}) et les bias d'échantillonnage présents dans les
jeux de données existants (\cite{Aguiar2019Revealing}, \cite{Poisot2021Global}),
nous sommes encore loin d'une compréhension exhaustive des interactions entre
espèces (\cite{Hortal2015Seven}). Les modèles prédictifs peuvent être utilisés
pour combler partiellement ce manque de données et faciliter l'étude des réseaux
d'interactions à différentes échelles spatiales et temporelles. Prédire en
premier lieu la structure des réseaux permet de générer de meilleures
prédictions sur les interactions entre espèces (\cite{Strydom2021Roadmapa}). La
plupart des modèles développés prédisant la structure des réseaux se basent sur
des mécanismes particuliers, comme ceux liés à la correspondance des traits (p.
ex. correspondance entre la niche écologique des espèces,
\cite{Williams2000Simple}) ou la dispersion et démographie des espèces
(\cite{Canard2012Emergence}). Cependant, puisque les mécanismes sous-jacents aux
réseaux d'interactions entre espèces sont nombreux, il peut être hasardeux de
construire une théorie autour d'un mécanisme particulier. De plus, la
disponibilité limitée de plusieurs types de données écologiques peut rendre
difficile l'utilisation des modèles prédictifs (\cite{Strydom2021Roadmapa}). 

En étant basée sur un petit nombre de variables écologiques, la théorie de
l'entropie maximale des réseaux trophiques facilite la prédiction de la
structure des réseaux d'interactions en réduisant la quantité d'informations
requises. Les seules données d'entrée du modèle sont les valeurs des variables
d'état. Par exemple, une version de la théorie prédisant la structure des
réseaux à partir du nombre d'espèces et du nombre d'interactions n'a besoin que
de ces deux variables comme entrées du modèle. Elle peut prédire simultanément
différents aspects de la structure des réseaux, comme la distribution de degrés
ou la matrice d'adjacence. Elle le fait en maximisant l'incertitude des 
distributions de probabilité prédites, c'est-à-dire en trouvant la distribution 
la plus uniforme possible qui respecte les contraintes écologiques choisies. 

\subsubsection{Identifier les mécanismes écologiques sous-jacents aux réseaux} 

Cette théorie
fournit une explication simple aux propriétés des réseaux trophiques, sans pour
autant être basée sur des mécanismes écologiques explicites. Ces
variables, qui peuvent différer selon la version de la théorie, capturent les
mécanismes écologiques sous-jacents aux réseaux (\cite{White2012Characterizing},
\cite{McGill2010Mechanisms}).  

Elle constitue une explication approfondie et
cohérente de la structure des réseaux. Les déviations du modèle peuvent être
utilisés pour inférer des mécanismes écologiques (\cite{Harte2014Maximuma}).

Comme pour les réseaux non écologiques, une riche littérature scientifique a
cherché à comprendre les mécanismes qui façonnent les réseaux écologiques. Les
réseaux sont-ils principalement façonnés par la stochasticité et les règles
topologiques ? Quel est le rôle des processus biologiques dans la structuration
des réseaux écologiques ?

Les écarts par rapport aux distributions MaxEnt peuvent indiquer que l'ensemble
de contraintes utilisé n'est pas adéquat, ou que des mécanismes non pris en
compte par ces contraintes sont importants (\cite{Harte2014Maximum}).

\subsection{Quels sont ses champs d'application?} 

J'ai testé la théorie de l'entropie maximale des réseaux trophiques sur des
réseaux d'interactions entre espèces. Cependant, elle peut être adaptée à
d'autres niveaux d'organisation taxonomique en sélectionnant les informations
écologiques (entrées du modèle) et mesures prédites (sorties du modèle)
conformément au niveau d'organisation choisi. Dans cette thèse, je propose
également une façon d'étendre la théorie pour qu'elle puisse prédire
\textit{simultanément} les propriétés des réseaux d'interactions entre individus
et espèces. À défaut d'avoir développé une telle théorie élargie, cette thèse
explore comment différents types d'interactions écologiques varient selon le
niveau d'organisation taxonomique, alimentant une réflexion sur le champ
d'application de la théorie au-delà des interactions entre espèces. 

J'ai formulé les principes de base de la théorie de l'entropie maximale des
réseaux trophiques autour des interactions prédateurs-proies et
plantes-herbivores. Néanmoins, la théorie de l'entropie maximale des réseaux
trophiques peut être adaptée et testée sur d'autres types d'interactions qui
impliquent un échange direct d'information, de matière ou d'énergie (comme les
interactions hôtes-parasites ou plantes-pollinisateurs).

Cette distinction entre réseaux locaux et régionaux revêt d'une grande
importance dans l'élaboration de la théorie de l'entropie maximale des réseaux
trophiques, celle-ci étant mieux adaptée aux réseaux locaux. En effet, cette
théorie trouve son fondement dans la complexité des écosystèmes locaux. Les
écosystèmes complexes sont constitués, par définition, d'interactions impliquant
un échange direct d'information, de matière ou d'énergie
(\cite{Ladyman2013What}), ce qui se produit à l'échelle locale. Les réseaux
régionaux décrivent plutôt le potentiel que de tels échanges aient lieu, sans
nécessairement être réalisés. De plus, la variabilité des interactions locales
introduit une incertitude qui est non réductible (c.-à-d. qui ne diminue pas
avec davantage de données). Cela a pour conséquence de diminuer la prédictibilité des
interactions locales, rendant possiblement les réseaux locaux plus complexes. La
théorie de l'entropie maximale des réseaux trophiques vise précisément à
identifier la structure la plus complexe possible des réseaux locaux, dans la
limite de contraintes écologiques. Un chapitre complet de cette thèse est dédié
aux caractéristiques et sources d'incertitude des réseaux locaux et régionaux.


Elle peut prédire
simultanément différents aspects de la structure des réseaux à différents
niveaux d'organisation (p. ex. à l'échelle de l'espèce et du réseau) à partir de
règles topologiques simples.

La théorie peut également être adaptée aux autres types de réseaux (p. ex. aux
réseaux d'interactions hôtes-parasites ou plantes-pollinisateurs). 

%%%%% Objectifs 

\section{Objectifs généraux de la thèse} 


%%%%% Chapitres

\section{Organisation de la thèse}

\subsection{Chapitre 1: Décrypter les réseaux d'interactions probabilistes} 

\subsection{Chapitre 2: Des modèles d'entropie maximale prédisant la structure des réseaux trophiques} 

\endinput
%%
%% End of file `01_introduction.tex'.
